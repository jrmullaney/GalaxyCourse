   
\documentclass[11pt]{article}
\renewcommand{\baselinestretch}{1.05}
\usepackage{amsmath,amsthm,verbatim,amssymb,amsfonts,amscd, graphicx}
\usepackage{graphics}
\topmargin0.0cm
\headheight0.0cm
\headsep0.0cm
\oddsidemargin0.0cm
\textheight23.0cm
\textwidth16.5cm
\footskip1.0cm

 \begin{document}
 
\title{Lecture 11:\\AGN discovery and observed properties}
\author{Dr. James Mullaney}
\maketitle

\section{Introduction}
Up to this point in the course we have considered ``normal'' galaxies
whose light is produced exclusively by stars. There is, however, a
class of galaxy in which a significant proportion of the total energy
output is being emitted instead by material falling into a central
supermassive black hole. Because of the extreme amounts of light being
produced at their centres, these are known as ``Active Galactic
Nuclei'' (AGN) and are currently thought to have had a major influence
on how today's galaxies have formed and evolved. In this lecture,
we'll cover how AGNs were first discovered and discuss their main
properties.

\section{The discovery of AGN}
The earliest known account of the study of an AGN is that of Fath in
1908, in which he obtained the spectra of a number of ``spiral
nebulae'' (this was before Hubble had demonstrated that such nebulae
were, indeed, galaxies external to our own). Fath noted that while
most of the spiral nebulae in his sample displayed spectra consistent
with a population of unresolved stars, one -- NGC~1068 -- had a
``composite spectrum, showing both bright $[$emission$]$ and
absorption lines''. In the years that followed, a few other observers
noted the peculiar ``planetary nebulae-like'' spectra of some other
spiral galaxies (notably NGC~4051 and NGC~4151), but little more was
made of these somewhat rare systems.

It wasn't until 1943 that AGNs were studied in any kind of systematic
way. At this time, Carl Seyfert obtained the spectra of six nearby
galaxies that all showed evidence of a bright, stellar-like
nucleus. Seyfert noted that all of these galaxies showed evidence of
strong emission lines superimposed on an otherwise normal galaxy
spectrum. Most interestingly, however, was that in some cases (e.g.,
NGC~3516, NGC~7469), the permitted Hydrogen emission lines (i.e.,
H$\alpha$ and H$\beta$ were extremely broad (i.e., they covered a wide
range of wavelengths). Seyfert attrbuted these broad lines to Doppler
shifts, corresponding to velocities of up to 8,500~${\rm km s^{-1}}$.
By contrast, in other cases (e.g., NGC~1068) the permitted Hydrogen
emission lines had similar widths as the comparatively narrow
(corresponding to up to $\sim3000~{\rm km s^{-1}}$) forbidden emission
lines. \footnote{Unfortunately, Carl Seyfert died in a car accident
  before his important work on AGN was fully appreciated. He did,
  however, present the daily weather bulletin the local news whilst
  employed as full-time astronomy professor in Tenessee, which makes
  him awesome.}  As we shall see, the differences in the widths of the
permitted emission lines continue to be used today to group AGNs into
two different types:
\begin{itemize}
\item Type 1 AGNs: Display broad ($>2000~{\rm km~s^{-1}}$) permitted
  emission lines (e.g., H$\alpha$, H$\beta$), but narrow (typically
  $<1000~{\rm km~s^{-1}}$) forbidden (e.g., $[$O {\sc iii}$]$, $[$N
  {\sc ii}$]$) emission lines in their optical spectra.
\item Type 2 AGNs: Display permitted and forbidden emission lines with
  roughly the same comparatively narrow (typically
  $<1000~{\rm km~s^{-1}}$) widths.
\end{itemize}

In addition to differences in their emission lines, Type 1 and Type 2
AGNs also tend to have different continuum features. Type 1 AGNs
typically have very blue continua that are quite devoid of absorption
lines. Type 2 AGNs, by contrast, have much redder continua which
display evidence of stellar absorption features. Indeed, the spectra
of Type 2 AGNs typically look like those of normal galaxies, but with
strong narrow permitted and forbidden emission lines.

\section{AGN in the radio era}
While Seyfert's work is now widely regarded as as groundbreaking in
terms of the first systematic study of AGNs, it was largely ignored by
other astronomers at the time. In fact, it took until the development
of radio astronomy before AGNs would start to more widely studied by
astronomers.

During the Second World War there was a lot of research conducted in
the areas of radio communication and radar. Once the war ended, some
of the engineers and scientists that had worked on radio
communications during the war turned their attention to trying to
determine where the persistant background ``noise'' that was picked up
by their receivers came from. While it was soon realised that a
significant fraction of radio noise came from the Sun, there remained
significant ``noise'' from other non-terrestrial sources. The problem,
however, was that the angular resolution of early radio telescopes was
extremely poor (i.e., many degrees on the sky), making it very hard to
pinpoint the positions of the radio sources to better than a few tens of
degrees. With this level of resolution, radio astronomers were only
able to locate a bright sources to be within an entire constellation,
so the first radio sources were referred to by the constellation in
which they were detected (e.g., Cygnus, Cassiopeia).

By the mid 1940s, radio astronomy technology had developed
sufficiently to begin to pinpoint the positions of radio sources
beyond the solar system. This was enabled by the development of radio
interferometry techniques, the first of which exploited the
``sea-cliff'' technique. This involved situating a radio reciever at
the edge of a sea-facing cliff. As the radio source rose above the
horizon, some of its radio waves would travel directly toward the
receiver, while others would be reflected off the surface of the sea
before reaching the receiver. This effectively creates two radio
sources that interfere with each other, creating a radio interference
pattern. The end result is an interferometer with a baseline twice the
height of the cliff, greatly increasing the angular resolution of the
telescope without the need for connecting cables and correlators
between two separate receivers.

Using this technique, early radio astronomers began to be able to
pinpoint the position of bright radio sources on the sky. This
revealed that the very bright source in Cygnus previously detected in
low-resolution studies was compact, extending less than 8 arcminutes
on the sky. With such levels of angular precision, by 1949 astronomers
were able to associate the radio sources with those detected at
optical wavelengths. In doing so, it transpired that Cygnus A was, in
fact, associated with galaxy merger at a distance of 252~Mpc, meaning
it has a total radio luminosity of $>10^{45}~{\rm ergs~s^{-1}}$ --
more than the luminosity of all the stars in the merging galaxies
combined.  With the ratio of radio to optical luminosity of Cygnus A
so much higher than that of a star, it must mean that the radio
emission must be coming from another type of source. Later, in the
1960s, it was realised that this alternative source of radio energy
was associated with galaxies showing optical properties the same as
those reported by Seyfert around 20 years earlier - i.e., AGNs.

\section{AGN classification}
With the realisation that many of the brightest radio sources in the
sky were associated with AGNs, interest in these objects increased
dramatically. However, with some of Seyfert's original optical AGNs
{\it not} associated with bright radio sources, it was soon
appreciated that not all AGNs were powerful radio emitters. Instead,
it seemed that in addition to the Type 1/Type 2 classification, there
should also be a radio ``loud'' and radio ``quiet''
distinction.\footnote{``Loud'' and ``Quiet'' typically refer to the
  ratio of radio luminosity to optical luminosity of an AGN, rather
  than the absolute radio luminosity. Despite being adjectives of
  sound volume, this is referred to as a measure of the radio {\it
    loudness} of a source, rather than its radio ``volume'' (since the
  latter has a double meaning in physics).} Further, both radio loud
and radio quiet AGNs can either be Type 1 or Type 2 AGNs; in other
words, there is no obvious connection between an AGN's optical
classification and its radio classification. So, there
are:
\begin{itemize}
\item Radio loud:
  \begin{itemize}
    \item Type 1: Broad Line Radio Galaxy
    \item Type 2: Narrow Line Radio Galaxy
  \end{itemize}
\item Radio quiet:
  \begin{itemize}
    \item Type 1: Seyfert 1
    \item Type 2: Seyfert 2
  \end{itemize}
\end{itemize}
Added to this, however, is another completely arbitrary distinction
based on the {\it optical luminosity} of AGNs, with the most optically
luminous AGNs known as {\it Quasars}. This unfortunate situation has
arisen because quasars are so bright that when they were first
discovered, their host galaxies could not be seen due to the ``glare''
of the light from the AGN. Indeed, they looked like very bright stars
but with peculiar spectra (hence their name, which refers to
``quasi-stellar objects''). As a consequence, their connection to
Seyfert AGNs (which are all in easily-seen galaxies) was not realised
at first, and by the time it was, the name had stuck. So, now we have:
\begin{itemize}
\item Radio loud:
  \begin{itemize}
  \item Type 1:
    \begin{itemize}
    \item High optical luminosity: Radio Loud Quasar
    \item Low optical luminosity: Broad Line Radio Galaxy
    \end{itemize}
  \item Type 2:
    \begin{itemize}
    \item High optical luminosity: Radio Loud Type 2 Quasar
    \item Low optical luminosity: Narrow Line Radio Galaxy
    \end{itemize}
  \end{itemize}
\item Radio quiet:
  \begin{itemize}
  \item Type 1:
    \begin{itemize}
    \item High optical luminosity: Radio Quiet Quasar
    \item Low optical luminosity: Seyfert 1
    \end{itemize}
  \item Type 2:
    \begin{itemize}
    \item High optical luminosity: Radio Quiet Type 2 Quasar
    \item Low optical luminosity: Seyfert 2
    \end{itemize}
  \end{itemize}
\end{itemize}
What a nightmare!

\section{AGN Unification}
As the tale of the discovery of quasars indicates, as the different
types of AGNs were being discovered, it wasn't at all clear to
astronomers that they were, indeed, the same type of object. Are, for
example, Type 1 AGNs completely distinct from Type 2 AGNs, or are they
somehow manifestations of the same object?

The anwer to this specific question came in the mid 1980s as a result
of the study of Type 2 AGNs in polarised light. When light is
reflected, it becomes polarised, so when it was discovered that Type 2
AGNs showed evidence of Type 1-like broad emission lines in their
optical spectra, it was interpreted that what was being detected was
the reflected light from an otherwise ``hidden'' region. This led to
the suggestion that the region responsible for the broad lines in Type
1 AGNs -- i.e., the {\it broad line region} (BLR) -- was indeed
present in Type 2 AGNs, but was simply hidden from our view. However,
the obscuring material could not be blocking {\it all} the light from
this BLR, since some of it was being observed in polarised light. The
interpretation was that surrounding the BLR of {\it all} AGNs is a
``torus'' of obscuring dust; in the case of Type 1 AGNs, we're looking
down the hole of the torus, directly at the BLR, whereas in the case
of Type 2 AGNs, the torus is side-on and thus blocks our line-of-sight
to the BLR.

Today, this ``dusty torus'' is a major component of the ``Unified AGN
model'', which explains the observed properties of AGNs in terms of a
specific geometry:
\begin{itemize}
\item At the very centre of an AGN is a supermassive black hole.
\item Surrounding the supermassive black hole is an accretion disk
  formed from gas and dust sprilling toward the black hole. As this
  material falls toward the black hole, it travels faster and
  increases in temperature due to friction. As it heats up, it
  radiates this heat in the form of light, sometimes outshining its
  entire host galaxy (as is the case of quasars).
\item Slightly further out from the accretion disk is the BLR. The BLR
  consists of clouds of gas that are being illuminated and ionised by
  the light from the accretion disk. Being close to the black hole,
  they orbit it at high velocities, hence the emission lines they
  produce are strongly doppler broadened. The are also dense, and
  so only emit permitted emission lines.
\item Surrounding the accretion disk and BLR is the dusty torus which,
  in the case of Type 2 AGNs, blocks out the light from {\it both} the
  accretion disk and the BLR.
\item Beyond the dusty torus is the so-called {\it narrow-line region}
  (NLR). This is simply ambient gas in the host galaxy that is being
  illuminated and ionised by the AGN. This gas is far more rarified
  than that in the dense BLR, and hence emits in both permitted and
  forbidden emission lines. Since it is larger than the dusty torus,
  we observe emission from the NLR in both Type 1 and Type 2 AGNs.
\end{itemize}
Within the unified model, Type 1 and Type 2 AGNs are manifestations of
{\it the same} object. The reason for their different appearances is
simply due to the orientation of the dusty torus relative to our line
of sight.

\section{The multiwavelength continua of AGNs}
An important consequence of the structure of an AGN is that they can
emit strongly at almost all observable wavelengths, with each
component of the unified model emitting at different wavelengths. With
its high temperatures, the accretion disk emits strongly at optical
through to ultraviolet wavelengths which can be seen directly in the
case of Type 1 AGNs. Furthermore, some of the gas surrounding the
accretion disk is heated to millions of degrees, which upscatters
ultraviolet photons to X-ray energies. As such, AGNs are the strongest
sources of X-rays in the Universe. Further out, the dusty torus is
heated by light from the accretion disk which is then re-radiated at
infrared wavelengths. At even longer wavelengths, in radio-loud AGNs,
magnetically entrained jets propogate from the central regions of the
accretion disk, emitting powerful synchrotron radiation as they do so.

\section{AGN identification}
Since AGNs are strong emitters of light spanning almost the whole
observable electromagnetic spectrum, many different techniques
emploiting different wavelengths of light have been developed to
identify AGNs among ``normal'' galaxies. In this section, we'll
consider some of the most commonly used approaches to identify
AGNs. It is important to note, however, that the various techniques
come with their own biases; for example, some are only sensitive to a
particular type of AGN (e.g., optical Type 1 or Type 2), whereas
others will only identify particularly luminous AGNs.

\subsection{Optical selection}
With Type 1 AGNs displaying strong blue optical continua, we can use
this property to identify AGNs using broad band optical photometry
(i.e., using imaging taken with different filters). Basically, what
astronomers do is search for point like sources with blue rest-frame
optical colours. Of course, this has to be able to account for
K-correction, so different combinations of filters are used to
identify the blue continua indicative of quasars at different
redshifts. However, the AGN continuum must dominate strongly over the
continuum of the rest of the galaxy for this selection to work. As
such, this technique is biased in favour of the most optically
luminous AGNs whose light dominates ovet the host galaxy. Of course,
it only also only applies to finding Type 1 AGNs, since the blue
optical continuum from the accretion disk is blocked by the torus in
Type 2 AGNs.

\subsection{Radio selection}
This is an easy one. Roughly 10\% of all AGNs are radio loud, so we
can use radio surveys to identify them. However, since AGNs are not
the only astronomical sources that produce radio emission we need to
be careful to account for other types of sources that could
contaminate our radio sample. The biggest type of contaminant are
star-forming galaxies (since star-formation is accompanied by
supernovae, whose remnants are strong radio emitters) and supernovae
remnants in the Milky Way. However, almost every source with radio
luminosities greater than $10^{23}~{\rm W~Hz}$ is a radio loud AGN, so
it is fairly easy to exclude contaminants via a radio luminosity
cut. However, this does mean that radio selection is biased toward
radio luminous AGNs (since AGNs with low radio luminosities will be
thrown out with the star-forming galaxies). Since radio wavelengths
are immune to dust obscuration, radio selection picks out both type 1
and type 2 radio AGNs.

\subsection{Infrared selection}
Being exposed to lots of high energy photons from the accretion disk,
the dust surrounding the AGN gets very warm (indeed, the dust nearest
the accretion disk gets so hot it actually evaporates). As such, AGNs
are strong sources of infrared emission and have characteristic
infrared colours. We can therefore use infrared colour selection to
identify AGNs. However, as we have seen, low mass stars and
star-formation can also produce a lot of near and mid infrared
emission, respectively, so the emission from the AGN must dominate
over the emission from the host galaxy in order for it to be
identified via infrared selection. As such, infrared selection is
biased in favour of AGNs above a certain AGN luminosity to galaxy
mass, or AGN luminosity to SFR ratio. However, since infrared is
relatively impervious to dust, infrared selection identifies both Type
1 and Type 2 AGNs.

\subsection{Emission line selection}
The accretion disk at the centre of an AGN -- the ``central engine''
-- produces large amounts of high ionising UV photons. Moreso even
than the hottest, most massive stars. This intense UV flux is capable
of ionising gas in the narrow line region to very high levels --
emission lines from species with five or six missing electrons are not
uncommon in the spectra of AGNs. This means that AGN spectra contain
strong emission lines from highly ionised species, such as $[$O~{\sc
  iii}$]$. Because the relative strengths of different emission lines
in AGN spectra differ from those of non-AGNs, we can use emission line
ratios to identify AGNs. The most commonly used emission line ratios
are $[$O~{\sc iii}$]$/H$\beta$ and $[$N~{\sc ii}$]$/H$\alpha$, with
AGNs having high values of each of these ratios compared to non-AGN
galaxies. Since each pair of emission lines are close to each other in
wavelength (i.e., $[$O~{\sc iii}$]$ at 5007~\AA , H$\beta$ at 4861~\AA
, and $[$N~{\sc ii}$]$ at 6548~\AA , H$\alpha$ at 6563~\AA ) the
effects of dust on each ratio cancels out.

As with all the AGNs selections highlighted here, however, emission
line selection does have its pitfalls. Firstly, obtaining spectra is
time-consuming, and only a fraction of all the galaxies in an imaging
survey will have been observed spectroscopically. Further, as with all
AGN selection techniques, the AGN features must dominate over any
features arising from the host galaxy. This means that only AGNs with
strong emission lines relative to those arising from the host galaxy
will be selected by emission line selection. However, if measured
carefully, emission line selection can identify both Type 1 and Type 2
AGNs. 

\subsection{X-ray selection}
As mentioned previously, AGNs are the strongest emitters of X-rays in
the known Universe, with some AGNs emitting
$>10^{46}~{\rm ergs~s^{-1}}$ in X-rays alone. Since no other objects
produce such large amounts of X-rays, X-ray observations are a
relatively ``clean'' way of identifying AGNs. Further, since X-rays
are emitted by the central regions of the accretion disk itself, they
give us a very good measure of the ``instantaneous'' luminosity of
the AGN.

The key pitfall of X-ray emission, however, is that it can be obscured
by large columns of gas and dust. However, X-rays are less susceptible
to absorption than the UV and the shorter wavelengths of optical
light, and the highest energy photons (i.e., energies about about
10~keV) can penetrate very dense clouds of gas and dust. Since X-rays
aren't totally impervious to obscuration, however, they are more
likely to identify Type 1 AGNs relative to Type 2 AGNs (but
like-for-like they find relatively more Type 2s than optical selection).

\section{Lecture 11 learning objectives}
In this lecture we took a turn away from normal galaxies to study
AGNs. As we shall see, astronomers now think that AGNs have played a
major role in shaping today's galaxies, so we'll explore them further
for the next six lectures. For now, however, here are the learning
objectives from this lecture:
\begin{itemize}
\item Know how AGNs were first discovered and studied at optical
  wavelengths during the first half of the 20th century.
\item Have an appreciation of the role that radio astronomy has had in
  raising the interest in AGNs.
\item Know the various different classes of AGNs, and understand the
  differences between them (especially the difference between a Type 1
  and Type 2 AGN).
\item Be able to describe the AGN Unified Model, and explain how it
  accounts for the different AGN types.
\item Know the main ways we identify AGNs, and the benefits and
  pitfalls of each approach.
\end{itemize}

\end{document}