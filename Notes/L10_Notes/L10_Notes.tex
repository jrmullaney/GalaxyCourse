\documentclass[11pt]{article}
\renewcommand{\baselinestretch}{1.05}
\usepackage{amsmath,amsthm,verbatim,amssymb,amsfonts,amscd, graphicx}
\usepackage{graphics}
\topmargin0.0cm
\headheight0.0cm
\headsep0.0cm
\oddsidemargin0.0cm
\textheight23.0cm
\textwidth16.5cm
\footskip1.0cm

 \begin{document}
 
\title{Lecture 10:\\Morphological evolution and spiral galaxies}
\author{Dr. James Mullaney}
\maketitle

\section{Introduction}
Previously, we covered the topic of the formation and evolution of
early-type galaxies. As mentioned in the last lecture, these
elliptical galaxies contain roughly 50\% of the stellar mass of
today's Universe. In this lecture, we'll cover the formation of the
other major class of galaxies in the local Universe: spiral galaxies.

\section{The Milky Way's fossil record}
Recall from lecture 5 that we can use the fossil record of the Milky
Way to get a first-order impression of the formation of spiral
galaxies. In that lecture, we learned that the Milky Way is formed of
many different components, each with different star-formation
histories. This points to a complex formation history consisting of
both early gas collapse, current gas streaming and successive
mergers.

The thin disk of the Milky Way is thought to be what's left of the gas
cloud in which the fledgling Milky Way formed around 8 billion years
ago. This thin disk is still being replenished by gas streaming onto the
Milky Way from its immediate vicinity, providing a fuel supply for
continuous star formation. By contrast, the thick disk is thought to
be the result of a merger that the early Milky Way encountered when it
was beginning to form (again, around 8 billion years ago), which
``heated up'' the orbits of the stars that had already formed by that
stage. Finally, the bulge is made up of stars both from the original
collapse of the early gas cloud that formed the Milky Way, and from
subsequent galaxy mergers (which, again, heated the stellar orbits to
create the spheroid).

However, that's just for one galaxy -- our own -- which could be freak
event. To determine whether the Milky Way's formation history is
typical of other spiral galaxies, we must study the morphological
evolution of many other galaxies.

\section{The morphological evolution of galaxies}
We can exploit the redshift technique to study the morphological
evolution of spiral galaxies. However, until recently, it was extremely
difficult to do this beyond about redshift 1.5, since we need
excellent spatial resolution and sensitivities to low surface
brightness galaxies to resolve and measure the morphologies of distant
(i.e., early) galaxies. However, redshift 1.5 corresponds to about 9.4
billion years ago, so even only going out to this ``modest'' redshift
corresponds to studying morphological evolution across 70\% of the age
of the Universe.

\subsection{Quantifying morphological evolution}
While taking care to overcome surface brightness dimming and
morphological $K$-corrections, astronomers can exploit the excellent
seeing of the Hubble Space Telescope to study how the morphologies of
galaxies have evolved over the past $\sim10$ billion years. However,
morphological studies are fraught with problems associated with human
subjectivity. While we can easily quantify the luminosity or size of a
galaxy, how do we quantify its shape in a non-subjective manner? How
can you put a number on how ``spirally'' a galaxy is? This is a
significant problem for morphological studies, with astronomers
disagreeing on whether a given galaxy shows spiral structure or not
(there's a few borderline cases in the lecture slides, if you want some
examples). As such, it can be difficult using human classification
alone to determine the degree of morphological evolution over time.

In an attempt to overcome the subjectivity problems associated with
morphological classification, astronomers have attempted to identify
ways to parameterise galaxy morphology, with varying levels of
success. One of the most popular of these is the CAS set of
parameters, which uses a simple set of measurements that can be easily
made from images of galaxies. These are:
\begin{itemize}
\item {\bf Asymmetry index (A)}: rotate the image of your galaxy of
  interest through 180 degrees, then subtract the rotated image from
  the original, unrotated image. Sum the absolute values of the
  intensities in the subtracted image, then divide by the total
  intensity of the original.
\item {\bf Structure index (S)}: smooth the image of your galaxy of
  interest then subtract the smoothed image from the un-smoothed
  image. Sum the intensities in the subtracted image, then divide by total
  intensity of the original.
\item {\bf Concentration parameter (C)}: Measure the radii that
  contain (i) 20\% and (ii) 80\% of the light of your galaxy of
  interest. Divide the radius containing 80\% of the light by the
  radius containing 20\%, take the log$_{10}$ of this ratio and
  multiply the result by 5.
\end{itemize}

When astronomers measure these parameters and plot them against one
another for various populations of galaxies, we find that they
separate-out along lines of visual classification:
irregular/peculiar/merger galaxies separate from spiral galaxies which
further separate from elliptical
galaxies.\footnote{Irregular/Peculiar/Mergers are often lumped together
  because, morphologically, it is often difficult to distinguish
  between mergers and irregular ``clumpy'' galaxies.} As such, they
provide a quantitative means of separating galaxies, reducing the
subjectivity associated with visual classification alone.

\subsection{The morphological evolution of galaxies}
With a quantitative means to measure galaxy morphologies in-hand,
astronomers can investigate how these morphologies have evolved with
redshift. One of the first attempts to do this was by measuring the
galaxy number counts of different galaxy types and comparing them
against those expected from non-evolution models. In doing so,
astronomers found that the numbers of irregular/peculiar/merger
galaxies are much higher in the early Universe compared to
today. However, this trend is reversed for luminous spirals -- there
are many more luminous spirals in today's Universe compared to at
$z>0.5$.

From this and subsequent studies, astronomers have now developed a
consistent picture of how the morphologies of galaxies have evolved
since $z\sim1.5$:
\begin{itemize}
\item At redshifts below about 0.3 (i.e., up to about 3.5 billion
  years ago), luminous ``grand design'' spirals -- like the Milky Way,
  Andromeda, the Whirlpool galaxy -- exist. As such, the Hubble
  diagram exist in its full form.
\item At around $z\sim0.5$ (i.e., around 5 billion years ago), barred
  spirals become rarer and the spiral arms of all massive spirals are
  much less well-defined. The splitting of Hubble's tuning fork among
  barred and unbarred spirals is much less obvious.
\item At earlier times ($z>0.6$, or $>6$ billion years ago) the
  proportion of merger and irregular galaxies relative to spirals
  increases significantly. By around $z\sim1$ around a third of all
  massive galaxies cannot be placed on today's Hubble diagram (in
  other words, they have irregular/peculiar/merger morphologies).
\end{itemize}

\section{The morphological evolution of field galaxies}
We saw in the last lecture that the most massive ellipticals have
lived in the densest regions of the Universe throughout all of cosmic
time. By contrast, other types of galaxies have predominantly lived in
the less dense regions of the Universe between the clusters. Galaxies
within these more sparse regions are known as ``field'' galaxies, with
the more massive ones dominated, in terms of number, by spiral
galaxies. So, if we want to know how spiral galaxies have evolved, we
therefore need to study massive field galaxies out to high
redshifts. One of the most comprehensive attempts to do this has been
the Canada-France Redshift Survey (CFRS).

\subsection{The Canada-France Redshift Survey}
To determine the environments of galaxies in 3 dimensions requires
both their position on the sky (i.e., x, y position) and their
redshift. The more precise the redshift measurement, the more reliably
astronomers can place a galaxy at a given location, and thus in a
given environment. Thus, the reliable identification of large samples
of field galaxies benefits greatly from spectroscopic redshifts for
lots of galaxies.\footnote{This is less of a requirement for dense
  environments, as these can be identified as overdensities in x-y
  space, then confirmed with redshift measurements of just a handful
  of cluster galaxies.} The development of multiplexing spectrographs
on large telescopes in the mid-90s greatly advanced the study of field
galaxies by providing redshifts measurements for large samples of
galaxies.

Using the multi-slit spectroscopic instrument on the
Canada-France-Hawai'i telescope on Mauna Kea, Hawai'i, Lilly et
al. (1995) took the spectra of around 600 field galaxies. This survey
is ``complete'' to all galaxies within the survey area with redshifts
$<1$, with B-band absolute magnitudes brighter than -20.4, and masses
between $3-30\times10^{10}~{\rm M_\odot}$. It therefore samples the
population of moderate mass field galaxies out to $z\sim1$; in other
words, the galaxies that would eventually become today's massive
spirals.

The CFRS survey has had a significant impact on our understanding of
how field galaxies have evolved. For example, the lower redshift
sections of the Madau diagram (i.e., $z<1$) are largely based on star
formation rates measured via the $[$O~{\sc ii}$]\lambda3726\AA$ emission
line in CFRS spectra.\footnote{Like ${\rm H\alpha}$,
  $[$O~{\sc ii}$]\lambda3726\AA$ is only produced by high energy
  photons, and thus provides a measure of the numbers of massive,
  young stars in a galaxy. The reason $[$O~{\sc ii}$]\lambda3726\AA$ is
  often used in preference over ${\rm H\alpha}$ in the distant
  Universe is because ${\rm H\alpha}$ gets shifted out of the optical
  bands at $z>0.4$, whereas $[$O~{\sc ii}$]\lambda3726\AA$ remains
  observable in optical spectra until $z\sim1.5$.} One the greatest
impacts that the CFRS survey has been, however, on our understanding
of the morphological evolution of field galaxies.

\subsection{The morphologies of CFRS galaxies}
With spectroscopic redshifts from the CFRS in hand, astronomers could
reliably identify field galaxies for more detailed follow-up. With
this in mind, Hammer et al. (2005) obtained high resolution Hubble
observations for a sub sample of 185 CFRS galaxies in order to measure
their detailed morphologies. These detailed morphological studies
confirmed the results from number counts described above: that the
population of massive field galaxies contained a significantly greater
proportion of irregular/peculiar/merger galaxies at $z>0.4$ compared
to today, while the relative numbers of spiral galaxies at these
redshifts are significantly reduced compared to locally. For example,
while around 70\% and 7\% of $\sim10^{11}~{\rm M_{\odot}}$ galaxies in
today's Universe are spirals and irregular/peculiar/mergers,
respectively, roughly 43\% and 34\%, respectively, are at
$z>0.4$. Furthermore, the number density of LIRGs (see lecture 8) at
$z>0.4$ among $\sim10^{11}~{\rm M_{\odot}}$ galaxies is roughly 30
times higher at $z>0.4$ than it is today, with 64\% of LIRGs at these
high redshifts having irregular/peculiar/merger morphologies. As such,
the CFRS provides clear evidence of {\it significant} morphological
evolution of field galaxies since $z>0.4$.

As the changing proportions of LIRGs suggests, the CFRS also provided
evidence of strong evolution in the star-forming properties of
$\sim10^{11}~{\rm M_{\odot}}$ field galaxies. Most of the spectra of
CFRS galaxies show clear evidence of recent star-formation, again
confirming that the star formation density of the Universe was
significantly higher at $z>0.4$. Importantly for this lecture,
however, is that the {\it bulges} of $z>0.4$ galaxies are {\it bluer}
than those of similar mass galaxies today. This can be interpreted as
evidence of increased levels of star formation within the central
regions of $z>0.4$ galaxies relative to galaxies in the local
Universe. Of course, these enhanced levels of star-formation led to the
production of heavy elements, which explains why galaxies in the CFRS
have roughly half the metallicity of today's field galaxies (i.e., the
metals were still in the process of being made in the $z>0.4$ field
galaxies).

Before constructing a theory for the morphological evolution of field
galaxies (which, remember, are dominated by Spiral galaxies today),
let us, in true ``Look through the keyhole''-fashion, consider the
evidence:
\begin{itemize} 
\item The numbers of $\sim10^{11}~{\rm M_{\odot}}$ field galaxies are
  dominated by irregular/peculiar/merger galaxies at $z>0.4$, whereas
  today they dominated by spirals.
\item The proportion of $z>0.4$ field galaxies that show evidence of
  rapid, recent star formation is significantly higher than for field
  galaxies today.
\item There is clear evidence of enhanced levels of star formation in
  the central regions of $z>0.4$ galaxies compared to today's spirals.
\item The metal abundances of $z>0.4$ field galaxies is roughly half
  that of local galaxies. 
\end{itemize}

\section{The formation of local spirals}
To explain the above observations, astronomers have developed a theory
of how today's field galaxies have evolved:

\subsection{An early merger phase}
The irregular/peculiar/merger morphologies of early field galaxies
suggests that most have undergone some form of major, gas-rich merger
during the past 8 billion years or so. During this time, the disk is
suppressed by the random motions of the stars while the dissipational
gas falls toward the centre of mass of the system. This merger is
associated with a sudden burst of star-formation, but which only lasts
for a few hundred million years. This explains the LIRG nature of a
lot of these early irregular/peculiar/merger galaxies. The merger
phase is then followed by...

\subsection{A compact galaxy phase}
Many of the stars that formed during the intense episode of
star-formation triggered by the gas-rich merger have now fallen toward
the central regions of the galaxy. These stellar populations are still
comparatively young, which explains the blue colours of the bulges of
early field galaxies. The star formation rate falls significantly over
the next couple of billion years from its peak during the merger. Any
further gas remaining from the merger that has not already fallen into
the central regions may dissipationally collapse to start to form a
fledgling rotating disk around the central bulge.

\subsection{Growth of disk phase}
Stars form from the collapsing disk of gas left over from merger, but
at a much slower rate than before. Additional gas is then accreted by
the galaxy (due to its high mass) from the surrounding
regions. Irrespective of the trajectory along which this accreted gas
falls onto the galaxy, because of the dissipational nature of gas
collapse it will ultimately become part of the rotating disk. This gas
feeding is responsible for the observed low level of star formation in
spiral galaxies since $z\sim0.4$.  The galaxy may then go through a
series of successive {\it minor} mergers which, since they will be too
small to disrupt the disk, will actually build up the mass of stars
and gas in the disk.

\subsection{Plus downsizing}
Finally, it should be noted that while the above applies to field
galaxies in general, precisely {\it when} each phase occurred during
the history of the Universe changes with galaxy mass in a way
consistent that is with downsizing. For example, while moderate mass
(i.e., $\sim10^{11}~{\rm M_\odot}$) field galaxies grew most of their
mass at $z<1$, it is thought that the most massive field galaxies
(such as Andromeda) will have started to form at even earlier times
(i.e., $z\sim2$).

\section{Massive disks forming at $z>2$} 
Of course, there has been significant advancements in observing
facilities since the first CFRS studies. The development of 8m class
telescopes and, in particular, adaptive optics has helped us to obtain
high spatial resolution integral field spectra for large numbers of
galaxies at high redshifts. Such spectra enables us to map-out the
kinematics of galaxies at high redshifts, which can be used to
identify rotating disks or major mergers {\it kinematically}.

These surveys have reported evidence of rotating disks among
moderately massive ($\sim10^{11}~{\rm M_\odot}$ galaxies at
$z>2$. Indeed, at these redshifts roughly two thirds of the
irregular/peculiar/merger galaxies appear to be ``clumpy'' galaxies
that are in the process of forming disks of stars, with the remainder
being mergers. As such, it has been suggested that these are today's
massive spirals (like the Milky Way) and disky ellipticals forming at
high redshifts.

\section{Lecture 10 learning objectives}
In this lecture, we have covered the evolution of the other main type
of large galaxy in today's Universe: spiral galaxies. These dominate
the numbers of field galaxies in the local Universe, so by studying
the morphological evolution of field galaxies, we can build up a
theory to explain the build-up of today's spirals. The learning
objectives for this lecture are:
\begin{itemize}
\item Revise the fossil record of the Milky Way.
\item Understand how astronomers quantify galaxy morphologies (i.e., CAS).
\item Have an appreciation of the how Hubble number counts and the
  CFRS have provided evidence of the morphological evolution of field galaxies .
\item Be able to describe our current understanding of how today's
  spirals have evolved, together with the observational evidence that
  backs this up.
\end{itemize}

\end{document}