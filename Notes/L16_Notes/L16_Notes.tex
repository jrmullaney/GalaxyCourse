   
\documentclass[11pt]{article}
\renewcommand{\baselinestretch}{1.05}
\usepackage{amsmath,amsthm,verbatim,amssymb,amsfonts,amscd, graphicx}
\usepackage{graphics}
\topmargin0.0cm
\headheight0.0cm
\headsep0.0cm
\oddsidemargin0.0cm
\textheight23.0cm
\textwidth16.5cm
\footskip1.0cm

 \begin{document}
 
\title{Lecture 16:\\The link between star formation and AGN activity}
\author{Dr. James Mullaney}
\maketitle

\section{Introduction}
In the previous lecture, we saw how over the past two decades our
perception of AGNs moved from them being regarded as an ``astronomical
curiosity'' to playing a key role in regulating the growth of
galaxies. The main way in which this regulation is achieved is through
AGNs affecting star-formation in their host galaxies. In this lecture,
we'll take a closer look at the connection between star formation and AGNs.

\section{Star formation in AGN hosts}
As we have seen previously, AGNs are the result of interstellar
material (i.e., gas and dust) falling into the supermassive black
holes that reside at the centers of galaxies. As well as producing
large amounts of energy, the other direct effect of this accretion is
that the black holes get more massive. As such, the mass of a
supermassive black hole is the sum of all its accretion events to date.

We have also already seen that the mass of a supermassive black hole
is tightly correlated with the (stellar) mass of its host
bulge. However, we still don't fully understand what has caused this
relationship (aside from the somewhat ambiguous catch-all term of
``AGN feedback''). To try to address this, astronomers have spent a
lot of time considering the star-forming properties of AGN hosts. The
justification being that since galaxy bulges are formed from stars,
which are produced in episodes of star formation, and black hole mass
is built-up during episodes of AGN activity, then there should be some
kind of link between star-formation and AGN. By studying the
star-forming properties of AGN hosts, we are measuring the concurrent
build-up of black hole and stellar mass.

\subsection{Measuring the build-up of black hole mass}
As we saw in Lecture 12, it's fairly straightforward to measure the
growth rates of black holes during an accretion phase (i.e.,
AGN). This is because the energy radiated by the accretion process
is directly proportional to the accretion rate of the black hole:
\begin{equation}
L_{\rm AGN} = \eta \dot{M}_{\rm BH}c^2 
\end{equation}
so all we need to do is measure the luminosity of an AGN which can
then be converted into a black hole growth rate. This is, of course,
complicated by the fact we can't measure the total {\it bolometric}
luminosity of an AGN; instead, we usually measure it in one or two
bands (e.g., X-rays, optical, UV). To overcome this problem, however,
astronomers measure the luminosity measured in one part of the
electromagnetic spectrum and multiply it by a {\it bolometric
  conversion factors} to give an approximate bolometric luminosity.

\subsection{Measuring the build-up of stellar mass in AGNs}
As we saw in Lecture 4, there are a number of different ways to
measure the star formation rates of galaxies. All these techniques
rely on measuring the numbers of hot, young stars in a galaxy (i.e.,
UV continuum, H$\alpha$, infrared emission). However, in the case of
measuring the star-forming properties of AGNs, it's difficult to use
either the UV or H$\alpha$ since AGNs contribute significantly to this
type of emission. Instead, the far-infrared is widely used to measure the
star-forming properties of AGNs, since AGNs are not thought to emit
strongly in this part of the spectrum.

\section{Non-AGNs: Main Sequence and Starbursting galaxies}
Prior to considering the star-forming properties of AGN host galaxies,
we should first consider what our baseline is. In other words, we need
to know what the star-forming properties of non-AGNs are before we can
assess whether AGNs show any systematic differences in terms of their
star-forming properties.

It turns out that, at a given stellar mass, the galaxy population is
bi-modal in terms of its star-forming properties: there are so-called
``star-forming'' galaxies and ``quiescent'' galaxies. These form two
very distinct populations in terms of their optical colours, with
star-forming galaxies being blue, and quiescent galaxies being red
(due to the dominating population of old stars, not due to
dust). Curiously, there are relatively few galaxies with intermediate
green colours. As such, when we plot galaxy colours as a histogram,
the population forms two peaks -- one blue peak, and one red peak -- with
a ``green valley'' in between (yes, it is actually known as the green
valley).

An important feature of the star-forming galaxy population is that the
{\it rate} at which they form stars is tightly correlated with the
mass of the host galaxy (at least for $>95\%$ of the star-forming
population). This correlation has become known as the galaxy ``Main
Sequence'' (MS). The remaining $\sim5\%$ of {\it star-forming}
galaxies have star-formation rates that are {\it above} that of MS
galaxies and are consequently known as ``Starburst'' (SB)
galaxies. Starbursts typically have SFRs three or more times higher
than MS galaxies of the same mass.

When astronomers measured the star-forming properties of MS galaxies
out to higher and higher redshifts, they found that the SFR of a
galaxy of a given stellar mass (which can also be express as the SFR
per unit stellar mass, or specific SFR $[$sSFR$]$) {\it increases}
with redshift. So, a typical MS galaxy at redshift 2 has a sSFR about
10 times higher than a typical Main Sequence galaxy today. At first,
it was suspected that this rising SFR of MS galaxies was due to an
increase in the occurance of gas-rich major mergers at earlier times,
since mergers are a key means of enhancing star-formation in today's
galaxies.

On more detailed morphological inspection, however, it turned out that
MS galaxies at high redshifts were {\it not} dominated by merging
systems. Instead, SBs tends to be associated with mergers at all
probed redshifts, whereas MS are typically undergoing ``secular''
(i.e., isolated) evolution. Since MS galaxies {\it dominate} the
numbers of star-forming galaxies at all redshifts, it therefore seems
that the dominant mode of star-formation in the Universe is {\it not}
triggered by major galaxies mergers.

If not major mergers, what {\it is} causing the rapid rise in the
sSFRs of MS galaxies? It transpires that MS galaxies at high redshifts
have significantly higher gas contents compared to their low redshift
counterparts. Because of this, it is thought that the reason they are
forming stars so rapidly at high redshifts is simply due to a far more
abundant supply of gas in the early Universe compared to today.

\subsection{Key points to remember about MS and SBs}
Main-sequence and starbursting galaxies are a key feature of our
current understanding of galaxy evolution, so I wanted to provide a
quick summary of their properties:

\begin{itemize}

\item At a given redshift, galaxies on the Main Sequence (MS) have
  star-formation rates (SFRs) that are proportional to their stellar
  mass.
\item So, at a given redshift, their specific SFRs
  (${\rm sSFR=SFR/stellar~mass}$) is constant (but with some scatter).
\item  Galaxies with sSFRs $\sim3\times$ above the MS are known as Starbursts (SBs).
\item Star-formation in MS galaxies is thought to be triggered in
  isolation by ``secular'' processes, whereas in SBs it is triggered
  by major mergers.
\item But, SBs are comparatively rare, so MS galaxies
  dominate the star-formation budget.
\item The average (or typical) sSFR of MS galaxies increases strongly
  with redshift.
\item This redshift evolution is thought to be due to the greater availability of cold gas in the
  early Universe from which to form stars.
\end{itemize}

\section{AGN and the Main Sequence}
Now that we have characterised the star-forming properties of normal
(i.e., non-AGN) galaxies, we can consider where AGNs fit within this
picture. Do AGNs predominantly live in starburst galaxies, which would
suggest they are also triggered by major mergers, or quiescent
galaxies (which may suggest they are ``switching-off'' star-formation
via AGN feedback). Actually, it turns out that most AGNs reside in
Main Sequence galaxies, suggesting that most AGNs are also triggered
via so-called ``secular processes''.

You may feel that the finding that AGN preferentially reside in MS
galaxies is somewhat contradictory to what we saw in Lecture 14, in
which AGNs were linked to merger events. It should be noted, however,
that the vast majority of AGNs in the Universe have relatively modest
luminosities, whereas the merger-triggered AGNs we considered in
Lecture 14 are among the most luminous AGNs in the local Universe. As
such, this result reinforces thet idea that the dominant population of
moderate luminosity AGNs are triggered by secular processes, but the
most luminous AGNs are triggered by major mergers.

Since AGNs seem to prefer star-forming galaxies, then it makes sense
to ask: ``Is the luminosity of an AGN (i.e., its BH growth rate) in
any way related to the star formation rate of its host galaxy?''. In
other words, is there a correlation between galaxy BH growth rates and
star-formation rate? To investigate this, AGN astronomers have
measured the average SFRs of AGN host galaxies binned in terms of the
luminosity of the AGN. However, this experiment revealed little or no
correlation between a galaxy's SFR and the current luminosity of its
AGN. Interestingly, when we instead calculate the average AGN
luminosity of galaxies binned in terms of their SFR (i.e, averaging the other
way round), however, then a strong correlated between SFR and AGN
luminosity {\it is} uncovered. In other words, when we average one
way, we find no correlation, but when we average the other way, we do
reveal a correlation. What's going on??

It is thought that the answer may lie in a key property of AGNs: that
they vary (stochastically) on timescales that are much shorter than
typical episodes of star formation. What this means is that an AGN
will vary in luminosity by many orders of magnitude whilst the SFR of
its host galaxy stays relatively constant. By grouping galaxies in
terms of their AGN luminosity, we're selecting galaxies based on a
highly stochastoc process. The effect of this is that it ``dilutes''
any underlying connection between the AGN and the host galaxy. By
contrast, selecting galaxies based on the far more stable property of
star-formation then {\it averaging over} the stochastic AGN
variability, uncovers the true underlying links between AGNs and
star-formation.

\section{The probability of AGNs}
As with any stochastic process, it is becoming increasingly common to
think in terms of what processes affect the {\it probability} a galaxy
hosting an AGN. For example, are AGNs more {\it likely} to reside in
star-forming galaxies or -- even better -- how does the SFR of a
galaxy affect the likelihood of it hosting an AGN of a given
luminosity?

By thinking in such terms, AGN astronomers have begun to identify some
important features of AGNs. Notably, it seems that the mass of a
galaxy has no effect on whether it hosts an AGN of a given accretion
rate. By contrast, recent studies have found that the likelihood of
{\it rapid} black hole growth is enhanced in galaxies with high
SFRs. As such, it seems that there may well be an underlying
correlation between black hole growth and star-formation, but that
uncovering this link in the face of AGN variability is going to take a
lot more effort.

\section{Learning objectives from Lecture 16}
We've covered some quite conceptuallly-advanced ideas in this lecture,
so there isn't as much reading to do as usual. I wanted to make sure
there was enough time during the lecture to really explain some of the
key concepts. So, don't worry too much if some of the ideas in these
notes are tricky to grasp, there'll be plenty of opportunity for
further explanations in the lecture. The key objectives you should
take from this lecture are:

\begin{itemize}
\item Understand the evidence to support a connection between AGN and
  star-formation, i.e,;
  \begin{itemize}
  \item BH-bulge relationship.
  \end{itemize}

\item Understand how we measure the AGN-SF connection, i.e., 
  \begin{itemize}
  \item $L_{\rm AGN}$ (X-rays, etc), SFR (Optical, UV, and
    particularly IR etc.) 
  \end{itemize}

\item Know what the star-forming Main Sequence is:
  \begin{itemize}
  \item sSFR rises with redshift due to the increased availability of gas in the early Universe.
  \end{itemize}
    
\item Understand the importance of AGN variability in hampering our
  ability to connect AGN luminosities to other galaxy properties:
  \begin{itemize}
  \item “Washes out” the underlying connections
  \end{itemize}

\item Be aware of our current understanding of the AGN-SF connection:
  \begin{itemize}
  \item it seems rapid BH growth is more prevalent in star-forming galaxies.
  \end{itemize}
      
\end{itemize}

\end{document}

