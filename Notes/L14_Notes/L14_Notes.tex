   
\documentclass[11pt]{article}
\renewcommand{\baselinestretch}{1.05}
\usepackage{amsmath,amsthm,verbatim,amssymb,amsfonts,amscd, graphicx}
\usepackage{graphics}
\topmargin0.0cm
\headheight0.0cm
\headsep0.0cm
\oddsidemargin0.0cm
\textheight23.0cm
\textwidth16.5cm
\footskip1.0cm

 \begin{document}
 
\title{Lecture 14:\\The triggering of AGN}
\author{Dr. James Mullaney}
\maketitle

\section{Introduction}
In the previous few lectures, we've seen that two ingredients
are needed to produce an AGN: a supermassive black hole and supply of
gas and dust to accrete onto it. In the last lecture, we considered
current theories on where the supermassive black holes originate from. Now,
we will consider the other major challenge in AGN astronomy: what
causes the gas and dust to accrete onto a supermassive black hole to
produce an AGN.

\section{Dormant and active black holes}
It has already been highlighted in previous lectures that it is
thought that most, if not all, massive (i.e., $\gtrsim10^9~{M_\odot}$)
galaxies contain a supermassive black hole at their centres (see the
Solan Argument of Lecture 12). Further support of this is the clear
evidence of the supermassive black hole at the centre of the Milky Way
(known as Sagittarius A$\ast$), which clearly demonstrates that normal
(i.e., non-AGNs) galaxies contain dormant supermassive black
holes. These dormant black holes may ``flare up'' once in a while as
they consume small amounts of gas and dust, but never normally enough
to warrant being labelled as an AGN.\footnote{I once attended a
  conference presentation in which it was said that Sagittarius
  A$\ast$'s flares correspond to it accreting roughly a mountain's
  worth of gas within a few hours. Pretty impressive, but far from the
  $\sim$solar mass per year needed to power a quasar.}

In order to become an AGN, the supermassive black hole needs to
accrete at a rate of at least a few percent of a solar mass per
year. This material forms an accretion disk which is of the order
0.01~pc in size, but to reach these small scales the gas must lose
$>>99\%$ of its angular momentum, which is no mean feat. How this is
achieved is known as the ``AGN triggering problem'', since the
transport of gas to the nucleus is needed to ``trigger'' an AGN, and
is the focus of much research in AGN astronomy.

\section{Suggested AGN triggering mechanisms}
Since their discovery, a number of different mechanisms have been
suggested as possible means of transporting gas from galaxy-scales
(i.e., $\sim$kpc) to accretion disk-scales (i.e., sub-pc) in order to
trigger an AGN. The most popular of these are:
\begin{itemize}
\item {\bf Galaxy mergers and interactions}: Because of the disruption
  of a galaxy's internal dynamics caused by a merger (i.e., from
  rotational support to a much more chaotic system) or even an
  ``interaction'' (i.e., a close fly-by leading to tidal streams),
  they are an extremely effective means of removing angular momentum
  from internal gas. As such, mergers have long been suggested as a
  possible means of triggering an AGN.
\item {\bf Secular processes}: The opposite of galaxy
  mergers/interactions are called ``secular processes'', and refer to
  when a galaxy is just going about its usual business in
  isolation. Suggested secular accretion mechanisms include: spiral
  arms or bars channelling gas to the nuclear regions, the accretion
  of small satellite galaxies (i.e., those that form the tidal streams
  around the Milky Way and are too minor to be considered true
  mergers), winds or ejecta from processes associated with
  star-formation (including supernovae). Basically anything {\it
    internal} that could channel cold gas in the galaxy toward the
  nuclear regions.
\item {\bf Accretion of hot halo gas}: All galaxies sit within a dark
  matter halo. These halos contain large amounts of very diffuse gas
  (indeed, the majority of gas in the Universe is in this state)
  believed to be kept hot ($\sim10^6$~K) by feedback mechanisms (i.e.,
  energy injected from the galaxy, not least by AGN) and shock
  heating. It has been suggested that some of this gas can penetrate
  to the centre of a galaxy and by accreted in a ``hot mode'' to via
  Bondi accretion (in which the gas does not form an accretion disk;
  feel free to look it up, but we won't go into more details
  here). Alternatively, some of this hot gas may cool (to $\sim10^4$
  to $10^5$~K) to form what are known as ``cooling flows'' which
  stream onto the galaxy (toward the centre of mass) from the halo.
\item {\bf Cold accretion from large-scale filaments}: Some galaxies
  (especially massive ellipticals) live at the nodes of large-scale
  (i.e., many tens of Mpc) filamentary gaseous structures. These
  structures are very effective at transporting gas toward the nodes
  where the galaxies sit, and it has been suggested that they may even
  penetrate the galaxies right to their nuclei. If that is the case,
  they may provide a direct channel to feed an AGN.
\end{itemize}

For the rest of the lecture, we'll consider if there is any evidence
for these various potential mechanisms (we'll bunch the final two
together, as they are essentially the same thing - accreting material
from scales far larger than the galaxy).

\section{Galaxy mergers and interactions}
Perhaps the easiest triggering mechanism to test for is mergers or
interactions. If we find a significantly higher proportion of AGNs in
galaxies that are undergoing mergers compared to non-AGNs, then we can
infer that the merger process is, indeed, an effective means of
funnelling gas toward the galaxy nucleus to trigger an AGN. Note,
however, the importance of a comparison sample in that statement: it
is crucial that we compare like-for-like AGN and non-AGN galaxies when
trying to identify AGN triggering mechanisms. This typically involves
identifying a mass-matched sample of non-AGNs which, thankfully, is
usually comparatively straightforward since non-AGNs outnumber AGNs by
many tens-to-one.

To date, a number of studies have explored the question of whether
AGNs preferentially reside in merging systems. Because of the
difficulty in spanning very broad ranges of AGN luminosity in our
samples (recall the lecture on extragalactic surveys, in which we saw
that different depths and areas of surveys were used to identify
different luminosity systems), these studies typically focus on a
comparatively narrow luminosity range. Results from deep-field surveys
in particular show no evidence of a higher fraction of mergers among
moderate luminosity AGNs (i.e.,
$L_{\rm Bol}\lesssim5\times10^{44}~{\rm erg~s^{-1}}$) compared to non-AGNs
in the same fields. This suggests that secular processes of the type
highlighted in the previous section are triggering these lower
luminosity AGNs.

The triggering mechanism for more luminous quasars may, however,
differ from more moderate, less luminous AGNs. This may well be
because more violent processes are needed to channel the greater
amounts of gas needed to trigger quasars than can be achieved with
sedate, secular processes. Indeed, computer simulations predict that
the peak of AGN activity (which we would observe as a quasar) take
place during the final stages of a major galaxy merger. To test this,
however, requires a sample of powerful quasars with sufficient quality
observations to see signs of recent merger activity.

\subsection{The 2Jy sample of radio galaxies}
Among the most well-studied samples of nearby (i.e., $z<0.7$) luminous
AGNs is the 2Jy sample of southern radio galaxies (a radio galaxy is
another name for a radio loud AGN). Although originally selected
because of their high radio luminosities (they all have radio fluxes
above 2~Jy, which makes them some of the brightest radio sources in
the whole sky), it turns out that almost 80\% of the AGNs in the 2Jy
sample are also optically-luminous quasars (although in some cases,
the nucleus is obscured from view; see the lecture 11 on AGN
unification).

Using very deep (i.e., sensitive) optical imaging for the 2Jy sample,
astronomers have found that around 15\% show evidence of undergoing
current major mergers. Furthermore, a further 70\% of them show clear
evidence of tidal features, which are a tell-tale sign of recent
galaxy interactions. This is a much higher fraction than found in
matched comparison samples of non-AGNs in the local Universe. This is
consistent with the idea that powerful radio galaxies (and possibly
most quasars) are triggered in galaxy interactions. However, contrary
to what is suggested by simulations, it seems that the triggering
isn't associated with a particular stage of a merger, simply that a
merger has taken place in the recent past (within a few 100~Myr; i.e.,
a late-stage merger).

\section{The role of star-formation}
One of the main problems with trying to figure out what triggers AGN
is that, invariably, multiple {\it potential} triggering mechanisms
are present in a galaxy {\it at the same time}. In particular, major
gas rich mergers also induce high levels of star formation within
galaxies (due to the compression of cold gas clouds within the
colliding galaxies). It can, therefore, be difficult to assess whether
the AGN is, indeed, triggered by the merger, or whether it's really
the star-formation that induces the AGN and the merger's role is
simply to enhance the levels of star-formation. While this may be a
moot point for mergers (after all, it's ultimately the merger which
triggers AGN), but it's important for AGN triggering in general to
know whether star-formation (which is common) is sufficient, or
whether some kind of interaction (which is rare) is critical to
trigger an AGN.

Trying to untangle the role of star-formation vs. merger is further
complicated by the difficulties in measuring accurate rates of star
formation in galaxies hosting powerful AGNs. This is because a
powerful AGN can contribute to all of the wavebands traditionally used
to measure star formation rates (SFRs). For example, a bright Type 1
AGN will dominate over any star formation at UV wavelengths, and even
a Type 2 AGN can contribute to the UV bands via reflected light
(reflected from clouds of gas within the galaxy). Similarly, while
AGNs are typically quite weak at far-infrared wavelengths, even here
they can dominate over low levels of star-formation (to add a further
complication: the intrinsic SED of AGNs remains poorly constrained at
infrared wavelengths).

Perhaps the most reliable means of measuring the SFRs and star-forming
histories of {\it powerful} AGNs is via sensitive spectroscopy
observations of the host galaxies. Stellar absorption features in the
spectra can be modelled (via spectral synthesis) to give precise ages
of the stellar populations in the host galaxy, and by placing the slit
off-nucleus, the emission from the AGN itself can be mitigated. When
this is done for the 2Jy sample of nearby bright quasars, young
stellar populations are only detected in about 20-35\% of the
sample. This has been interpreted as evidence that, while interactions
may play a key role in triggering AGNs, in most cases they are not
triggered at the {\it peak} of major, gas rich mergers when most of
the star-formation takes place. This has led some to speculate that
there is a {\it delay} between the closest approach of a merger and
the triggering of an AGN.

\section{Triggering via cool gas accretion}
The final suggested means of AGN triggering we will consider is via
direct accretion of cold gas from intergalactic space (via cooling
flows or channelled along large-scale filaments). In order to power a
quasar with a bolometric luminosity of
$L_{\rm Bol}>10^{45}~{\rm erg~s^{-1}}$, a black hole must accrete at a
rate of roughly 0.2~${\rm M_\odot~yr^{-1}}$. If a typical quasar
lifetime is of the order $10^6$ to $10^8$ years (the former based on
the size of the largest radio jets, the latter based on the fraction
of massive galaxies hosting quasars within a given redshift range
$[$i.e., within a given time interval $]$), then this means that the
black hole will accrete roughly
$2\times10^{5}-2\times10^{7}~{\rm M_\odot}$ of gas during a typical
quasar episode.

However, that only represents the gas that falls into the black hole,
yet the black hole-to-bulge mass relationship tells us that for every
one solar mass of gas/dust that falls into the black hole, there must
be 500~${\rm M_\odot}$ that forms stars. As such, to fuel a
quasar for about $10^6$ to $10^8$ years requires a {\it total} gas
reservoir of $10^8 - 10^{10}~{\rm M_\odot}$ (and that's assuming a
100\% efficiency in converting gas into stars, which is far from the
case in reality). So, the key question is: ``Is there any evidence
that such a large gas reservoir is even available to fuel a quasar?''.

In astronomy, measuring the mass of gas contained within a given
region is notoriously difficult. If the gas is ionised, we can use the
strength of ionisation lines as a proxy-measure. However, the ionised
phase only represents a small fraction of gas in a galaxy available to
form stars or accrete onto a black hole. Instead, the dominant gas
supply is either in the neutral or molecular phase, which doesn't emit
at optical wavelengths. It is possible to use the Hydrogen 21~cm line
in the radio bands, but this is weak and so only detectable in the
most nearby galaxies. So, to measure the amount of neutral gas available,
astronomers use the fact it is often accompanied by large amounts of
dust, and so use the dust mass as a proxy for gas mass (a ratio of
$M_{\rm Gas}/M_{\rm Dust}\sim100$ is typically assumed). Since dust
emits as a black body, if we know its temperature (which we can
calculate using infrared colours), then we can calculate its mass from
its (infrared) luminosity. 

If quasars are being fuelled by cold gas within their host galaxies,
then we should measure dust masses of around $10^6-10^8~{\rm M_\odot}$
(i.e., around 1\% of the required gas mass) in galaxies hosting
powerful quasars. Using the {\it Herschel} infrared telescope, which
was launched in 2009, astronomers have measured the infrared
temperatures and luminosities (and consequently, masses) of the dust
around powerful quasars, including the 2Jy sample. As predicted, they
do indeed contain typical dust masses of around $10^7~{\rm M_\odot}$,
confirming that there is sufficient gas to fuel their resident quasars
for $\gtrsim10^6$~years.

Finally, to give you some sense of how large $10^8~{\rm M_\odot}$ of
gas is, the Large Magellanic Cloud (LMC) contains roughly this amount
of gas. So it is feasible that, should the Milky Way eventually merge
with the LMC (which would be classed as a minor merger), there would
be sufficient gas supplied by the interaction to trigger Sagittarius
A$\ast$ into becoming a quasar.

\section{A summary of AGN triggering mechanisms}
Over the course of this lecture, we have seen how various different
mechanisms have been suggested as possible means of triggering an
AGN. As you have probably already noted, there is no ``single -fix''
to this issue, with mergers/interactions, secular processes and cool
accretion all possible mechanisms (we'll also consider evidence that
star-formation may also be linked to AGN in Lecture 16). However, this
probably shouldn't be too much of a surprise; all it takes to power an
AGN is material falling onto a black hole. The black hole doesn't care
how the material is funnelled onto it, so it's probably to be expected
that different triggering processes can all play a role.

I feel the key thing to take away, however, is that there seems to be
evidence that AGNs of different luminosities seem to be triggered by
different processes. For more moderate luminosity AGNs, it seems that
secular processes are sufficient (including, as we'll see in L16,
non-merger-induced star-formation). By contrast, there is increasing
evidence that the most luminous AGNs are, indeed, triggered (or at
least, helped) by a galaxy merger/interaction. One thing I can tell
you for {\it for certain} is that this is a highly active area of
current research (so nothing is really certain!), so our understanding
of AGN triggering may change considerably over the coming years.

\section{Learning objectives for Lecture 14}
In this lecture we've considered the possible mechanisms of driving
gas and dust from the outskirts of a galaxy toward its nuclear regions
in order to trigger an AGN. This is a highly active area of research
and, consequently, remained surrounded in uncertainties. Having said
that, there are some key things you should take from this lecture(!):
\begin{itemize}
\item It appears that moderate luminosity AGNs are triggered by
  ``secular'' processes.
\item Local radio galaxies (i.e., the 2Jy sample) are diverse in terms
  of their detailed morphologies, star formation properties, and cool
  ISM contents.
\item A small but significant minority ($\sim$15\%) are triggered in
  major, gas-rich mergers in which both the super-massive black holes
  and stellar masses of the host galaxies are growing rapidly.
\item But the majority of local radio galaxies represent much later
  stages of galaxy interaction, possibly indicating a late-time
  re-triggering of AGN activity via galaxy interactions and/or minor
  mergers (~2xLMC gas mass)
\end{itemize}

\end{document}