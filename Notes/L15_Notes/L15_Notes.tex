   
\documentclass[11pt]{article}
\renewcommand{\baselinestretch}{1.05}
\usepackage{amsmath,amsthm,verbatim,amssymb,amsfonts,amscd, graphicx}
\usepackage{graphics}
\topmargin0.0cm
\headheight0.0cm
\headsep0.0cm
\oddsidemargin0.0cm
\textheight23.0cm
\textwidth16.5cm
\footskip1.0cm

 \begin{document}
 
\title{Lecture 15: \\ AGN feedback and outflows}
\author{Dr. James Mullaney}
\maketitle

\section{Introduction}
Over the past few lectures we've looked extensively at supermassive
black holes (SMBH) and the AGN that are produced when these black holes
accrete matter. However, while you may have (hopefully) found AGNs
interesting in their own right, you may be asking yourself what
they've got to do with galaxy evolution. In this lecture, we will be
addressing that very question.

\section{Indirect evidence of links between AGN and galaxy evolution}
As you should probably be aware by now, no AGN is in complete
isolation: as the name suggests, they're all found in the centres of
galaxies. Furthermore, we saw in the previous lecture that the
trigerring of AGNs is intrinsically linked to their host galaxies
since all accreted material ultimately comes from the host. Indeed, we
saw that it's likely that some AGNs are triggered by galaxy
interactions. As such, there is clearly at least some level of
co-evolution between AGNs and their host galaxies.

Until the mid-1990's, this ``co-existence'' picture of AGNs and
galaxies formed pretty much our whole understanding of the
relationship between AGNs and their host galaxies. However, toward the
end of that decade and into the mid 2000's, a number of major discoveries
were made in AGN astronomy that suggest that the relationship between
AGNs and their host galaxies is much closer than previously
thought.

\subsection{Evidence from observations}
The first of these discoveries was that the masses of supermassive
black holes are tightly correlated with the masses of their host
galaxy bulges. As we saw in Lecture 12 (\S 4.4), this tight
correlation is remarkable because the gravitational influence of the
black hole only dominates within a few tens of parsecs, meaning that
-- gravitationally at least -- the vast majority of stars within the
galactic bulge ``knows'' nothing of the black hole in the
centre.\footnote{It should go without saying that, in terms of
  gravity, the black hole known nothing of the bulge. Gauss's theorem
  tells us that.} The tight correlation between black hole mass
and bulge mass has been interpreted as evidence of a level of
interaction between the two {\it beyond} that expected from gravity
alone.

The second discovery that was interpreted as signalling a strong
evolutionary link between SMBHs/AGNs and galaxies is that the star
formation rate density and the black hole accretion rate density of
the Universe have evolved ``in-sync'' over the entire observable
history of the Universe. As such, by one way or another, the total
amount of material accreting onto a SMBH (during an AGN episode) at
any epoch is closely tied to the total amount of material that forms
stars. This is seen as clear evidence that black holes and galaxies
have co-evolved over the past 13 billion years.

\subsection{Evidence from simulations}
The final major discovery linking AGNs to galaxy evolution that we
will consider came not from observations, but from computer
simulations (or, more correctly, from comparing the predictions made
by computer simulations against observations). One of the most
important -- and reliable -- outcomes of cosmological-scale,
dark-matter dominated semi-analytic models is the mass function of
dark-matter halos.\footnote{In the same way that there is a galaxy
  mass function -- simply the histogram showing the numbers of
  galaxies in a given mass bin per unit volume -- there is a dark
  matter halo mass function.} It is ``reliable'' because dark matter
only interacts with itself and other matter via gravity, which we
understand {\it very} well on cosmological scales. What semi-analytic
models predict is that the dark-matter halo mass function can be
described very well as a simple power law (i.e., a straight line on a
log-log plot). Of course, we can't actually measure the dark matter
mass function to compare against this prediction, so instead we to use
analytic prescriptions to populate the dark matter halos with galaxies
which we can then compare to the (measured) {\it galaxy} mass
function.

The prescriptions used in the first truly cosmological-scale
semi-analytic models (in the mid-2000s) predicted that the galaxy mass
function was very different to that observed in the real Universe. In
particular, the first simulated mass functions hugely overpedicted the
numbers of very low mass and very high mass galaxies in today's
Universe (by many orders of magnitude). In fact, the earliest
simulated galaxy mass functions were also well-described by a single
power-law, whereas as we have seen throughout the course, the observed
mass function is better described by a broken power law (or, more
specifically, a Schechter function). It became clear that what was
needed was a way to prevent too many very low and very high mass
galaxies from forming within the dark matter halos.

The way to prevent a galaxy from forming or getting bigger is to
prevent gas from cooling and collapsing to form stars. This is done by
injecting energy into the gas to either heat it up or drive it out of
the gravitational potential well that is pulling it together. At the
time of the first simulations, it was already known that a major
source of energy input is from stars themselves, in the form of stellar
winds, supernovae and simply radiation from the photosphere. However,
when this ``feedback'' from stars was included in the semi-analytic
models, it was found they only solve one part of the problem: they
only prevent the formation of too many small galaxies. In order to
prevent too many {\it massive} galaxies from forming, much more energy
needs to be injected. It was found that by including (reasonable
levels of ) energy from AGNs in the semi-analytic models prevented gas
from collapsing and forming too many massive galaxies, and closely
reproducing the observed galaxy mass function. As such, the comparison
between the galaxy mass function predicted by semi-analytic models and
that observed in the real Universe provides indirect evidence of
``feedback'' processes (both stellar and AGN in origin) regulating
galaxy growth.

Further indirect evidence of ``AGN feedback'' came from smaller-scale
simulations of merging galaxies. Around the same time that
cosmological-scale semi-analytic models were predicting the need for
AGN feedback (i.e., mid-2000s), simulations of galaxy interactions
were also providing evidence of a link between AGNs and their host. In
these smaller-scale simulations, it was found that, unless energy from
an AGN was used to ``sweep-out'' the gas from a galaxy post-major
merger, then the resulting galaxies were far too compact (compared to
real galaxies) and did not reproduce the observed black-hole to bulge
mass correlation.

\section{How might AGNs affect galaxy evolution}
The indirect evidence of a close link between AGNs and galaxies
highlighted above caused a significant shift in our consideration of
AGNs. In the space of a decade, AGNs went from being studied as a
mere ``curiosity'' to being regarded as having a major impact on how
galaxies have evolved.

Despite their perceived importance, it remains unclear precisely {\it
  how} AGNs affect galaxy evolution in the {\it real}
Universe. Although we know that AGNs produce a lot of energy, it's not
clear how (or even whether) this energy is effectively transferred
into their host galaxies to affect a change. However, simulations do
predict two key mechanisms:
\begin{itemize}
\item Heating gas on large (i.e., dark matter halo) scales which
  prevents it from cooling and collapsing toward the centre of the
  halo (where the galaxy lives) and forming stars;
\item Heating-up or driving-out the cool gas already within the galaxy which
  {\it would} have formed stars.
\end{itemize} 

Of course, it is also possible that once gas driven out by the second
mechanism reaches large enough scales it is prevented from re-cooling
by the first mechanism. In the following sections, we'll look in more
detail at the observational evidence supporting these two mechanisms.

\section{Heating inter-galactic gas}
To heat inter-galactic gas (i.e., the gas {\it between} galaxies), an
AGN would be required to have an influence on the environment well
outside its host galaxy. In other words, on scales of tens to
thousands of kiloparsecs (remember, a galaxy is a few tens of kpc
across, while the distances between galaxies are typically measured in
megaparsecs). These are the kinds of scales that are reached by the
jets of radio-powerful AGNs, so these provide an obvious potential
means for an AGN to influence the inter-galactic material.

With clear observational evidence of kpc to Mpc scale radio jets being
produced by AGNs, the next key question is: do these jets actually
heat the inter-galactic material and prevent it from cooling onto the
host galaxy where it can form stars? One way this could be achieved is
via {\it shock heating}. Shocks occur when discontinuities in gas
pressure and density move though a medium (in this case the
inter-galactic gas) at speeds faster than the local sound speed. When
this happens, the atoms of gas are accelerated at the interface of the
discontinuity, which heats up the gas. 

The plasma from which AGN jets are formed are travelling at close to
the speed of light, so much, much faster than the local sounds speed
of the inter-galactic material (which is typically around
300~${\rm km~s^{-1}}$. As such, these jets generate shockwaves that expand
outward from the jet as it advances through the intergalactic
material. Importantly, the shock waves expand spherically around the
progressing jet, so are not just confined to the small opening angle
of the jet and, instead, will propagate throughout the surrounding
intergalactic material. These advancing shocks are predicted to heat
the intergalactic material to $\sim10^6$~K, which is easily high
enough to prevent it from cooling onto the galaxy.

Do we actually see any evidence of AGN-launched jets actually
transmitting shockfronts into the intergalactic gas? At temperatures
of $10^6~{\rm K}$, gas emits X-rays as an extremely hot black body. By
observing inter-cluster gas around radio AGNs, astronomers have indeed
found evidence of extremely hot, shocked gas forming ``bow-shocks''
around the the collimated radio jets. Crucially (and as expected)
these shocks have a much larger opening angle compared to the highly
collimated jets, so display clear evidence of radio jets transmitting
large amounts of energy into large volumes of the inter-cluster
material. This energy is, indeed, heating the gas up to millions of
Kelvin, preventing it from cooling to form stars.

Another way that AGN-launched radio jets can affect the inter-cluster
gas is by also physically moving the gas away from the galaxy. In this
scenario, the jets excavate cavities of gas which is similar, in principle, to
blowing up a bubble. The energy contained within these bubbles is
sufficient to prevent the cooler gas outside the bubble from cooling
and falling onto the host galaxy. Such bubbles of hot gas are, indeed, seen
in X-rays behind the shock waves caused by the propagating jet.

\section{Affects on the inter-stellar material}
As well as preventing gas {\it outside} the galaxy from cooling and
collapsing to form stars, it is thought that AGNs may also affect the
gas {\it within} the host galaxy; the so-called inter-stellar
material. After all, a typical AGN produces many times more energy
than is required to disrupt (i.e., ionise, heat, eject) the {\it
  entire} gas content of a galaxy. In order to do so, however, a
significant fraction of this energy must be captured by the interstellar
gas. It remains to be determined whether the coupling between the
energy released by AGNs and the interstellar gas is sufficient to
prevent the latter from forming stars. 

There are two main mechanisms by which astronomers think that AGNs are
affecting the interstellar gas in their host galaxies: one is via
radio jets (again) and the other is via non-relativistic
``winds''. Both, ultimately, lead to the same result: the evacuation
of interstellar gas from the host galaxy.

\subsection{Jet-ISM interaction}
As well as the huge, Mpc-scale radio jets displayed by some AGN (in
fact, a minority of about 10\%), many more show evidence of compact
radio emission. On closer inspection, these compact radio AGNs do
display evidence of radio jets, but they are typically extended on
sub-kpc (rather than Mpc) scales. As such, these jets are extended on
galaxy-scales.

What is particularly interesting about these galaxy-scale jets in
terms of AGN feedback is that optical spectroscopy has revealed that
they are often associated with high-velocity gas in the interstellar
material.\footnote{Recall that we can extract the kinematics of the
  gas from the profiles of its emission lines.} It is not unusual for
this gas to have speeds of 1000~${\rm km~s^{-1}}$ or more, but they
are {\it not} relativistic. In other words, this is {\it not} the
plasma that is forming the jets, but is instead interpreted as
interstellar material that is being accelerated to many hundreds of
${\rm km~s^{-1}}$ {\it by} the jet. Based on reasonable (but highly
uncertain) assumptions about its density, this jet-accelerated
interstellar material is thought to be outflowing at a rate of many
hundreds (and in some cases up to thousands) of solar masses per
year. If this is maintained over a the expected lifetime of an AGN of
a few tens of millions of years, this could, in principle result in
around $10^{10}$~${\rm M_\odot}$ of gas being evacuated from the host
galaxy -- enough to have a significant impact on the future growth of
the galaxy.

\subsection{AGN ``winds''}
So far, we've seen that radio powerful AGNs are prime candidates for
``AGN feedback''. However, only a minority of AGNs are radio loud, so
what about the dominant population of radio-quiet AGNs? Are they
thought to induce ``AGN feedback''? If so, has this been confirmed by
observations?

In previous lectures, we've seen how an AGN (radio quiet or otherwise)
can produce the same amount of light as an entire galaxy in a region
of space only a few times bigger than our solar system. Not only that,
but a lot of this light is emitted in the UV. This concentration of
energy production is truly staggering, and leads to some intense
physical processes. One of the most interesting of these in terms of
AGN feedback are ``AGN winds''.

It is thought that one of the consequences of the intense heat and
radiation of the accretion disk is that it ``boils off'' large
quantities of gas (possibly as much as half of the mass of the
accretion disk may evaporate in this way). This is not dissimilar
to stellar winds being driven-off massive stars. As soon as the gas rises
out of the accretion disk, it is subject to the full intensity of the
radiation from the disk and is accelerated from the accretion disk via
photon pressure. This gas can reach speeds of tens of thousands of
${\rm km~s^{-1}}$ very close to the central engine, but as they travel
outwards they lose velocity due to gravity (from the stars in the
galaxy) and because they crash into the ambient interstellar
gas. However, despite their loss of speed, it is still predicted that
they are sufficiently powerful to eject a substantial amount of the
interstellar gas from the host galaxy.

But do we {\it see} any evidence of such AGN winds among radio-quiet
AGNs?  The answer is an emphatic Yes. Indeed, roughly a fifth of all
quasars show clear evidence of broad, blushifted absorption lines in
their rest-frame UV spectra. It is not unusual for these absorption
lines to be shifted by many tens of thousands of ${\rm km~s^{-1}}$,
meaning that the absorbing gas is moving {\it toward} us (relative to
the host galaxy) at an incredibly high speed. This has been
interpreted as clear evidence of gas {\it outflowing} along our line
of sight toward the central engine. Due to the levels of variability
displayed by these lines, it is thought these they trace outflows with
roughly 10~pc of the central engine.

At even smaller scales (and higher energies), there is evidence of
even faster winds in X-ray observations. These show absorption
features that are blueshifted by many tens of thousands of
${\rm km~s^{-1}}$ (the fastest are approaching relativistic
speeds). Again, these are interpreted as providing evidence of
extremely fast moving outflows on scales of less than 1~pc from the
central engine. Further, the amount of mass contained within these
``ultra fast outflows'' is an appreciable fraction of the Eddington
luminosity of the SMBH.

So, there is clear evidence that radio-quiet AGNs do, indeed, drive
powerful outflows. However, until fairly recently, these discoveries
were limited to small scales, whereas for an AGN to influence galaxy
evolution these outflows must interact with the interstellar material
on kpc-scales. Recently, however, integral field observations of AGNs
both in the local Universe and in the high-redshift Universe show
evidence of strongly blueshifted {\it emission} lines on kpc
scales. Again, this has been interpreted as evidence of AGN-driven
outflows, but with velocities somewhat lower than those observed in UV
and X-ray absorption lines. Despite this, they still show evidence of
outflowing gas travelling at hundreds or, in some cases, thousands of
${\rm km~s^{-1}}$ on {\it galactic} scales. It is thought that these
outflows are in the processes of sweeping-out the interstellar
material within the host galaxy, and thereby quenching future
star-formation. However, finding galaxy-scale outflows is one thing,
but actually {\it proving} that {\it quenching} is actually taking
place is proving to be far more of a challenge and is currently one of
the hottest topics in extragalactic astronomy research.

\section{Learning objectives from Lecture 15}
In this lecture, we've considered how, rather than a simple
astronomical curiosity, AGNs are now thought to have had a major
influence of galaxy evolution. The key learning objectives are:
\begin{itemize}
  \item Have an understanding of the indirect evidence of AGN
    feedback, including that derived from simulations.

\item Have knowledge of the ways in which AGN might affect the evolution of
  the host galaxies;

\item Have an understanding of the differences between jet mode and quasar
  mode feedback

\item Have Knowledge of the direct observational evidence for radio
  and quasar mode feedback, and the impacts on the host galaxies.

\end{itemize}

\end{document}