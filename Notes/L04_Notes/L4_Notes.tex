   
\documentclass[11pt]{article}
\renewcommand{\baselinestretch}{1.05}
\usepackage{amsmath,amsthm,verbatim,amssymb,amsfonts,amscd, graphicx}
\usepackage{graphics}
\usepackage{siunitx}
\topmargin0.0cm
\headheight0.0cm
\headsep0.0cm
\oddsidemargin0.0cm
\textheight23.0cm
\textwidth16.5cm
\footskip1.0cm

 \begin{document}
 
\title{Lecture 4:\\Spectral synthesis and star-formation indicators}
\author{Dr. James Mullaney}
\maketitle

\section{Introduction}
In this lecture, we will cover how we actually measure the properties
of galaxies. This is critical to our understanding of galaxy evolution
since, if we can't measure their properties, we can't build up a theory
of how they evolve.

\section{Measuring galaxy properties}
In astronomy, we can only study objects from a distance, normally via
the electromagnetic radiation they emit (and, as of 2015,
gravitational waves). We can't handle them, look at them from a
different angle etc. etc. So how do we measure their physical
properties such as mass, composition etc.?

In the case of stars, we have the luxury of binaries that can help us
to measure their masses, while the fact that stars are ``simply''
spherical balls of gas also helps (though, I'm sure Profs. Crowther
and Dhillon would disagree!). In the case of galaxies, we don't have
such ``ideal'' scenarios, so we're simply limited to analysing the
combined light from their populations of stars.

But, how do we take the light from a galaxy and convert it into a
mass, metallicity, star-formation rate, etc.? One of the most
important and widely-used means is by spectral synthesis.

\subsection{Spectral synthesis}
The basic idea of spectral synthesis is to build a model that
represents the light from all the stars in a galaxy. If that model
accurately reproduces the spectrum (ideally) or colours (more usually)
of the galaxy, then we can infer various galaxy properties from that
model. For example, if the spectrum of a galaxy was perfectly
reproduced by combining the spectra of a billion identical, sun-like
stars, then we'd be able to say that the mass of the galaxy is a
billion solar masses. In reality, however, it is far more complicated
than that since, as you'd expect, galaxies are made up of stars
spanning a wide range of masses, ages and metallicities. So, to
produce these ``synthetic'' spectra astronomers have to combine the
spectra of lots of different types of stars. Thankfully, we have a
recipe to help us constrain the range of stars we include in our models.

\subsubsection{Initial mass function}
The first part of the recipe requires us to choose an Initial Mass
Function (IMF). As with most such ``functions'' we've met so far, it's
really more of a histogram. It is the histogram of stellar masses that
a cloud of gas produces the moment it collapses to form stars. It
tells us that for every single ${\rm10~M_\odot}$ star, there are about
a hundred ${\rm1~M_\odot}$ stars, and about ten thousand
${\rm0.1~M\odot}$ stars. Why is this important? Because the spectrum
of a star (and how it evolves, see later) depends hugely on its mass,
so it is important that our model includes the relative numbers of
stars of different masses.

Finally, while there is some uncertainty in the shape of the IMF and
whether it was different at earlier times in the Universe (indeed,
whole conferences are dedicated to this topic), it seems that the IMF
is fairly universal. As such, astronomers typically assume a constant
IMF when doing extragalactic studies. However, since there are various
different IMF (e.g., the Salpeter IMF, the Kroupa IMF, the Chabrier
IMF), it is important that if you want to compare physical properties
you use the same IMF throughout. This is also true if you want to
compare your results to another study -- it is often the case that
astronomers need to ``convert'' their results to a different
IMF. Usually, however, this is simply a case of multiplying the
physical parameters (e.g., mass, star formation rate) by a constant
factor.

\subsubsection{Stellar evolutionary tracks, or isochrones}
The IMF gives us the relative numbers of stars of different masses at
time $t=0$. However, all but a tiny fraction of stars in a galaxy will
have an appreciable age (i.e., millions to billions of years
old). So, to be able to model the stellar population of a galaxy, we
need to know how the stars (whose masses are given by the IMF) evolve
over time.

Thankfully, how a star evolves is largely dictated by its mass, with a
lesser dependency on metallicity. Since we know the mass of every star
in our model (from the IMF), we can model how it will evolve across
the Herzsprung-Russel (HR) diagram. Of course, more massive stars will
travel across the HR diagram more quickly than low mass stars, since
they ``live fast, die young''. A line joining together all the
different stars on the HR diagram at a given time, $t$, is called an
{\it isochrone}; i.e., {\it iso} - meaning the same, {\it chrone} -
meaning time. At this stage, we can also specify a metallicity
dependence.

Provided we know our stellar evolutionary tracks, we will know the
positions on the HR diagram of all the stars on our model at any given
time.

\subsubsection{Stellar spectra}
At this stage, we can imagine our population of stars at time $t$ as
being represented by a whole load of points on the HR diagram. There
will be lots and lots of low mass stars that will have barely evolved,
and a handful of high mass stars that will have evolved very quickly
(indeed, some of the most massive may even have undergone
supernova). To generate the synthetic spectrum of the population, we
simply have to sum the spectra of all the individual stars represented
by those points.

Thankfully, we have observed spectra for stars across almost all the
HR diagram, so we can use real, observed spectra in most cases. In
some parts of the HR diagram where there are few stars, however, we
sometimes have to resort to synthetic spectra due to a lack of
observed spectra.

\subsubsection{Star formation histories}
Up to this point, we've only considered a single, instantaneous burst
of stars. However, it is unlikely that a galaxy will form all its
stars in one single burst. Instead, a galaxy will form its stars over
a protracted period of time. This will result in a different synthetic
spectrum. For example, if we have an extended period of
star-formation, then massive young stars (which are hot and therefore
emit strongly in the UV and blue part of the spectrum) will continue
to be produced. Therefore, the synthetic spectrum of a continuous
burst will stay bluer for longer than an instantaneous burst.

How do we model a continuous period of star-formation? Well, all we
need to do is take lots of instantaneous bursts going off one after
another. We can even modulate the size of each burst (i.e., how many
stars are produced in a given burst) according to a {\it star
  formation history}. For example, we could model a constant,
continuous episode of star formation by a set of bursts, one after
another, which all produced the same number of stars. Alternatively,
we could model an exponentially declining episode as a series of
bursts, each containing:
\begin{equation}
N_{\rm Stars}(t) = N_{\rm Stars}(t=0){\rm exp}\left( -\tau t \right)
\end{equation}
stars at time $t$ (where $\tau$ is a constant that described how
quickly the rate of star formation falls away with time). Then, we
simply treat each burst separately as described above.

\subsubsection{A gardening analogy}
Sometimes an analogy can really help to explain an idea; I like this
gardening one I came up with for generating synthetic spectra (it
doesn't contain caterpillars!).

Let's say I have a garden and some seeds. It's March, I'm just about
to plant my seeds, and I want to have an idea of what my garden will
look like in June, July and August.

We can think of the IMF as the relative numbers of different types of
seeds. Say 100 daisies, 50 marigolds, 20 sunflowers. The equivalent of
stellar evolutionary tracks would be how quickly each variety grows;
sunflowers grow really quickly, marigolds more slowly. Finally, the
stellar spectra would be the colours of the flowers: yellow
sunflowers, orange marigolds, white daisies. Given the relative
numbers of seeds, their growth rates and their colours, I can predict
what colours I will have in my garden during each month of the
summer. The whole, {\it fabulous} floral effect would be the synthetic
spectrum.

Finally, if I want to, I can plant lots of different sets of seeds
each week to prolong the flowering season. This would be the
equivalent of the star-forming history.

\subsection{Model fitting}
With modern computers, we can genarate a population of stars and
produce its synthetic spectrum extremely quickly (i.e., in a few
milliseconds). As such, given an observed spectrum/photometry of a
real galaxy, what astronomers typically do is model a whole range of
different metallicities, ages, and star-formation histories and see
what combination reproduces the data the best. This iterative
trial-and-error is simply a form of model fitting. Indeed, astronomers
often take a $\chi$-squared minimisation approach to find the
best-fitting synthetic spectrum to observed spectra or photometry.

\subsection{Caveats to using spectral synthesis}
In principle, spectral synthesis and model fitting is relatively
straight-forward. As such, it's relatively easy to use spectral
synthesis to measure galaxy properties. However, just getting a
measurement is only part of the effort. We must also consider how
reliable these measurements are. We must consider what could affect
the reliability of results from spectral synthesis and model
fitting. Here are considerations:

\begin{itemize}
\item {\bf Dust}: Since blue light is more readily absorbed by dust,
  if there is intervening dust then it will change the shape of the
  observed spectrum. Dust can make a population of stars look older
  than it is, because the blue light from the young stars is absorbed.
\item {\bf Uncertainties in post-MS tracks}: Any uncertainties in the
  paths that stars take across the HR diagram as they evolve will
  result in uncertainties in the resultant synthetic spectrum. In
  particular, there is a huge debate on how post-asymptotic giant
  branch stars evolve (especially if they're pulsating) because they
  are {\it really} bright so can contribute a lot to a galaxy's total
  spectrum.
\item {\bf Incomplete spectral libraries}: While we have taken spectra
  for many millions of stars, there remains some parts of the HR
  diagram that remian poorly sampled. In particular, since most of the
  stars in our region of the Milky Way have similar metallicities, we
  only have a few spectra of non-solar metallicity stars (and these
  are biased to more luminous stars). Further, since very massive
  stars don't live very long, they tend to be rare, making it
  difficult to find ones to measure their spectra.
\item {\bf Uncertain IMF}: We've already touched upon this
  point. Uncertainties in the IMF introduce uncertainties in the range
  of stellar masses, which subsequently introduces uncertainties into
  measured physical properties.
\item {\bf Uncertain star-forming histories}: One can imagine that a
  galaxy can have multiple episodes of star-formation during its
  evolution, possible with different regions having different
  histories. While, in priciple, we could model such complex
  histories, in reality there is not enough data to discriminate
  between such complex models. As such, we typically assume simple
  models (single burst, continuous, exponentially falling), which
  introduces uncertainties.
\end{itemize}

\section{Measuring integrated star formation rates}
\label{SFR}
One of the most important measurements we can make of a galaxy is its
star formation rate (SFR). This effectively tells us how quickly a
galaxy is growing, and is thus a fundamental aspect to understanding
how today's galaxies have formed. With SFRs, we can ask questions
like: When did galaxies grow fastest? Are today's galaxies growing
faster or slower than previously? What causes galaxies to grow more
quickly?

By modelling the stellar content of a galaxy, spectral synthesis {\it
  will} give us a measure of the SFR of a galaxy. Often, however, we
don't have to go to such lengths, as there are other ways to measure
accurate SFRs that rely on observations in just one band (i.e., UV,
infrared) or a single emission line.

All of the single-band/emission line measures of SFR rely on one key
fact: {\bf that the hottest, most massive stars die young}. The most
massive stars ($>5M_\odot$) will live for ``only'' around 100 million
years (compared to 10 {\it billion} years for the Sun). So, if we can
count the number of massive stars in a galaxy, then we can calculate
the average rate of {\it massive} star formation in that galaxy over
the past 100 million years (which is pretty instantaneous for a
galaxy!). However, the average rate of massive star formation {\it is
  not} the same as the actual SFR; recall that the IMF tells us that
for every massive star formed, there are many, many more lower mass
stars produced. So, to calculate the true SFR, we take the number of
massive stars, then multiply it by the ratio of the total mass of
stars per the total number of stars with $>5M_\odot$, i.e.:
\begin{equation}
\label{sfreqn}
{\rm SFR} = \frac{N_{\rm M>5M_\odot}}{\rm 10^8 yr}\times R_{\rm
  Stars}({\rm IMF})
\end{equation}
where, $N_{\rm M>5M_\odot}$ is the {\it observed} number of stars with
masses greater than 5~$M_\odot$ and $R_{\rm Stars}({\rm IMF})$ is
given by:

\begin{equation}
\label{reqn}
R_{\rm Stars}(\rm IMF) = \frac{\rm Total~mass~of~all~stars}{\rm Number~of~>5M_\odot~stars}
\end{equation}
which is calculated from our chosen IMF. As you should be able to see,
the SFR is in units of $M_\odot~{\rm yr^{-1}}$. 

So, to calculate a galaxy's SFR (assuming a given IMF), all we need to
do is measure how many stars with $M>5~M_{\rm \odot}$ it contains. How
can we do that?

\subsection{The UV luminosity of a galaxy}
Massive stars are very hot, so they produce a lot of ultraviolet
radiation. In fact, in the absence of an AGN, massive stars are pretty
much the only sources of UV radiation in a galaxy. As such, measuring
the UV luminosity of a galaxy provides a measure of the number of
$>5~M_\odot$ stars in a galaxy. With this in mind, astronomers have
calculated a simple conversion from UV luminosity to total SFR:

\begin{equation}
{\rm SFR~(M_\odot~yr^{-1})} = 4\times10^{-41} L(\rm FUV)~(erg~s^{-1}~A^{-1})
\end{equation}
As with all the other SFR indicators in this section, this takes into
account the increased luminosity of high mass stars relative to the
Sun, and for the corrections mentioned in \S\ref{SFR} and summarised
in Eqns. \ref{sfreqn} and \ref{reqn}, i.e., it is the ``true'' rate of
star formation in the galaxy, averaged over the past $\sim100$~Myr.

The main problem with using the UV emission to measure SFRs is that it
is readily affected by dust. Attempts can be made to correct for dust
extinction, but they rely on various assumptions and even a small
amount of dust can strongly affect measured UV flux.

\subsection{The H$\alpha$ luminosity of a galaxy}
Because hot, massive stars emit strongly in the UV part of the
spectrum, they emit a lot of photons with wavelengths shortward of
912~\AA . Photons shortward of this wavelength are energetic enough to
ionise Hydrogen. When this ionised hydrogen recombines with free
electrons, the electrons cascade through the atomic levels, emitting
photons as they drop to lower energyies. One of the strongest lines
produced by this process is the Balmer-alpha (i.e., H$\alpha$)
emission line at 6563~\AA , which corresponds to an electron
transition from $n=3$ to $n=2$. Measuring the luminosity of the
H$\alpha$ line therefore provides a measure of the number of young
stars in a galaxy, from which we can calculate the SFR:

\begin{equation}
{\rm SFR~(M_\odot~yr^{-1})} = 7.9\times10^{-42} L(\rm H\alpha)~(erg~s^{-1})
\end{equation}

Ionising Hydrogen requires particularly energectic photons, which only
the most massive, hottest stars are capable of producing. Indeed,
stars with masses below about 10~$M_\odot$ are unable to produce
sufficient numbers of high energy photons to ionise Hydrogen. This
means that H$\alpha$ traces only the very hottest, most massive stars
that live for {\it only} around 10~Myr. As such, the H$\alpha$ line
gives a measure of the SFR averaged over only the past 10 million
years. It is pretty much the most ``instantaneous'' measure of SFR
that we can get.

There are a few of caveats that come with using H$\alpha$ to measure
SFR. Firstly, like the UV, H$\alpha$ is affected by extinction due to
dust. However, it is not as badly affected by the UV and is easier to
correct-for using the ``Balmer decrement'' (the intrinsic ratio of the
H$\alpha$ to H$\beta$ lines is 3:1, so if we measure, say, a 4:1
ratio, we know it is being extinguished by dust, and can correct for
it). A second caveat is that some of the ionising photons may escape
the gas cloud without hitting a H-atom, meaning we'll measure a lower
SFR rate. Thirdly, since AGN are also prodigious producers of
H$\alpha$, it is difficult to use this to measure the SFR of galaxies
hosting AGNs (although these are the minority of galaxies).

\subsection{The infrared luminosity of a galaxy}
Most stars form out of gas and dust. When the massive stars start to
shine, their heat warms the dust that surrounds them and the dust
re-radiates this warmth in the form of infrared radiation. As with
H$\alpha$ and UV emission, it is only the most massive
($>5~{\rm M_\odot}$) stars that are hot enough to warm the surrounding
dust to produce this infrared radiation. As such, by measuring the
infrared luminosity arising from a galaxy, we can get a measure of the
number of young stars it contains, and thus its SFR:

\begin{equation}
{\rm SFR~(M_\odot~yr^{-1})} = 1.8\times10^{-44} L(\rm FIR)~(erg~s^{-1})
\end{equation}

Since stars with masses greater than about 5~${\rm M_{\odot}}$ are
capable of heating the dust, the infrared provides a measure of a
galaxy's SFR averaged over the past $\sim100$~Myr.

The benefit of using infrared wavelengths is that, unlike UV and
H$\alpha$ is it largely unaffected by dust obscuration. However, since
infrared wavelengths are readily absorbed by the Earth's atmosphere,
we need to use expensive space missions to observe at these long
wavelengths. The constraints on telescope size that this introduces,
combined with the long wavelengths, mean that infrared observations
typically suffer from poor spatial resolution.

\section{Learning objectives}
In this lecture, we have covered how we actually measure the physical
properties of galaxies. This is crucial if we want to be able to
build-up a coherent theory of galaxy exolution. The key learning
objectives you should take from this lecture are:
\begin{itemize}
\item Have an understanding of how spectral synthesis modelling can be
  used to determine the masses, SF histories and metallicities of
  galaxies.
\item Have an understanding of the main ingredients and uncertainties
  in spectral synthesis modelling.  
\item Appreciate of the main techniques used to determine the
  integrated SFR of galaxies and their pros and cons
\end{itemize}

\end{document}