   
\documentclass[11pt]{article}
\renewcommand{\baselinestretch}{1.05}
\usepackage{amsmath,amsthm,verbatim,amssymb,amsfonts,amscd, graphicx}
\usepackage{graphics}
\topmargin0.0cm
\headheight0.0cm
\headsep0.0cm
\oddsidemargin0.0cm
\textheight23.0cm
\textwidth16.5cm
\footskip1.0cm

 \begin{document}
 
\title{PHY405: Galaxy Evolution}
\author{Dr. James Mullaney\\F19 Hicks\\j.mullaney@sheffield.ac.uk}
\maketitle

\section{Course description}
This module will cover one of the most exciting and fast moving topics
in current astrophysics research, the formation and evolution of
galaxies from an observational perspective. Starting with a brief
historical introduction, the module will then summarise what we can
learn about galaxy evolution from studies of galaxies in the local
Universe, before discussing the results obtained from recent deep
field observations of the high redshift Universe. The last part of
the module will concern the important role that active galactic nuclei
play in galaxy evolution. The final lecture will cover the future of
galaxy evolution over the next $10^{100}+$ years.

\section{Administration}
\subsection{Lectures}
The course will consist of 18 lectures ($+$2 revision lectures) with
two lectures held per week:
\begin{itemize}
\item Monday 10-11am; Hicks Building F20
\item Tuesday 14-15am; Jessop Building 117
\end{itemize}

\subsection{Assessment}
\begin{itemize}
\item Exam: 70\%; Answer three out of five questions.
\item Coursework: 20\%; Paper summaries -- deadlines Friday 17$^{\rm
    th}$ March and Friday 28th April (10\% each).
\item Presentation: 10\%; 15 minutes ($+$5 minutes for questions) on
  a galaxy evolution topic pitched to your peers.
\end{itemize}

\subsection{Office Hours}
I'm happy to arrange dedicated office hours with you, should you
wish. Alternatively, please speak with me after lectures or contact me
to arrange a time.

\end{document}