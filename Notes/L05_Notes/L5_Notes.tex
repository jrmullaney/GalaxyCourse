   
\documentclass[11pt]{article}
\renewcommand{\baselinestretch}{1.05}
\usepackage{amsmath,amsthm,verbatim,amssymb,amsfonts,amscd, graphicx}
\usepackage{graphics}
\topmargin0.0cm
\headheight0.0cm
\headsep0.0cm
\oddsidemargin0.0cm
\textheight23.0cm
\textwidth16.5cm
\footskip1.0cm

 \begin{document}
 
\title{Lecture 5:\\Studying galaxy evolution via the fossil record}
\author{Dr. James Mullaney}
\maketitle

\section{Introduction}
Until now, we've largely considered investigating galaxy evolution by
exploiting the light travel-time to study galaxies in the early
Universe. We will continue to cover this ``high-redshift'' approach
later in the course. In this lecture, we will consider the alternative
(but complementary) approach: the fossil record. Here, we study nearby
galaxies in a lot of detail to try to understand how they have been
``built''.

\section{Early type -- or, Elliptical -- galaxies}
Elliptical galaxies are the big, boring ones without spiral arms. When
Hubble came up with his galaxy ``tuning fork'' diagram in the early
20$^{\rm th}$ century, it was thought that elliptical galaxies evolved
into spiral galaxies (i.e., left-to-right on the diagram), which is
why they are also known as ``early-type'' galaxies. We now know that
elliptical galaxies {\it do not} evolve into spirals but,
unfortunately, the name has stuck and so it's still used extensively
today.

\subsection{Stellar populations of elliptical galaxies}
As well as their shape, elliptical galaxies in the local Universe
share similar properties. They tend to be quite massive (stellar
masses of $\sim10^{11}~{\rm M_\odot}$ or above) and and are red in
colour, which suggests an old stellar population (i.e., the massive,
hot, blue stars have had time to evolve and die, leaving just the less
massive, cool red ones). When we analyse their spectra using spectral
synthesis (see Lecture 4), we indeed find that the spectra of the
majority ($>70\%$) of local ellipticals are best modelled by a
population of old stars with ages $>8$ billion years. This suggests
that today's massive ellipticals have remained mostly as they are
today for over half the age of the Universe.

\subsection{The colour-magnitude relation for elliptical galaxies}
When we study elliptical galaxies in more detail, we find there are
other consistencies within the class. While elliptical galaxies are,
in general, redder than their spiral counterparts, it's not the case
that they are all exactly the same colour. Indeed, some are ``redder''
than others.\footnote{In astronomy, we define a colour by the relative
  amounts of light in two bands, which means we can actually quantify
  how ``red'' or ``blue'' a galaxy is.} It turns out that the redness
of an elliptical galaxy is not random.  Instead, it is tightly
correlated with its total luminosity such that more luminous
elliptical galaxies are redder. 

Since the luminosity of a galaxy is related to how massive it is
(i.e., in general, the more stars a galaxy contains, the brighter it
is likely to be), this also means that more massive elliptical
galaxies are redder. Astronomers have confirmed this by plotting the
colour (i.e., ``redness'') of elliptical galaxies against their {\it
  velocity dispersion}, or $\sigma$, and find that redder galaxies
have higher velocity dispersions. Elliptical galaxies are supported
against gravity by the velocity of their stars (more on this later),
so their velocity dispersion gives a direct proxy measure of their
mass; the higher their velocity dispersion, the greater their mass.

The tightness of the relation between colour and mass/luminosity for
elliptical galaxies in a given cluster suggests that they were all
formed at roughly the same time (in that particular cluster). The
argument goes that if elliptical galaxies in a given cluster were
formed at different times, then they would show much more diversity in
their colours (i.e., there'd be some that formed their stars recently,
so would be blue, and some that formed their stars a long time ago, so
would be red). However, if they're all virtually the same age, then
why haven't they all got {\it the same} colour; why is there a colour
dependence on mass/luminosity?  That can be explained in terms of
metallicity. More massive galaxies hold onto their reprocessed gas
more easily than low mass galaxies because their gravity is stronger
(they have greater ``binding energies''). Due to their stronger
gravities, it is more difficult for stellar winds and supernovae to
push this enriched gas out of more massive galaxies. Reprocessed gas is metal
rich, so when it re-collapses to form more stars, it forms metal-rich
stars, which are redder than metal-poor stars of the same mass.

In summary, the slope of the colour-magnitude/luminosity/mass relation
for elliptical galaxies is due to metallicity effects. The tightness
of the correlation is because all ellipticals in a cluster formed at
roughly the same time.

\subsection{Light profiles of Elliptical galaxies}
As well as measuring the total flux and luminosity of galaxies, we can
also measure their light {\it profiles}, i.e., how their surface
brightness changes as a function of distance from their centre.

\subsubsection{Extended light profiles}
When we model the light profiles of elliptical galaxies, we find that
the light profiles {\it outside their central regions} fall off as
roughly $R^{-\frac{1}{4}}$:
\begin{equation}
\label{deV}
I(R)=I_e{\rm exp}\left(-7.67\left(\left(\frac{R}{R_e}\right)^{\frac{1}{4}}-1\right)\right)
\end{equation}
where $I_e$ is the surface brightness at the effective radius of the
galaxy, $R_e$, defined as the radius which contains half of the total
light from a galaxy.

On closer inspection of the light profiles of elliptical galaxies,
astronomers found that a more general form of Eqn. \ref{deV} gave a
better fit to the light profile:
\begin{equation}
\label{deV}
I(R)=I_e{\rm exp}\left(-b\left(\left(\frac{R}{R_e}\right)^{\frac{1}{n}}-1\right)\right)
\end{equation}
which is known as a {\it Sersic} profile. Typically, $n=4$ to 6.

\subsubsection{Core light profiles}
When we look at the very centres of elliptical galaxies, astronomers
find that some depart from the Sersic profiles. They do this in such a
way that their light profiles {\it flatten} within their centres
(within about 1 arcsec for nearby examples, corresponding to the
central few 100 pc). These are known -- rather ambiguously -- as {\it
  cores}. Thus, ellipical galaxies fall into two groups, those with
{\it cores} and those without (i.e., their light profiles keep rising
toward the centre). In general, galaxies with cores have higher
luminosities than those without.

Today, it is thought that the reason for the deficit of light in the
centres of ellipticals with cores is a consequence of how they
formed. It is thought that massive ellipticals are formed by major,
gas-poor -- or, dissipationless -- mergers (i.e., they don't contain
much gas, so the kinetic energy of the merger is kept within the
galaxy in the form of stellar kinematics, rather than being dissipated
away). During such mergers, the central supermassive black holes at
the centres of the two merging galaxies fall toward the central
regions of the final galaxy. As they do so, they ``throw-out'' stars
via gravitational sling-shot. This is known as ``scouring'' and is an
effective way of removing stars from the central regions of a galaxy
post-merger.

\subsection{Disky vs. Boxy ellipticals}
As well as core ellipticals, early type galaxies also separate into
two separate classes according to their morphologies. Some ellipticals
have a slightly ``rugby-ball'' shape and are known as ``disky'',
whereas others are ``squarer'' in appearance and are known as
``boxy''. Quantitatively, we distinguis between the two by performing
an angular Fourier series fit to the light profile of elliptical
galaxies and measuring the fourth term:
\begin{equation}
\Delta r(\theta)\approx\sum_{k\gtrsim3}a_k{\rm cos}k\theta + b_k{\rm sin}k\theta
\end{equation}
If $a_k>0$, then it is a disky elliptical, otherwise it is a boxy elliptical.

When we consider the two types of galaxies separately, it turns out
they differ in more ways than their morphologies. While all elliptical
galaxies rotate more slowly than spiral galaxies, disky ellipticals
tend to rotate more quickly than their boxy counterparts. Indeed, boxy
ellipticals are almost entirely supported by random motions, rather
than bulk rotation. Also, disky ellipticals tend to be less luminous
(and thus less massive) than their boxy counterparts. Boxy ellipticals
also tend to have ``cores''.

All this suggests that the two types of elliptical galaxies formed via
different paths. It is now thought that disky ellipticals formed at
high redshifts ($z\gtrsim1$) as a result of the merger of two gas-rich
galaxies. Since gas is dissipational, it would have lost energy, and
collapsed to form a rotating disk, producing lots of stars in the
process. Once the gas had finished producing stars, what is left is a
rotating elliptical - rotating because of the imprint of the gas, and
elliptical because the orbits of the original stars would have
undergone a degree of randomisation during the merger.

Boxy ellipticals are also thought to have been produced by merging
galaxies, but in this case are thought to have been produced by gas
poor (``dry''), dissipationless mergers. The stars were already
in-place prior to the mergers (and thus are very old; estimated to
have formed at $z>3$). In this situation, there is very little gas to
collapse into a rotating disk, so pretty much all the energy of the
merger goes into randomising the orbits of the stars. This creates a
galaxy that is entirely supported by the random motions of their
stars, and is boxy due to the lack of rotation. As explained above,
the cores are due to ``scouring'' due to black holes falling toward
the cores of the merger remnant. Unlike in disky ellipticals in which
new stars are formed by the gas-richer merger, in boxy ellipticals the
``scoured'' stars are not replaced by star formation.

\section{Spiral galaxies}
The other main type of galaxy on the Hubble diagram are spiral
galaxies. Since the Milky Way galaxy is a spiral, we can learn a lot
about spiral galaxies by studying our own Galaxy.

\subsection{The strucure of the Milky Way}
While the stars in the Milky Way are close compared to other galaxies,
establishing its structure is hampered by the fact that we sit inside
it. By measuring the distances to various stars within the Milky Way
we can, however, map-out its structure, in the same way that if you're
sat {\it inside} a wharehouse, you'd have a pretty good idea of the
{\it outer} dimensions of the wharehouse simply by looking around
you. In the case of the Milky Way, however, this is hampered by dust
along our line-of-sight, which blocks out the light from some
regions. This problem can be mitigated by observing at longer
wavelengths which more easily penetrate the dust clouds.

When we look at our Galaxy from within using penetrating infrared
wavelengths, we see that the Milky Way is a disk galaxy with a
bulge. Judging by the prominence of the bulge (and comparing to other
external galaxies) it is likely that the Milky Way is a ``late type''
spiral galaxy (i.e., it is to the right of the Hubble tuning fork),
probably an Sc or SBc galaxy. The latter classification (i.e., SBc) is
based on the shape of the Milky Way's asymmetric bulge, which sugests
it has a weak bar.

But that's just the overall appearance. Studying the fossil record of
the Milky Way involves much more detailed analyses of the stellar
content of the Galaxy, such as:
\begin{itemize}
\item {\bf Kinematics}: This is the motion of stars within the Milky
  Way relative to the Sun or, more generally, the Galactic Centre.
\item {\bf Metal Abundances}: Since stars generate metals as they
  evolve and, eventually, die in supernovae, the metal content of
  stars provide information on the ages and star-forming histories of
  stellar populations.
\item {\bf Ages}: Calculating the ages of indvidual stars is hard, but
  for populations, we can use the relative numbers of a population
  across the HR diagram (akin to the technique using in spectral
  synthesis) to calculate accurate ages of populations.
\item {\bf Precise positions}: Mapping out the above properties as a
  function of position relative to the Galactic centre provides clues
  as to how the Milky Way (or, for that matter, other galaxies) was
  built-up.
\end{itemize}

By considering all the above, we now know that the Milky Way is made
up of different populations of stars which make up four different
components of the Galaxy: the {\it Thin Disk}, the {\it Thick Disk},
the {\it Bulge}, and the {\it Halo} populations.

\subsubsection{The Thin Disk}
The thin disk is made up of relatively young, metal rich stars (known,
confusingly, as Population {\sc I} stars. At the location of the Solar
System, this disk rotates {\it in bulk} around the centre of the Milky
Way at a speed (relative to the centre) of about
220~${\rm km~s^{-1}}$. This disk is about 50~kpc in diameter (outer
edge to outer edge), but has a scale height of just 325~pc (i.e.,
$\rho(z)\propto e^{-z/325~{\rm pc}}$, where $\rho$ is the density of
stars and $z$ is distance perpendicular to the disk). The ratio of
thickness to diameter of the thin disk is roughly the same as a vinyl
LP or a CD; it is {\it very} thin! 

The Milky Way is still continuously forming stars at a rate of about
1~${\rm M_\odot~yr^{-1}}$, most of which takes place within the thin
disk. The thin disk contains a lot of gas and dust from which these
stars are being made. Today this disk is thought to be what's left of
the dissipational collapse of a large gas cloud that formed the
fledging Milky Way about 8 billion years ago.

\subsubsection{The Thick Disk}
In addition to the Thin Disk, there is a Thick disk which, as you may
expect, has a larger scale height ($\approx$1000~pc), but which has
roughly the same diameter. The thick disk is made up of older, more
metal-poor stars (Population {\sc II}) than the thin disk. Thick disk
stars orbit slightly more slowly around the Milky Way's centre -- at
around 180~${\rm km~s^{-1}}$ -- compared to thin disk stars. As they
rotate, however, they also oscillate up and down throughout the thick
disk, so at any given time, some of the thick disk stars are {\it
within} the thin  disk. However, the distance between two stars in the
thin disk is so large that the likelihood of collision between thin
and thick disk stars is tiny.

It is thought that the thick disk is what is left over from when the
very young Milky Way collided with a smaller galaxy around 8 billion
years ago (i.e., just as the natal gas cloud was collapsing to form
the thin disk). The stars in the then collapsing thin disk were
``thown out'' by this collision, which formed the ``puffed up'' thick
disk. Because kinetic energy was tranferred from the galaxy collision
to partially {\it randomize} the stellar motions, this process is
known as ``heating'' (elliptical galaxy have very ``hot'' stellar
kinematics because they are almost entirely help-up by random
motions). Because there is little gas within the thick disk (other
than at the point it is cospatial with the thin disk), it does not
form stars, so the thick disk stellar population provides a snapshot
of what the thin disk was like very early-on in the Milky Way's history.

\subsubsection{The Bulge}
The bulge of the Milky Way is roughly spherical in shape, with a
radius of about 3~kpc. It is formed from metal-rich, Population {\sc
  I} stars, the rotation velocity of which is proportional to their
metallicity. The stellar density of the bulge is significantly higher
than the rest of the galaxy with, on average, $5\times10^4$ stars per
cubic parsec.\footnote{This means that the average distance between
  two stars in the bulge is about 0.03~pc, compared to the 1.3~pc
  between the Sun and our nearest star, Proxima Centauri. Yet, 0.03~pc
  is still about 120 times larger than the radius of Pluto's
  orbit. Yep, space is {\it big}.} That's just the average, the
stellar density profile of the bulge falls off as $\approx
r^{-2.2}$, so it is even more dense in the centre.\footnote{You should
be able to take this density profile and use the above numbers to
calculate the average density of the central 10~pc.}

The bulge is thought to have a very complex formation history. Some of
the stars in the bulge are likely to have formed early on in the
history of the Milky Way from the natal cloud of gas. Then, later, the
successive accretion of satellite galaxies likely heated the bulge up
(helping to make it the shape it is) and introduced more metal-rich
stars.

\subsubsection{The Halo}
Finally, we have the low-density spherical halo of stars that
surrounds the entire Milky Way. This consists of very old, metal poor
stars, that orbit the milky way very slowly (at about a charectristic
velocity of about 40~${\rm km~s^{-1}}$. The density of the halo drops
off as $r^{-3}$.

It is believed that many of the halo stars are the remnants of the
tidal disruption of satellite galaxies that ultimately merged with the
Milky Way. These stars were formed early-on in these satellite
galaxies, then were tidally stripped as the galaxy orbited and,
eventually, fell into the Milky Way. There is also the prospect that
some of the halo stars were formed out of the natal gas cloud that
would eventually collapse to form the Milky Way.

\subsection{Streams around the Milky Way}
As we have seen, the merger of satellite galaxies can explain a lot of
the bulk features of the Milky way. Today, there is significant
evidence that the Milky Way has cannibalised many smaller, satellite
galaxies over the past few billions of years. The most striking of
such evidence are the ``Fields of Streams'', which are streams of
stars that loop around the Milky Way.

As a smaller satellite galaxy orbits and spirals into a larger
satellite like the Milky Way, the differential gravitational pull on
one side relative to the other causes it to get stretched-out and
ripped apart. The resulting ``stellar debris'' forms impressive loops
around the galaxy, which are known as ``tidal streams'' (tidal forces
are those created by differential gravitational pulls; the tides on
the Earth are caused by the Moon's gravitational pull being different
on one side of the Earth compared to the other). 

\subsection{Is the Milky Way normal?}
If we're going to use the Milky Way as an example of how spiral
galaxies have formed, we'd better check that it is typical of the
population. It turns out, in fact, that the Milky Way is somewhat
atypical of Spiral galaxies since, as far as we can tell, it falls
below the Tully-Fisher relationship (i.e., the correlation between
orbital velocity and the luminosity of a spiral galaxy). It is thought
that this may be because the Milky Way has actually undergone {\it
  fewer} merger events compared to other spirals, which has maintained
its high rotation speed relative to its mass (i.e., other spirals are
``hotter'' and have more support from random motions). Despite this,
the Milky Way isn't {\it drastically} different from other spirals, so
it can give us a decent ``first-order'' view of how spiral galaxies
have formed.

\section{The local galaxy population}
Going beyond the Milky Way, we can get a better understanding of
galaxy evolution {\it in general} by surveying galaxies in the local
Universe. When we do this, we find that the galaxy population forms a
bi-modal distribution in terms of their colours. There is a population
of red galaxies -- known as the ``red sequence'' when plotted on a
colour-magnitude plot -- and blue galaxies --known as the ``blue
cloud''. Red-sequence galaxies are mainly ellipticals and tend to be
more massive than blue cloud galaxies (which are mainly
spirals). Their colours give a clue to their star-forming activities
-- galaxies in the blue cloud are star-forming, whereas red sequence
galaxies tend to be gas-poor and ``dead''.

\subsection{The population fossil record}
With the availability of spectra for tens of thousands of nearby
galaxies, astronomers can perform spectral synthesis on the whole
local population to work out their star-forming histories. In doing
so, they have been able to piece-together the average star formation
rates of the Universe throughout its history.

For example, say we perform spectral synthesis for a single galaxy --
galaxy A -- and find it formed a quarter of its stars in a burst 1~Gyr
ago, another quarter of its stars were formed in a burst 5~Gyr ago,
and the remaining half was formed continously over the past 10~Gyr. We
could then plot the star formation rate of this galaxy A as a function
of time -- it would have a two sharp spikes of star-formation
superimposed on a continuous low-level. Then, we could consider the
next galaxy -- galaxy B -- and add its history of star-formation to
our plot. Then move onto the next, and the next, each time adding the
star-forming histories to our plot. Eventually, if we did this for all
the nearby galaxies, we'd have a plot of the total star-forming
history of the local Universe. If we then assume that we don't live in
a special place in the Universe (i.e., the Extended Copernican
Principle) then we can apply this result to the whole Universe, i.e.,
we can say it is a reasonable measure of the star-forming history of
the Universe.

When astronomers do this, they find that the star-forming history of
the Universe peaked at about redshift 2, corresponding to roughly
10 billion years ago. Since this time, it seems that the rate of
stellar production in the Universe has slowed significantly -- by a
factor of $>10$. We'll explore why this is the case later in the
course.

Finally, since we have pieced together the global star forming history
from individual galaxies, we can also explore the star-forming
histories of galaxies split into various different subcategories
(e.g., what's the global star-forming histories of spirals
vs. ellipticals? or for galaxies of different mass?). A very important
result that we get when we do this is that the rate of star formation
for today's most massive galaxies peaked at earlier times than less
massive galaxies. In other words, today's massive galaxies started to
form {\it first}, with less massive galaxies forming later. This
process is known as {\bf Cosmic Downsizing}.

\section{Learning objectives for Lecture 5}
\begin{itemize}
\item Knowledge of what detailed observations of nearby galaxies (the “fossil record”) tells us about the evolution of both spiral and elliptical galaxies
\item An appreciation of how less detailed (statistical) studies of large samples of galaxies in the local Universe aid our understanding of galaxy evolution
\item Familiarity with the concept of cosmic downsizing
\end{itemize}

\end{document}