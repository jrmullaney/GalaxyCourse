   
\documentclass[11pt]{article}
\renewcommand{\baselinestretch}{1.05}
\usepackage{amsmath,amsthm,verbatim,amssymb,amsfonts,amscd, graphicx}
\usepackage{graphics}
\topmargin0.0cm
\headheight0.0cm
\headsep0.0cm
\oddsidemargin0.0cm
\textheight23.0cm
\textwidth16.5cm
\footskip1.0cm

 \begin{document}
 
\title{Lecture 9:\\The evolution of early-type galaxies}
\author{Dr. James Mullaney}
\maketitle

\section{Introduction}
Now that we have covered the many ways astronomers study galaxy
evolution -- through simulation, deep multiwavelength surveys, and the
fossil record -- we'll explore how we currently believe the main types
of galaxies in today's Universe have evolved to their current
state. In this lecture, we'll cover early type (i.e., elliptical)
galaxies, while in the following lecture we'll consider the evolution
of spiral galaxies.

\section{Ellipticals in the context of all galaxies}
Recall from Lecture 5 that elliptical galaxies in the local Universe
tend to be fairly massive, with masses typically between $10^{10}$ and
$10^{12}~{\rm M_\odot}$. Indeed, the most massive galaxies in today's
Universe are elliptical galaxies that reside in the centres of dense
galaxy clusters. By contrast, spiral galaxies (which we'll consider in
the next lecture) dominate the numbers of galaxies with masses below
about ${\rm 10^{10}~M_\odot}$ galaxies.\footnote{In that respect, the
  Milky Way and Andromeda galaxies are relatively rare, being both
  massive and spiral galaxies.}

Roughly 50\% of the stellar mass in today's Universe is contained
within elliptical galaxies, which also means that they contain about
half of all stars by number. If we also include the {\it bulges} of
disk galaxies in that number (which are sort of like mini-ellipticals)
then that proportion goes up to as high at 70\%. By that measure,
so-called {\it spheroids} (which groups together ellipticals and
spiral bulges) are the most important group of objects in today's
Universe.

\section{A recap of ellipticals from the fossil record}
Another thing we covered in Lecture 5 was that elliptical galaxies
tend to be quite red in colour, but also that their colour is tightly
correlated with their mass: more massive ellipticals tend to be redder
than less massive ellipticals.\footnote{You should be able to explain
  why this is the case.} We also saw how there are different types of
elliptical galaxies that separate roughly according to mass:
\begin{itemize}
\item Boxy ellipticals tend to be the more the massive of the two, and
  have ``cores''. They are almost entirely supported by the random
  motions of their stars.
\item Disky ellipticals tend to be the less massive of the two, and
  tend not to have ``cores'' - their stellar density continues to rise
  toward their centres. They are supported by a combination of
  rotation and random motions.
\end{itemize}
It is widely regarded that these differences arise due to different
formation and evolutionary histories, with disky ellipticals forming
via the merger of gas-rich galaxies and boxy ellipticals forming via
the merger of gas-poor galaxies. Judging by the fossil record of disky
ellipticals, it seems that many of their stars have been formed
relatively recently (since $z\sim1$), during an episode of intense
star-formation triggered by the gas rich merger. By contrast, there is
very little evidence of any recent star-formation in massive, boxy
ellipticals.

\section{The disky elliptical formation link to ULIRGs}
If disky ellipticals are formed via the major-merger of gas-rich
galaxies, then does this mean that the ULIRGs we observe in the local
Universe are ellipticals in the making? We saw in the last lecture
that ULIRGs are produced when two gas rich galaxies collide, their
high star formation rates a result of the merger process compressing
the gas and causing it to collapse to form stars. But, even though we
see these collisions taking place, it doesn't necessarily mean that
they will form elliptical galaxies. After all, a major galaxy merger
will typically last for tens of millions of years, so we can't ``wait
and see'' what will happen.

One fairly strong piece evidence that major mergers in the local
Universe will ultimately lead to elliptical galaxies are the motions
of stars (i.e., stellar kinematics) within ULIRGs. These are
comparable in magnitude to those of elliptical galaxies and have a
significant random component. Crucially, the level of rotational
support vs. support from random stellar orbits is closer to that of
disky ellipticals, rather than boxy ellipticals. This makes sense in
terms of our understanding of how disky ellipticals are formed. The
rotation is left-over from the dissipational collapse of the gas
within the merging galaxies into a rotating, star-forming disk. The
random orbits are those of stars that existed in the two merging
galaxies {\it prior} to the merger. Since stars are not dissipational,
any rotational velocity they had prior to the merger is converted into
random motions post-merger. Finally, even the morphologies of ULIRGs
in the final stages of major mergers are similar to those of disky
ellipticals.

A significant amount of evidence therefore suggests that local ULIRGs
represent the transformation of gas-rich spirals into disky
ellipticals via major mergers. As such, galaxy evolution is not
confined to the high redshift, early Universe. However, more massive
disky elliptical galaxies tend to have older stellar populations
compared to low or moderate mass disky ellipticals. This means more
massive disky ellipticals must have formed via gas-rich mergers in the
early Universe, whereas less massive disky galaxies formed from more
recent mergers. This is in agreement with the ``downsizing''
description of the Universe in which the most massive galaxies formed
first. Finally, the star-forming histories of local disky ellipticals
are consistent with a large fraction of their stars being formed
within a very intense, very short period of star-formation -- exactly
what results from a gas-rich major merger.

So, the formation and evolution of disky ellipticals can be summarised
as:
\begin{itemize}
\item The most massive disky ellipticals were formed by major, gas
  rich mergers in the early Universe, as evidenced by their old
  stellar populations and rotation (due to dissipative gas collapse).
\item Less massive disky ellipticals also formed via the major merger
  of gas rich galaxies, but at later times. This is another case of
  {\it cosmic downsizing}.
\end{itemize}
In both cases, a large fraction of the stars in disky ellipticals were
formed in a short period of time due to the intense levels of
star-formation induces during the merger. This is supported by
the measured star-forming histories of disky ellipticals.

\section{The formation and evolution of boxy ellipticals}
As we have seen (from Lecture 5 and above), our current understanding
of elliptical galaxies is that major galaxy mergers have played a
crucial role in their formation. For disky ellipticals this means
major, gas-rich mergers. But what about the other type of elliptical
galaxies -- the boxy ellipticals that dominate the numbers of high
mass elliptical galaxies?

Boxy ellipticals show very little evidence of rotation, which
indicates that their mergers were highly non-dissipational. This means
they must have contained very little gas, as gas is {\it very}
dissipational. As we know, this strongly suggests that boxy
ellipticals are formed by gas-poor (also known as ``dry'')
mergers. Since gas-poor mergers do not form stars (since gas is needed
to form stars), this means that all of the stars that make up a
massive boxy elliptical galaxy were already in place prior to the
merger. All the merger process did was randomise, or ``heat-up'' the
orbits of the stars that already existed pre-merger. Since no stars
are produced in the merger process, the stellar populations in boxy
ellipticals should have simply been evolving ``passively'' for the
past few billion years.

Do we see evidence of this passive evolution? To investigate this,
astronomers need to find populations of massive, boxy ellipticals at
higher and higher redshifts. At any redshift, the most massive
galaxies always live in the most dense environments: massive clusters
and superclusters. So, to find large numbers of boxy ellipticals at
different redshifts, astronomers search for increasingly distant
clusters of galaxies. To determine the evolutionary state of these
massive ellipticals, astronomers measure their rest-frame colours. If
they have evolved passively since their formation, then boxy
ellipticals at high redshifts should be bluer than local boxy
ellipticals by a predictable amount.

When astronomers plot the colour-magnitude diagram for massive
ellipticals in dense clusters at different redshifts, they do indeed
find that earlier ellipticals are bluer by {\it precisely} the right
amount to be explained by passive evolution. This has been confirmed
up to at least $z\sim1$, indicating that massive, boxy ellipticals have
evolved passively (i.e., have not formed any significant numbers of
stars) for {\it at least} the past $\sim$6 billion years (or roughly
half the age of the Universe). Further, neither the slope nor the
scatter around the colour magnitude diagram has changed significantly
over this time. This suggests that the stellar populations in massive
ellipticals were already well-evolved even before $z\sim1$.

\subsection{Identifying elliptical galaxies at $z>1$}
Finding dense clusters of galaxies at higher and higher redshifts
becomes increasingly difficult for both practical (they're faint) and
physical (they become less dense, and thus more difficult to detect)
reasons. As such, beyond about $z>1$, astronomers use other techniques
to identify elliptical galaxies.

Since elliptical galaxies don't contain lots of hot, massive stars
producing UV photons (i.e., they're not strongly star-forming), we
can't use either the Lyman break technique to identify them, nor
exploit the benefits of the positive $K$-correction in the sub-mm.
Instead, astronomers use the fact that the old stellar populations of
massive ellipticals will look extremely red due to their age. As such,
they can use colour selection to find very red objects, often
exploiting rest-frame near-infrared observations, which are
particularly sensitive to populations of old, cool stars which
dominate in massive ellipticals. Because of their colours, such
galaxies are known as {\it Extremely Red Objects} (EROs). Although it
is an effective means of identifying old populations, astronomers need
to be careful to avoid galaxies that are, instead, red due to dust
absorption. Indeed, roughly half of EROs are, in fact, reddened
star-forming galaxies.

On studying the stellar populations of non-star-forming EROs at
$z\sim1.5-1.8$ via spectroscopy, astronomers find that their stellar
populations are roughly 3.5~Gyr old. This is quite incredible,
considering that the age of the Universe at $z\sim1.8$ is only
3.7~Gyr! Clearly, this suggests that stars in early EROs must have
formed very soon after the Big Bang. Even more incredibly, some of
these galaxies at $z\sim1.5-1.8$ have masses of $10^{11}$ to
$10^{12}~{\rm M_\odot}$, indicating that they must have formed hundreds of
billions of stars in the space of just a few hundred million years!
Interestingly, however, their sizes are very compact, having a
half-light radius of about a 1~kpc compared to the $\sim5$~kpc of
similar mass galaxies today.

\subsection{A consistent evolutionary theory for boxy ellipticals}
Collating all this observational evidence, astronomers have developed
a consistent evolutionary theory to explain the appearance of massive
(boxy) ellipticals at different redshifts. Their old stellar
populations (even at high redshift, when the Universe was only a few
Gyr old) indicates that the stars in massive ellipticals were formed
at a very early time - just a few hundred million years after the Big
Bang. However, these were not formed in-situ. Instead, it is thought
that they were formed very early-on by the collapse of lots of
comparatively small clouds of gas. Initially, these clouds of gas
would have collapsed dissipationally, forming rotating stellar
disks.

In the dense regions of a natal cluster, these early rotating disks of
stars (which would have quickly exhausted their gas supply) would have
soon merged together in dry major mergers.\footnote{Remember,
  ``major'' simply refers to the {\it ratio} of galaxy masses, not the
  absolute mass. The merger of two small galaxies (e.g., each of
  $10^8~{\rm M_\odot}$) would still be referred to as a major merger,
  as the ratio of galaxy masses is 1:1} This explains why we see
ellipticals in the early Universe that contain old stars -- the stars
are old because they formed in these dense natal clusters from
collapsing gas clouds soon after the Big Bang, and they are elliptical
due to the dry merger of all these smaller, natal galaxies in rapid
succession. An important consequence of successive, major dry mergers
is that the galaxies will ``puff-up'' in size, which explains why
today's massive ellipticals are larger (in physical extent) than
similar mass galaxies in the early Universe.

So, to summarise, it is thought that massive ellipticals formed by:
\begin{itemize}
\item Clouds of gas collecting in the densest parts of the early
  Universe (which would become today's clusters and superclusters).
\item These clouds collapsing to form stars in small ``natal''
  galaxies just a couple of hundred million years after the Big Bang.
\item The successive non-dissipative mergers of these small, now
  gas-poor galaxies which randomise the orbits of these stars
  (destroying any diskiness and rotation) and ``puffs them up'' in
  size.
\item Since the mergers are gas poor, little or no new stars are formed
  after the initial collapse of the early gas clouds.
\item After a few billion years, all of the small galaxies that are
  going to merge have already done so, meaning that the masses of the
  most massive ellipticals stay largely the same from $z\sim1$.
\end{itemize}

\section{The evolving elliptical galaxy luminosity function}
The different evolutionary paths of massive (boxy) and less massive
(disky) ellipticals can be summarised nicely in terms of the evolving
elliptical galaxy luminosity function.

When astronomers measure the luminosity functions of elliptical
galaxies at various redshifts, they find that the number density of
the brightest (and thus most massive) elliptical galaxies in the
Universe has barely changed over the past $\sim6$ billion years. As
such, the numbers and masses of the most massive (and thus boxy)
elliptical galaxies in the Universe has remained largely the same:
these galaxies were already in place in the early Universe and have
barely evolved since.

By contrast, the low-luminosity end of the elliptical galaxy
luminosity function has evolved considerably. Today, there are many
more low mass (and thus disky) elliptical galaxies than there
were in the early Universe. This implies that gas-rich mergers have
been slowly building up the numbers of disky ellipticals over the past
13 billion years to ``catch up'' with the numbers of massive boxy
ellipticals that were formed much earlier on.

Thus, the evolution of elliptical galaxies is perfectly consistent
with the idea of {\it downsizing}: that the most massive galaxies
formed first in the early Universe, followed by more moderate, and
then low-mass galaxies. It's just that the two different types of
elliptical galaxies (boxy vs. disky) have formed via different routes.

\section{Lecture 9 learning objectives}
Some of this lecture is a recap of what we covered in Lecture 5 on the
fossil record. As such, there has not been as much new material to
read as usual. You may now want to take another brief look through the
Lecture 5 notes to ensure that it all makes sense to you. Since
massive ellipticals contain a large fraction of all the stars in the
Universe, understanding how they have formed and evolved is a major
component of our understanding of galaxy evolution overall. Here are
the learning objectives for this lecture:

\begin{itemize}
\item Knowledge and understanding of the fossil record for early-type galaxies in the local Universe
\item Understanding of the methods used to detect early-type galaxies in the distant Universe
\item Knowledge of the main results obtained for early-type galaxies at high redshifts
\item Understanding of the main evolutionary trends with redshift for both high and low mass early-type galaxies
\item Understanding of the concept of “cosmic downsizing”
\item Be able to describe the formation and evolution of both boxy and
  disky ellipticals.
\item Understand how the elliptical galaxy luminosity function supports
  these our understanding of their evolution.
\end{itemize}


\end{document}