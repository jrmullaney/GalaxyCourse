   
\documentclass[11pt]{article}
\renewcommand{\baselinestretch}{1.05}
\usepackage{amsmath,amsthm,verbatim,amssymb,amsfonts,amscd, graphicx}
\usepackage{graphics}
\topmargin0.0cm
\headheight0.0cm
\headsep0.0cm
\oddsidemargin0.0cm
\textheight23.0cm
\textwidth16.5cm
\footskip1.0cm

 \begin{document}
 
\title{Lecture 7:\\The star-forming history of the Universe \& Lyman Break Galaxies}
\author{Dr. James Mullaney}
\maketitle

\section{Introduction}
In the previous lecture, we covered the general philosophy and
techniques of survey astronomy. In this lecture, we will cover two
important results that are rooted in data from extragalactic surveys:
the cosmic history of star formation, and Lyman Break Galaxies.

\section{The cosmic history of star formation}
Galaxies are made of stars, so asking ``when did today's galaxies
form?'' is akin to asking ``when did today's galaxies form their
stars?''.\footnote{We'll leave {\it how} stars are formed to Simon's
  Star Formation course.} In lecture 5, we looked at how we could use
the fossil record to determine this: effectively combining the star
forming histories of all local galaxies to determine when their stars
were put in place. We can also, however, try to answer this question
using the redshift approach. We can study galaxies at different
redshifts (i.e., at different {\it lookback times}) and measure how
much star formation was taking place at different times throughout the
history of the Universe. We do this by combining the methodologies
outlines in lectures 4 and 6: using UV or infrared surveys to survey
the star formation rates (SFR) of galaxies at different times during the
history of the Universe. From this, we can determine -- on average --
when galaxies formed their stars.

\subsection{The Madau diagram}
One of the first attempts to systematically measure the star-forming
history of the Universe using blank field surveys is described in
Madau et al. (1996). As such, the resulting plot of SFR vs. redshift
is known as the {\it Madau diagram} or {\it Madau plot}. Madau used
rest-frame UV emission (from the Hubble space telescope) to measure
the average SFRs of galaxies out to a redshift of $z\sim5-6$ (i.e.,
over the past $\sim12$ billion years (or $\sim90\%$ of the age of the
Universe). What Madau et al. found was that the {\it SFR density} --
that is, the average SFR per unit volume -- appeared to peak at around
$z\sim1-2$, corresponding to about 6 to 8 billion years ago. Since
that time, the SFR density of the Universe has slowly declined until
today when it is about one tenth of what it was at its peak. It's like
galaxies in the Universe had a growth spurt around 6-8 billion years ago,
and has now settled into middle age. Further, since heavier elements
(i.e., remember, in astronomy anything heavier than He is a {\it
  metal}) are formed in the cores of stars and when stars go
supernova, the peak of SFR at $z=1-2$ also means that the rate of
metal production also peaked at these times.

It's interesting to consider just how rapidly galaxies were forming
their stars when they were at their peak SFRs. From the Madau plot, we
see that the SFR density of the Universe peaked at around
$0.1~{\rm M_\odot}$ per year per ${\rm Mpc^3}$. That may not sound
like much, but bear in mind that most of the Universe is empty. As
such, some galaxies were producing well over a hundred stars {\it per
  cubic kpc} per year.

Because of the impact that Madau et al. (1996) had on the field of SFR
history, the terms ``Madau diagram'' or ``Madau plot'' have now become
synonymous with all subsequent plots of SFR density vs. redshift (no
matter who publishes them).

\subsection{Uncertainties in the Madau diagram}
Of course, as with any measurement -- especially first attempts --
there were considerable uncertainties associated with the earliest
attempts to measure the SFR density of the Universe. In particular, by
using rest-frame UV light to measure SFRs, the earliest Madau diagrams
suffered significantly from uncertainties due to dust obscuration
(recall, the UV is {\it strongly} attenuated by dust). Furthermore, as
we saw in Lecture 2, there are significant difficulties associated
with measuring light from high redshift galaxies. In particular,
Hubble images are strongly biased toward detecting high surface
brightness galaxies (remember the $1/(1+z)^4$ law for surface
brightness from Lecture 2), meaning they could easily miss any diffuse
regions of star formation and thus underestimate the SFR
density. Finally, we also have the usual problems associated with
converting UV fluxes to SFRs: how to convert UV luminosity to
numbers of high mass stars (which depends on uncertain models of
stellar evolution) and from there to total numbers of stars (which
depends on uncertain IMFs, especially in the high redshift Universe). 

As our telescopes have become more sensitive at more wavelengths
(especially infrared wavelengths), our ability to mitigate these
uncertainties has improved (consider material in L2). As such, since
its first depiction in 1996, the Madau diagram has evolved somewhat,
especially at the high redshift end. While it remains the case that
the SFR density of the Universe peaked at around $z=1-2$, the most
recent Madau plots show less of a steep decline {\it at higher
  redshift} than first thought. In other words, the SFR density of the
Universe was fairly constant (if slightly increasing) from between
$z=6$ to about $z=1-2$, after which it has dropped significantly. As
we shall see later in the course, the reason for this drop-off in SFR
density over the past 6-8 billion years is due to the declining
availability of gas -- the raw material of star formation -- in the
Universe as it gets used up to form stars.

\section{Lyman Break Galaxies}
Identifying galaxies at high redshift -- as required if we want to use
this approach to study galaxy evolution -- can be extremely
challenging. Not only do we have to contend with the faintness of
these galaxies due to their distance from us (as outlined in Lecture
2), but once you have a sample of galaxies, how do you reliably
identify the few that are at the highest redshifts out of the millions
of more local galaxies? Even with spectroscopic surveys, it would be
highly inefficient to target all galaxies with the hope of finding the
small handful that were at the highest redshifts.

One approach we saw in the last lecture was to use photometric
redshifts. However, that approach requires observations in {\it many}
different bands and so can be costly (in terms of telescope time). In
this section, we'll look at a way we can identify high redshift
galaxies using as little as two wavebands. This technique relies on
exploiting a ``break'' in a galaxy's spectrum -- typically where the
flux lonward of the break is higher than shortward of the break. One
of the strongest breaks in the spectra of galaxies is the {\it Lyman}
break, so galaxies that are identified by exploiting this break are
known as {\it Lyman Break Galaxies} (we like to be original in astronomy!).

\subsection{Identifying Lyman break galaxies}
Before we consider how we actually identify Lyman break galaxies,
we'll first look into what the ``Lyman break'' actually is. Consider a
galaxy containing a population of massive, hot stars. Because they are
hot, they release a strong continuum of UV light. Usually, these stars
will be sitting in a pool of surrounding gas (containing mainly
Hydrogen) from which they have been born. This means that, almost
immediately after the UV photons leave the surface of the stars, those
with wavelengths shorter than 912\AA\ (and thus capable of ionising H)
hit a neutral H atom and is {\it absorbed} by it, producing a H ion. The H
ion will (eventually) recombine with an electron but -- importantly --
it is unlikely to recombine directly into the lowest energy level. It
is far more likely that it will recombine to a higher energy level,
emitting a lower energy photon than the original ionising photon followed
by a series of other transitions. As such, it is {\it highly unlikely}
that the original $<912$\AA\ will be re-emitted. What all this means is
that, because of the ready absorption of $<912$\AA\ photons by
Hydrogen, there is a sudden drop in flux shortward of this wavelength
-- known as a ``break''. And because it is caused by absorption by the
first, or {\it Lyman}, level of Hydrogen, it is known at the {\it Lyman
  Break}.

As mentioned already, astronomers can use this break to identify high
redshift galaxies. To explain how this works, however, we'll first
consider a galaxy at $z=0$. In this low redshift example, we'd detect
the galaxy at wavelengths {\it longward} of 912\AA, but not {\it
  shortward} of 912\AA , since these short wavelengths photons are
absorbed by the hydrogen gas in the galaxy. For galaxies at higher
redshifts, however, the break {\it shifts} to {\it longer}
wavelengths, from the UV to optical wavelengths. For example, for a
galaxy at $z=3.2$, the break will be at $912\times(1+3)=3830$~\AA
. This {\it redshifted} break at 3830\AA\ now lies between the U and B
wavebands (centred at 3650\AA\ and 4450\AA ), but the same principle
still applies. A bright, star-forming galaxy at $z=3.2$ would be
detected in the B band, which samples {\it longward} of the break, but
it would not be detected in the U-band, which samples {\it shortward}
of the break. Such galaxies are referred to as ``U-band {\it
  dropouts}'' because they are said to have {\it dropped-out} of the
shorter-wavelength U-band. Because of the wavelength separation
between the U and B filters, these bands are sensitive to dropouts
between $z\approx3650/912-1=3$ and $z\approx4450/912-1=3.9$.

By using longer wavelength bands, astronomers can identify Lyman-break
galaxies at higher and higher redshifts. For example, using the B and
V (5500\AA ) bands would identify B-band dropouts between $z\approx4$
and $z\approx4.9$, while the V and R (6580\AA ) bands would identify
V-band dropouts between $z\approx4.9$ and $z\approx6.1$. As a test,
what redshift could an I-band dropout have (centred at 8060\AA ; the
next longest wavelength is the z-band at 9000\AA )? What type of
dropout would the highest-redshift galaxy, at $z=11.09$ be? (You'll
need to look up the central wavelengths of common telescope
filters. {\it Clue}: Check out the near-infrared)

Since their discovery, a large number of Lyman break galaxies
(hereafter, LBG) have had their redshifts confirmed spectroscopically
using large (8-10~m class) telescopes. The dropout technique makes
this feasible -- out of the thousands or even millions of galaxies in
the deep field surveys, only a few thousand might be LBGs, which can
(relatively) easily be followed-up with multi-fiber or multi-slit
spectrographs. However, only the brightest (${\rm m_R < 25.5}$ LBGs
can have their redshifts confirmed spectroscopically. Most fainter
ones will remain unconfirmed until larger, more sensitive telescopes
are commissioned.

\subsection{The spectral characteristics of Lyman break galaxies}
For the LBGs that astronomers {\it have} been able to obtain spectra
for, not only do we know their (confirmed) redshifts, we can also
measure some of their physical properties. Typically, the spectra of
LBGs show a combination of both nebular emission and absorption lines,
as well as (weak) absorption lines associated with stellar
photospheres. Some also show strong Ly$\alpha$ (at rest-frame 1216\AA )
and C{\sc IV} emission from ionised gas surrounding the hot, young
stars in these galaxies. However, these lines often appear asymmetric
as the photons in the blue wings of the lines are absorbed by Hydrogen
gas in the galaxy (which becomes excited to the $n=2$ level).

The spectra of LBGs appear similar to galaxies in the local Universe
that are undergoing episodes of rapid star formation -- known as {\it
  starburst} galaxies. However, LBGs tend to have spectra that are
metal-poor compared to galaxies in the local Universe, suggesting
large amounts of {\it pristine} gas (i.e., unprocessed by
star-formation) in LBGs. This makes sense when considering that we're
observing LBGs as they were when the Universe was young before lots of
reprocessing in stars had ``contaminated'' gas with metals.

\subsection{The morphological characteristics of Lyman break galaxies}
Once astronomers have identified LBGs using the dropout technique,
they can also study their morphologies (i.e., shapes) in the bands
{\it longward} of the break. Because of their high redshifts, this
typically requires the high spatial resolutions only afforded by the
Hubble Space Telescope (or adaptive optics, in which the distorting
effects of the atmosphere are counteracted using deformable mirrors
placed in your telescope optics.)

Detailed morphological studies of LBGs have reveals that the tend to
be physically smaller than local galaxies of the same luminosity, with
half-light radii similar to the bulges of local spiral galaxies or
small ellipticals (although, as ever, we need to be wary of missing
low surface brightness features). While some LBGs appear relatively
smooth in structure (like today's disk galaxies), many show signs of
being highly clumpy and irregular in nature, which is suggestive of
mergers and/or interactions in the high redshift Universe. You will
recall from Lecture 2 that we need to be mindful of morphological
K-corrections if we see such ``clumpy'' structures in high redshift
galaxies. However, LBGs show little evidence of suffering from this
problem, with both the rest frame optical (probed by the near-infrared
bands) and rest-frame UV (probed by the optical bands) showing similar
morphologies.

\subsection{The star-forming properties of Lyman break galaxies}
By the very nature of how they are identified, LBGs must have a strong
UV continuum. There must be a lot of UV photons just longward 912\AA\
(which is still the UV) in order for a LBG to be seen to drop out of
the shorter waveband. Since only massive, hot, {\it young} stars
produce a strong UV continuum, and since these stars only live for a
short time, then LBGs must have recently undergone a recent bout of
star formation.

While LBGs must be star-forming, actually measuring their star
formation rates (SFRs) can be problematic. This is because their SFRs
are usually measured via their rest-frame UV flux, or sometimes via
their nebular emission (if a spectrum is available), both of which can
be strongly affected by dust obscuration. Prior to correction for
dust, LBGs are measured to have SFRs of a few tens of solar masses per
year. After correcting for the effects of dust, however, we find they
have SFRs of around $100~{\rm M_\odot yr^{-1}}$. This is a very high
SFR by today's standards - the SFR of Milky Way is about
$1~{\rm M_\odot yr^{-1}}$; only a handful of galaxies undergoing
intense starbursts due to major gas-rich mergers in the local Universe
have SFRs even approaching $100~{\rm M_\odot yr^{-1}}$.

Since LBGs are strongly star-forming, we can include them in our
measurements of the star-forming history of the Universe and plot them
on the {\it Madau plot}. Since we have identified lots of LBGs, we
have good statistics for them, meaning the average SFR density that we
derive for them is well-defined and has comparatively small error
bars. When we include (dust corrected) LBG SFRs to the Madau plot, we
find that they agree well with SFR densities derived using other
techniques: the SFR density of the Universe peaked at around $z=1-2$,
and was relatively constant before that time.

\subsection{The masses of Lyman break galaxies}
As well as measuring their SFRs from UV observations, we can also
measure the stellar masses of LBGs using spectral synthesis (recall
spectral synthesis fitting can be performed on photometric data, it
doesn't require spectroscopic observations). When we do this, we find
that the stellar masses of LBGs are typically between
$10^9-10^{11}~M_{\odot}$ (some of the spread is introduced by the
uncertainty in the model fits). Interestingly, however, when we
measure the {\it dynamical} mass of LBGs via the (assumed)
gravitational motions of their stars, we find that they are about a
factor of ten {\it lower} than the total stellar mass measured from
spectral synthesis. Clearly, the gravitational mass can't be lower
than the mass of all the stars (if it were {\it higher} it could
simply be explained in terms of dark matter). This suggests that the
dynamical masses of LBGs are an underestimate, possibly introduced by
their motions having a non-gravitational component (e.g., mergers).

\subsection{Evidence of outflows in LBGs}
When the spectra of LBGs are obtained, it is often observed that their
emission and absorption lines are broadened and/or shifted relative to
their rest-frame wavelengths. Since shifts in emission or absorption
lines are produced by gas moving at high velocities relative to the
galaxy, this is seen as strong evidence of inflowing and outflowing
gas in LBGs. While the inflowing gas is likely feeding the
star-formation, the outflowing gas is thought to be driven by strong
winds, powered either by supernovae or stellar mass loss.

Such winds are thought to be extremely important in shaping galaxies,
not least in the early Universe. By expelling gas from the galaxy,
such outflowing winds are one of the key feedback mechanisms employed
in hydrodynamic and semi-analytic models of galaxy evolution. Indeed,
winds driven by stellar mass loss and/or supernovae are widely thought
to have suppressed the formation of smaller galaxies. This is why the
{\it galaxy} mass function is less steep than the dark matter halo
mass function at low masses (see Lecture 3).

\subsection{The volume density and clustering of LBGs}
Finally, we will consider the distribution of LBGs in space. Since
LBGs are detected in blank field surveys, we can calculate how many
there are per unit volume of the Universe. To do this, we measure the
{\it luminosity function} of LBGs which, if you recall, provides the
number of galaxies (in this case LBGs) per unit volume in a given
luminosity bin.

As with most luminosity functions in astronomy, the LBG luminosity
function is well fit by a Schechter function: roughly a broken power
law with a break at a given luminosity, $L^\ast$. When we measure the
luminosity function of LBGs, we find that the volume density of LBGs
with luminosities close to $L^\ast$ is roughly the same as the volume
density of $L^\ast$ galaxies in the local Universe. This similar
volume density suggests that LBGs may well have evolved to form many
of today's massive galaxies. This is further backed up by observations
that LBGs cluster together in a similar way to today's galaxies.

\section{Where are LBGs today?}
Based on their masses, volume density and clustering properties, there
is reasonable evidence that LBGs have evolved to form many of the
intermediate to high mass galaxies in the local Universe. However, by
now, the stars we see forming in LBGs at $z>3$ will be over 10 billion
years old, so will represent some of the oldest stars in today's
galaxies. Further, the characteristic masses and sizes of LBGs are
similar to those of the cores and central bulges of local elliptical
and spiral galaxies, respectively. However, their morphologies are far
more disturbed by mergers than nearby bulges or ellipticals. As such,
it is thought that what we are seeing when we observe LBGs are the
central bulges of spiral and elliptical galaxies in the process of
being formed.

\section{Learning objectives for Lecture 7}
We've covered a couple of important topics in this lecture. Here are
the learning objectives:
\begin{itemize}
\item Know what is mean by the star-forming history of the Universe.
\item Know what the Madau diagram is, including its shape, how it is
  affected by extinction due to dust, and other challenges associated
  with defining it.
\item Know what a Lyman Break Galaxy (LBG) is.
\item Know how LBGs are identified (i.e., be able to decribe the
  dropout technique).
\item Understand the spectroscopic, morphological, star-forming, mass, and
  clustering properties of LBGs.  
\item Have an idea of the importance of LBGs in the build-up of today's galaxies.
\end{itemize}

\end{document}