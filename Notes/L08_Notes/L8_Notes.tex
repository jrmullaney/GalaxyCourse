   
\documentclass[11pt]{article}
\renewcommand{\baselinestretch}{1.05}
\usepackage{amsmath,amsthm,verbatim,amssymb,amsfonts,amscd, graphicx}
\usepackage{graphics}
\topmargin0.0cm
\headheight0.0cm
\headsep0.0cm
\oddsidemargin0.0cm
\textheight23.0cm
\textwidth16.5cm
\footskip1.0cm

 \begin{document}
 
\title{Lecture 8:\\Studying galaxy evolution in the IR/sub-mm}
\author{Dr. James Mullaney}
\maketitle

\section{Introduction}
In the previous lecture, we covered the use UV wavelengths to measure
the star-forming properties of high redshift galaxies. We also looked
at Lyman Break Galaxies, which are those galaxies that are {\it
  identified} exclusively via a {\it break} in their rest-frame
continuum.\footnote{{\bf Important note}: The presence of a breaks in
  galaxy continua is relied-upon heavily when measuring photometric
  redshifts. A break is a dramatic feature in the {\it shape} of an
  SED, so it helps a lot when ``shifting'' the synthetic spectrum in
  wavelength to find a photometric redshift}. However, as we've
seen extensively already, the UV is heavily affected by absorption
by interstellar dust, which needs to be accounted-for using uncertain
correction factors. In this lecture, we'll consider the use of the
longer infrared (${\rm 1~\rm \mu m - 500~\mu m}$) and sub-mm
(${\rm 500~\rm \mu m - 1~mm}$) wavelengths to measure the star-forming
properties of galaxies. These long wavelengths, especially
$\gtrsim30~\mu m$, are almost impervious to absorption by dust.

\section{What produces infrared emission in galaxies?}
Infrared emission is produced by warm stuff. That's the case whether
it's a (living) human body, coals on a fire, or a galaxy. In the
latter case, the infrared is largely emitted by dust in the galaxy
that is being heated by shorter wavelength, UV photons. These UV
photons hit the dust particles, their energy goes into heating the
dust particle, which then re-radiated the energy at infrared
wavelengths. Eventually, this absorption of UV photons, heating and
re-emission at infrared wavelengths reaches an equilibrium and the
dust becomes a black body (i.e., with a continuum spectrum described
by the Planck equation).

So, to emit in the infrared, a galaxy needs two things: dust, and a
source of UV photons to heat the dust. It's still not entirely clear
where dust comes from, but it's thought that supernovae (and thus
star-formation) play a key role in dust production. There are two main
sources of UV photons in a galaxy: the first and most common are hot,
young stars, the second is emission from the accretion disk of an
AGN. Typically, UV photons from populations of young stars will
``heat'' the dust to about 20-100~K (yes, it's still pretty cold, but
much warmer than the $\sim3$K Cosmic Microwave Background). By
contrast, AGN tend to produce more high energy UV photons, so are
thought to heat the dust to warmer temperatures (although, as Clive
will tell you, there is some contention on this issue).

\section{Early observations of the Universe at infrared wavelengths}
Since most of the infrared part of the spectrum (${\rm 5-500~\mu m}$)
is absorbed by the Earth's atmosphere (largely by water molecules; the
${\rm 1-5~\mu m}$) wavelengths less-so), observations of the Universe
at these wavelengths normally require space-based observatories. The
first astronomically useful space-borne infrared telescope as {\bf
  IRAS}, which was launched in 1983. This conducted the first infrared
survey of the whole sky (at 12, 25, 60 and ${\rm 100~\mu m}$).
Remarkably, it remains the {\it only} all-sky survey at 60 and
${\rm 100~\mu m}$).

During its lifetime, {\bf IRAS} identified around infrared 350,000
sources. Many of these remain unclassified, but it is known that
around 75,000 are extragalactic sources. Due to infrared astronomy
being in its infancy when {\it IRAS} was launched, it suffered from
poor sensitivity by today's standards. As such, it only detected the
brightest infrared galaxies; all but the nearest ones are there
extremely luminous at infrared wavelengths. Such galaxies are known as
either {\it luminous infrared galaxies}, or LIRGs (with infrared
luminosities $L_{\rm IR}=10^{11}-10^{12}~{\rm ergs~s^{-1}}$), and {\it
  ultra luminous infrared galaxies}, or ULIRGS
$L_{\rm IR}=10^{12}-10^{13}~{\rm ergs~s^{-1}}$.\footnote{There are
  also HyperLIRGS, with $L_{\rm IR}>10^{13}~{\rm ergs~s^{-1}}$, but
  these are extremely rare}.

On discovery of these bright infrared galaxies by IRAS, astronomers
observed them in follow-up observations at optical wavelengths. What
they found was that many of these infrared bright galaxies were in the
process of undergoing {\it major mergers} of gas-rich
galaxies.\footnote{ A major merger is usually defined as a galaxy
  merging with another galaxy with at least one third of the mass of
  the first. By contrast, a {\it minor merger} is usually defined as
  when a galaxy merges with another galaxy with a mass {\it less} than
  a third of the first.} In such cases, intense episodes of
star-formation is triggered by the merger process compressing the gas
within the merging galaxies, causing it to collapse to form stars at a
very high rate. A LIRG is typically forming stars at a rate of
10-100~${\rm M_\odot}$ per year, whereas a ULIRG forms stars at a rate
of 100-1000~${\rm M_\odot}$ per year. Because of their very high star
formation rates, such galaxies are known as {\it starburst} galaxies.

Due to its low sensitivity, however, IRAS was not very good for
exploring the high redshift Universe, so provided little insight into
the evolution of galaxies at infrared wavelengths. Instead, this had
to wait until the development of sub-mm astronomy in the mid-90's.

\section{Sub-mm astronomy}
As the name suggests, sub-mm astronomy exploits the wavelengths of
light shortward of 1~mm, just before you enter the infrared regime at
around 500~${\rm \mu m}$. It is the part of the spectrum between the
radio and infrared. One of the greatest benefits of sub-mm astronomy
over infrared astronomy is that the Earth's atmosphere is relatively
transparent at these wavelengths, especially at very high altitudes at
very dry sites.\footnote{There are only a few places around the world
  that are high and dry enough to conduct sub-mm astronomy. Two of
  particular note are the summit of Mauna Kea on Hawai'i, and the
  Atacama Desert in Chile.} As such, they are not as limited in terms
of aperture size as infrared telescopes; indeed, the James Clarke
Maxwell (JCMT) sub-mm telescope on Mauna Kea is 12~m in
diameter. Despite its size, however, it still suffers from poor
spatial resolution due to the long wavelengths it detects (remember,
angular resolution$=1.22\lambda/D$, where $\lambda$ is wavelength and
$D$ is aperture size). For example, the angular resolution of JCMT is
roughly 15 arcsec compared to $\sim$1 arcsec of ground-based optical
telescopes.

\subsection{The positive $K$-correction of sub-mm observations}

Another great benefit of sub-mm astronomy derives not from practical
considerations, but the very shape of the infrared/sub-mm SED of
galaxies. Recall from Lecture 2 the $K$-correction that needs to be
applied to observations of high redshift galaxies due to the shifting
of shorter wavelengths into the observed band. At optical wavelengths,
this $K$-correction is said to be negative because it works against
us: galaxies tend to be {\it less} luminous at shorter optical
wavelengths, so when these fainter, shorter wavelengths are redshifted
into our observing bands they are more difficult to detect. By
contrast, at infrared and sub-mm wavelengths (longward of about
100~${\rm \mu m}$), galaxies tend to be {\it more} luminous at shorter
wavelengths. As such, when these {\it brighter} shorter infrared
wavelengths are shifted into our sub-mm observing bands, they are {\it
  more easily} detected. This {\it positive} $K$-correction almost
exactly counteracts the $1/r^2$ cosmological dimming, meaning that a
$z\sim1$ galaxy is as easy to detect at $850~\rm {\mu m}$ as a
$z\sim3,4,5$ galaxy! Not only that, but it can actually be {\it easier} to
detect distant galaxies in the sub-mm than ones in the local Universe. 

\subsection{High redshift astronomy at sub-mm wavelengths}
Because of the positive $K$-correction, sub-mm wavelengths are ideally
suited to studying high redshift -- and therefore distant, early --
galaxies. However, early (1990s to the mid-2000s) sub-mm studies of
high redshift galaxies were severely hampered by the poor resolution of
the telescopes. When the sub-mm point spread function (PSF) is around
15~arcsec in diameter, it is extremely difficult to pinpoint which
galaxy seen at optical wavelengths it corresponds to. For example, in
the deep fields -- where the first sub-mm surveys were conducted --
Hubble could detect ten or more galaxies within a circle of diameter
15~arcsec. Which (if any) of those Hubble galaxies actually correspond
to you sub-mm detection? Indeed, there could be more than one
contributing to the total sub-mm flux of your sub-mm source. This
causes significant problems for measuring the redshifts of your sub-mm
galaxies. If you don't know which galaxy it is, you can't even use
photometric redshifts, let alone target them for spectroscopic
follow-up. It is possible, however, use the overall {\it shape} of the
sub-mm SED to get a very crude redshift (often by just using a flux
ratio to define the shape), but this only gives a redshift that's
accurate to within $\Delta z \sim 1$, corresponding to roughly half
the age of the Universe!

\subsection{Using sub-mm wavelengths to measure SFRs}
Under normal circumstances, the large uncertainties in the redshifts
of sub-mm galaxies would scupper our chances of measuring the
(rest-frame infrared) luminosities (see Lecture 4 for how we use
infrared wavelengths to measure SFRs). And if we don't know their
luminosities, we can't convert these into meaningful physical
parameters such as star formation rates. Again, however, the positive
$K$-correction saves us. Because the positive $K$-correction
counteracts the $1/r^2$ geometric dimming, it means that a galaxy of a
given luminosity {\it will have roughly the same sub-mm flux},
irrespective of its redshift (to within reasonable limits, and
certainly to within $\Delta z \sim 1$). As such, as long as we have
{\it some idea} of the redshift (from sub-mm flux ratios, for example)
and a measured flux, we can convert it into a fairly confident
infrared luminosity and, therefore, SFR (yep, it feels like cheating,
but it actually works!).

With SFRs in hand, astronomers could work out the SFR density (i.e.,
the SFR per unit volume in the Universe) arising from sub-mm galaxies
(SMGs) and plot this on the Madau plot (which we saw in L7). Although
only based on a small number of sub-mm detected galaxies detected in
the deep fields (remember, this was during the 1990's when sub-mm
astronomy was still in its infancy) -- and thus subjected to
significant uncertainties -- it appeared that SMGs were
responsible for roughly the same contribution to the SFR density
optical/UV-detected galaxies. Thus, sub-mm observations roughly
doubled our estimates of the amount of star formation (per unit
volume) taking place in the high redshift Universe.

Most importantly, sub-mm observations showed us that a significant
proportion of star-formation activity in the Universe had been {\it
  completely missed} by optical/UV observations due to absorption by
to dust. That's the equivalent of measuring global birth rates, then
discovering India and China!

\subsection{Pin-pointing the positions and redshifts of SMGs}
As we have seen, during the mid-90's it became increasingly clear
that SMGs were an important component of the star-forming history of
the Universe. As a consequence, it became increasingly important that
astronomers were able to pinpoint their positions more accurately, not
least to obtain accurate redshifts.

Since all sub-mm galaxies are strongly star-forming, it was realised
that observations at other wavelengths that are also sensitive to
star-formation could also help to pinpoint them, provided they were
higher resolution. However, as we've seen, we can't use the
optical/UV, as it is blocked by dust. The (partial) solution came in
the form of radio observations. This is because when stars undergo
supernovae, they release a lot of radio emission due to synchrotron
emission in the supernova ejecta. If a galaxy is strongly
star-forming, it have a high rate of supernovae, and thus a strong
radio luminosity. The benefit of radio wavelengths over UV/optical is
that they are {\it completely impervious} to dust (if a radio wave can
get through the walls of your house, it can easily penetrate
interstellar dust). The other great benefit of radio wavelengths is
that we can reach very high angular resolutions via radio
interferometers (in fact, very long baseline radio interferometers
provide the highest angular resolutions of {\it any} telescope).

By targeting sub-mm galaxies with radio interferometers, astronomers
were able to pinpoint the positions of some of them to within about 1
arcsecond. This is easily accurate enough for follow-up optical
spectroscopy, from which spectroscopic redshifts could be
obtained. Unfortunately, not all sub-mm galaxies are detectable at
radio wavelengths due to the relative sensitivities of our telescopes
at different wavelengths (recall the biases introduced by
multiwavelength surveys discussed in Lecture 6). Similarly, even if we
have a radio position, that is no guarantee that our optical
telescopes will be able to obtain a redshift since, by their nature,
sub-mm galaxies are heavily obscured at optical wavelengths. As such,
by 2005, only around 50\% of the 150 brightest sub-mm galaxies had
secure spectroscopic redshifts.

\subsection{SMGs in context}

Acknowledging the above problems, with the redshifts that we {\it did}
have, it was clear that sub-mm galaxies were, indeed, more common at
high redshifts. Their numbers peak at around redshift $z\sim2.4$. With
precise redshifts, we could finally obtain precise infrared
luminosities and, thus SFRs, which revealed that sub-mm galaxies
typically had SFRs over ${\rm 100~M_\odot}$ per year. It was also
confirmed that the SFRs of sub-mm galaxies are at least 100 times {\it
  higher} than what would be estimated via the rest-frame UV (when
they are even detected at these short wavelengths and after trying to
correct for dust obscuration), meaning that most of the star formation
is hidden by vast columns of dust.

Sub-mm galaxies therefore have SFRs that are roughly similar to ULIRGs
today. However, ULIRGs in the local Universe are extremely rare -
roughly one per ten million Mpc$^3$. By contrast, sub-mm galaxies are
roughly a thousand times more common at $z\sim2.5$ than ULIRGs are
today. This correxponds to a much higher SFR density, as seen when the
contribution from sub-mm galaxies are included on the Madau plot.

It is interesting to compare SMGs against the other main type of
star-forming galaxies at high redshift: Lyman break galaxies
(LBGs). After correcting for dust extinction, LBGs and sub-mm galaxies
contribute roughly equal amounts to the total SFR density of the
Universe at $z\sim2-2.5$. However, unlike LBGs, whose contribution
continues to remain significant to higher redshifts ($z\sim5$), the
contribution from sub-mm galaxies drops away at higher redshifts. As
such, the sub-mm ``phase'' of the Universe is, by comparison, fairly
short lived, lasting from $z\sim3$ to $z\sim1$, or roughly 4 billion
years.

Finally, since the density of SMGs peak at $z\sim2.4$, and with
today's most massive galaxies already in place by $z\sim3$, it is
unlikely that we are witnessing in SMGs the build-up of today's most
massive galaxies. That role seems to be being played-out by LBGs at
higher redshifts. Indeed, it's still not entirely clear what SMGs
evolved to become in the current Universe.

\section{Spitzer's contribution to infrared astronomy}
As we have seen, for many years after IRAS completed its all-sky
survey, infrared astronomy was dominated by observations at sub-mm
wavelengths (and that's not a contradiction: sub-mm observations of
high redshift galaxies samples the rest-frame infrared). This changed
in 2003 with the launch of {\it Spitzer}, which was significantly more
sensitive (by roughly a factor of 1000) at $3.6 - 160~{\rm \mu m}$
than any telescope that had gone before. Unlike IRAS, however, it did
not perform an all-sky survey. However, it was the first infrared
telescope to conduct blank-field surveys comparable in depth to those
provided by Hubble in previous years.

Due to its excellent sensitivity at infrared wavelengths, Spitzer was
capable of detecting tens of thousands of {\it high redshift}
galaxies in the deep field, {\it despite} the negative
$K$-correction.\footnote{It should be noted that, unlike sub-mm
  telescopes, Spitzer sampled the side of the dust's black body
  spectrum {\it shortward} of the peak and, as such, was subject to
  negative $K$-correction.} Most of the sources that Spitzer detected
in the deep fields were either LIRGs (at redshifts between 0.8 and 1.5)
or ULIRGs (at $z>1.5$).

The greatest benefit that Spitzer has over previous infrared/sub-mm
observatories was, however, its angular resolution, allowing its tens
of thousands of infrared-detected galaxies to be pinpointed to with a
few arcseconds. This allowed the Spitzer-detected galaxies to be
matched to galaxies detected at other wavelengths in the deep fields
(recall, deep survey fields tend to be observed in many different
wavelengths) which, correspondingly, enabled accurate photometric
redshifts to be determined. For the first time, therefore, astronomers
were able to determine accurate infrared luminosities -- and thus SFRs
-- for {\it tens of thousands} of galaxies across large redshift
ranges. 

\subsection{The evolving infrared luminosity function}
With the availability of accurate luminosities and redshifts for many,
many galaxies, astronomers used Spitzer to determine -- for the first
time -- how the infrared luminosity function has evolved over the
history of the Universe. This is a really important result: since the
infrared traces star-formation, what they were really plotting is how
the histogram (recall, a luminosity function is just a histogram) of
galaxy star formation rates have evolved over the history of the
Universe.

In a result that echoed that of sub-mm observations, what astronomers
found using Spitzer was that the number density (i.e., number per unit
volume) of LIRGs and ULIRGs in the early Universe was {\it orders of
  magnitude} higher than in the current Universe. So, the peak in the
Madau diagram wasn't caused by there being a higher density of
galaxies of roughly the same SFR as today, but by today's galaxies
growing much more quickly in the early Universe than they are
today. It's as though most of today's galaxies went through a major
growth spurt around 10 billion years ago.

\subsection{Caveats with infrared astronomy}
Hopefully by now you will have a sense of how important infrared and
sub-mm astronomy is for our understanding of galaxy evolution. Before
summarising this lecture, however, I would like to raise some caveats
associated with infrared astronomy.

Firstly, recall all the problems associated with sub-mm
observations. All these lead to significant selection biases that
lead to large (systematic) uncertainties on our results. Despite
Spitzer's high angular resolution solving some of these problems, it
is still the case that many of the redshifts associated with infrared
sources are photometric, rather than spectroscopic, and thus come with
large uncertainties.

Secondly, throughout all of this Lecture, I've breezily mentioned that
we can calculate infrared luminosities from observations sometimes at
single infrared wavelengths (e.g., a single sub-mm or Spitzer
band). However, implicit to this are a number of assumptions, in
particular the shape of the SED and the conversion from infrared
luminosity to SFRs. There is a whole host of poorly understood physics
which determine the shape of the infrared SED and thus the conversion
between a single band observation to an infrared luminosity. And there
are less well-understood physics in converting infrared luminosities
to star-formation rates. As such, there are a number of (often
skipped-over) systematic uncertainties in measuring SFRs from infrared
observations (although it's probably no worse than any other wavelength!).

\section{Learning objectives for Lecture 8}
In this lecture we covered the study of galaxy evolution at infrared
and sub-mm wavelengths. Because these wavelengths probe the SFRs of
galaxies, they provide a direct insight into how galaxies have grown
over the history of the Universe. Here are the learning objectives:
\begin{itemize}
\item LBG, SMG and infrared-selected galaxies each contribute similar
  amounts of the global star formation density at z~2.
\item For SMG and infrared-selected galaxies <10\% of the star
  formation is directly visible at UV wavelengths.
\item The SMG and infrared-selected galaxies together contribute
  $>50\%$ of the star formation density at high redshifts (based on
  the far-IR luminosities).
\end{itemize}


\end{document}