   
\documentclass[11pt]{article}
\renewcommand{\baselinestretch}{1.05}
\usepackage{amsmath,amsthm,verbatim,amssymb,amsfonts,amscd, graphicx}
\usepackage{graphics}
\topmargin0.0cm
\headheight0.0cm
\headsep0.0cm
\oddsidemargin0.0cm
\textheight23.0cm
\textwidth16.5cm
\footskip1.0cm

 \begin{document}
 
\title{Lecture 12:\\AGNs and supermassive black holes}
\author{Dr. James Mullaney}
\maketitle

\section{Introduction}
Last lecture, we took our first look at Active Galactic Nuclei
(AGNs). We learned how they were discovered, the different types of
AGNs and how they are ``unified'', and how different parts of the
electromagnetic spectrum are used to distinguish AGNs from ``normal''
galaxies. In this lecture, we'll consider where all the energy in an
AGN comes from by taking a closer look at black hole and accretion disk
system that forms the so-called ``central engine''.

\section{Central engine characteristics}
As we saw in the previous lecture, the first AGNs to be studied in any
kind of detail were moderate luminosity ``Seyfert-like'' AGNs. While
these first studies highlighted the curious observed properties of the
nuclei of these galaxies, they didn't make any real attempt to explain
their cause. This all changed when the distances to the first known
quasars were measured and the consequent appreciation of their extreme
luminosities. With the brightest known quasars having
luminosities equivalent to 10,000 Milky Way-like galaxies, you can see
why astronomers quickly began to question what could be the generating
so much power. 

Clearly, any theory to explain what powers a quasar must be able
account for their extreme luminosities. However, it was soon realised
that the power was not the only peculiar aspect of AGNs that any model
of energy production must be able to explain. Take their optical
spectra, for example. Type 1 AGNs in particular display spectra that
are unlike any other type of object; no other celestial body displays
emission lines with widths corresponding to thousands or even tens of
thousands of ${\rm km~s^{-1}}$. What was causing such extreme
velocities? In addition, any theory of energy production must be able
to explain the very large radio luminosities of AGNs. Indeed, as the
angular resolution of radio telescopes improved, it was discovered that
the radio emission of some AGNs is being emitted by powerful radio
jets that extend over many tens or even hundreds of kpc. Any model of
the central engine must be able to explain what is launching such
powerful jets (which, recall, can be more luminous at radio
wavelengths than the {\it bolometric} output of all the stars in the
host galaxy).

An interesting and important consequence of the discovery of {\it
  extended} radio jets was that it revealed that some AGNs had been
generating power for long enough to launch jets that are millions of
light-years long. Clearly this implies that some AGNs had been
generating vast amounts of energy for millions of years, and were thus
not the type of short ``explosive'' events normally associated with
more familiar extreme phenomena such as supernovae. Clearly, any
theory of energy production must therefore be able to explain how this
energy output can be maintained for so long.

\subsection{AGN variability}
Of all the extreme properties of AGNs, perhaps the most puzzling --
yet ultimately revealing -- feature for astronomers was their rapid
variability. Soon after the first discovery of quasars it was realised
that their optical flux varied on timescales of a few days to
years. While they always stayed extremely luminous, their brightness
would change be a factor of a few within a relatively short period of
time.

The reason why the rapid variability of quasars is so interesting is
because how rapidly something varies reveals something about its
size. For example, it would be impossible for anything as large as a
whole galaxy to vary in brightness in the space of a year. This is
because it takes time for the ``information'' regarding that change to
be transmitted from one side of the galaxy to the other. The fastest
any information can travel is the speed of light, so the fastest
something as large as a galaxy can vary in brightest would be a few
tens of thousands of years. So, if a quasar changes brightness over a
period of just a few days, then it implies that its {\it maximum} size
is a few light-days across. This means that the central engine of
some AGNs is, at maximum, not much bigger than our solar system -- an
absolute minuscule size in galaxy terms. So, early AGN astronomers
were faced with the incredible concept that more power than that
emitted by all of the stars in the Milky Way is coming from a region
of space not much bigger than our solar system!

\section{Suggested AGN energy sources}
There's an old saying that goes something like: ``if the only tool you
have is a hammer, you'll treat every problem as if it's a nail''. Up
until the middle of the 20th century, the only thing astronomers had
ever known was stars and planets (galaxies are, after all, just
collections of lots and lots of stars), so it was tempting to try to
solve the ``problem'' of the AGN power source by appealing to
stars. However, as we shall see, all of these theories had extreme
shortcomings.

\subsection{A $10^6$ solar mass star}
Among the first suggestions for what could be powering bright quasars
was that it was a single extremely massive star. As we have seen, the
most massive stars in a galaxy are very hot and are disproportionately
bright for their mass (i.e., a star ten times more massive than the Sun is
more than ten times more luminous). Perhaps inspired by the
stellar-like appearance of quasars, some early AGN astronomers
suggested that AGNs were, in fact, a single star with a mass of over a
million solar masses. Such a star would, indeed, be extremely bright
and would also emit strongly at ultraviolet wavelengths and could
therefore explain the blue continua of AGNs. A very massive star would
also satisfy the compactness requirements, and its strong gravity
could be used to explain the broad emission lines through
gravitational doppler broadening.

Despite accounting for some AGN properties, the massive star theory
falls down on many key points. Firstly, such a massive star would be
extremely short lived, possibly only lasting a few tens of thousands
of years -- not long enough to maintain the extended radio jets seen
in some AGNs. Speaking of which, there is no known star that produces
such collimated radio jets; how would a massive star do so? Further,
stellar masses are known to form a continuum distribution (i.e., the
initial mass function), so if there are million solar mass stars
powering AGNs, where are all the slightly smaller, hundred thousand
solar mass stars?

\subsection{A massive star cluster}
Another star-based early suggested model for the central engines of AGNs was a
very dense central stellar cluster consisting of thousands or even
millions of massive (i.e., $>20~{\rm M_\odot}$) stars. Being massive,
these stars would quickly undergo supernovae one after another, which
would explain both the high luminosity and variability of AGNs. Also,
the combined radio emission of the supernovae remnants could explain
the high radio luminosities of AGNs.

While it would be technically possible for this number of massive
stars to be packed into a volume the size of the solar system, this
model has a number or drawbacks. Firstly, it isn't clear how such a
setup would create the collimated jets we see being launched from
AGN. Most importantly, however, is that the spectral features of AGNs
simply do not look like those of supernovae remnants

\subsection{A central supermassive black hole}
With all the attempts to use stars to explain the properties of AGNs
looking extremely contrived (and yet still failing to explain key
observed AGN features), a few theorists started to consider other
non-stellar possibilities. As we now know, the most successful of
these was the idea that a supermassive black hole is, ultimately,
responsible for the AGN phenomena. With black holes the densest form
of matter in the Universe, they are ideal candidates to meet the
``compactness'' requirement. However, since black holes by definition
emit no light, how can they account for the most fundamental property
of AGNs: their extremely high luminosities?

As we mentioned in the previous lecture, the key to an AGN's
luminosity is not the black hole itself, but the accretion disk that
forms around the black hole as it consumes surrounding gas and
dust. As the material falls toward the black hole, it releases its
gravitational potential energy in the form of light. In fact, this
release of gravitational energy as material falls onto a supermassive
black hole is an incredibly efficient process, with roughly 10\% of
the rest mass of the falling material being converted into light. This
compares to $<1\%$ efficiency for nuclear fusion in stars. Indeed, the
release of energy due to gravitational collapse is the second only to
matter/anti-matter annihilation in terms of its efficiency in
converting rest mass into radiated energy. As such, even a relatively
modest accretion rate of about one solar mass per year is sufficient
to power a quasar of bolometric luminosity
$\sim 5\times10^{45}~{\rm erg~s^{-1}}$.

So, an accreting supermassive black hole can explain both the power
output and variability of AGNs (the latter because of its highly
compact nature). But what about those radio jets seen in a subset of
AGNs?  Can an accreting black hole explain those as well? Well, one
popular theory is that as the accreting material gets hotter and
hotter, it starts to become ionised. As such, it forms a fast-moving
plasma, which generates a strong magnetic field (as all moving charges
do). It is thought that if the black hole is spinning, it will tangle
the magnetic field lines up into a spiral pattern. Any electrons and
plasma caught within this magnetic field will be accelerated to close
to the speed of light, forming tightly collimated jets.

Finally, as we saw in the previous lecture, a supermassive black hole
plus accretion disk system can explain the very broad emission lines
seen in the spectra of Type 1 AGNs. As a reminder, these broad lines
are being emitted by dense, ionised gas orbiting close to the black
hole, which explains their extremely high velocities.

\section{Finding evidence for supermassive black holes}
While it seems that accreting supermassive black holes can explain
very well many of the observed properties of AGNs, this does not
constitute a proof that it is correct. For that, we would ideally
identify evidence of supermassive black holes in the centres of
galaxies. Before doing so, however, it would be useful to have some
kind of indication of the expected masses of a supermassive black
hole, just so we've got some idea of how challenging measuring them
will be.

\subsection{The Eddington Limit}
One way of obtaining a rough lower limit for the mass of an accreting
supermassive black hole is by assuming that they are accreting close
to their Eddington luminosity. You should recall from previous lecture
courses that the Eddington luminosity for an accreting object (whether
a star or a black hole) is reached when the gravitational force
pulling the material inward is balanced by the radiative force from
the photons emitted by that accreted material. Equating these two
forces gives:
\begin{equation}
L_{\rm Edd}=\frac{4\pi G M m_p c}{\sigma_T}
\end{equation}
where $G$ is the gravitational constant, $M$ is the mass of the
central object (in our case, a supermassive black hole), $m_p$ is the
proton mass, and $\sigma_T$ is the Thompson cross-scattering
area. Entering the numerical values for the constants, and converting
the units gives:
\begin{equation}
L_{\rm Edd}=1.3\times10^{38}~\left(\frac{M}{\rm M_\odot}\right){\rm erg~s^{-1}}
\end{equation}

We have already seen that the brightest quasars accrete at a rate of
roughly one solar mass per year, producing a bolometric luminosity of
about $5\times10^{45}~{\rm erg~s^{-1}}$. Plugging this luminosity into
the above equation gives a black hole mass of
$5\times10^{45}/1.3\times10^{38}\approx4\times10^7~{\rm
  M_\odot}$.
However, this assumes the black hole is accreting at its Eddington
luminosity, whereas it is likely to be accreting at a rate somewhat
lower than this. As such, this gives a {\it lower limit} to the mass
of the black hole. Clearly, this is much more massive than a
conventional, stellar mass black hole (which may reach a few tens of
solar masses at most).

\subsection{The Soltan Argument}
Another way of getting an estimate of the mass of all supermassive
black holes in today's Universe is to (a) sum up all the material that
has ever fallen into a supermassive black hole and (b) assume that
this mass is distributed evenly among all massive galaxies. This
clearly assumes that all massive galaxies contain a central
supermassive black hole, and thus again represents a lower mass
limit. This methodology was first proposed by Soltan in 1982, and so it
referred to as the Soltan Argument.

Part (a) sounds incredibly difficult: How can we possibly measure all
of the mass that's ever fallen into a supermassive black hole? In
practice, however, it's relatively straightforward (given some
reasonable assumptions and some careful measurements). Black holes
grow by accreting material, and as we have seen, the amount of light
they emit is directly proportional to their accretion rate (assuming a
given accretion efficiency). This means that the AGN luminosity
function (i.e., the histogram of AGN luminosities per unit volume)
provides a measure of the accretion rate density of the Universe. The
nice thing about black holes, is that once matter has fallen into the
black hole, it can't get back out.\footnote{At least, it's nice thing
  for this calculation, less so if you're the material falling in.} As
a consequence, integrating all these accretion rates over time gives
the total amount of mass that accretes into a black hole in a given
time interval. All we need to do, therefore, is measure the evolving
AGN luminosity function and integrate it with respect to time to give
total amount of mass (per unit volume) contained within today's black
holes. Thankfully, since AGNs are bright, we are able to measure their
luminosity function across large swathes of cosmic time and perform
this calculation.

When astronomers integrate the evolving AGN luminosity function over
cosmic time, they estimate that they get for the total amount of mass
contained within black holes per cubic Mpc is about
$5\times10^5~{\rm M_\odot}$.  Dividing this by the number density of
massive galaxies in today's Universe (i.e., about 0.01 galaxies per
${\rm Mpc^3}$ gives an average black hole mass of about
$5\times10^7~{\rm M_\odot}$ per massive galaxy. Again, this is far
more massive than a typical stellar-mass black hole.

\subsection{Weighing supermassive black holes}
The above arguments make a strong case for the black holes at the
hearts of AGNs to be supermassive (i.e., far more massive than normal,
stellar mass black holes). They are, however, still just
predictions. It would be nice to be able to actually measure the mass
of these black holes directly. To do this we need to be able to
measure the motion of ``test particles'' moving under the influence
of the black hole's gravity.

\subsubsection{Stellar orbits near the BH}
One way to achieve this is by measuring the velocity of stars
very close the black hole. However, the density of stars in the
central regions of a galaxy is so high that the gravitational force of
the central supermassive black hole only dominates (over the stars in
the galaxy) within about 30~pc from the black hole. It took until the
launch of the Hubble Space Telescope before these small scales could
be anywhere near resolved in nearby galaxies.

Using the Hubble Space Telescope, astronomers in the mid-1990s
measured the velocities of stars as a function of distance from the
centres of two nearby galaxies: M84 and M87. What they found was that
the stellar velocities showed a {\it sudden} turnover (i.e., from
redshifted to blueshifted) within just a few parsecs of the central
region. This demonstrates that whatever is governing the orbits of
these stars must be incredibly compact (otherwise it wouldn't be such
a sudden turnover). Indeed, based on these measurements, it was
estimated that the central few parsecs of M87 contained a ``dark''
mass of ${\rm 3.2\times10^9~M_\odot}$ (i.e., it did not ``shine'' as
much as you'd expect from a billion stars).

Later, in the early 2000s, aided with adaptive optics on ground-based
near-infrared telescopes, astronomers measured the motions of stars
around the central point of the Milky Way (known as Sagittarius
A$^\ast$ and pinpointed via long baseline radio interferometry). In
doing so, they discovered that the stars close to Sagittarius A$^\ast$
are orbiting a very massive, compact object (i.e.,
$4.3\times10^6~{\rm M_\odot}$ that emits virtually no optical
light. This object is so compact and massive that the only viable
explanation is that it is a supermassive black hole at the centre of
the Milky Way.

\subsubsection{Reverberation mapping}
The other main means of measuring the mass of a supermassive black
hole involves using one of the main observable features of Type 1 AGNs
as a test particle -- the broad line region (BLR).

Under the assumption that the gas in the BLR moves under the influence
of the black hole's gravity, we can use simple Newtonian dynamics to
derive the black hole's mass:
\begin{equation}
M = \frac{v^2r}{G}
\end{equation}
where $M$ is the mass of the black hole, $G$ is the gravitational
constant, $v$ is the velocity of the clouds in the BLR and $r$ is the
radius of the BLR (i.e., equating $F=GMm/r^2$ and $F=mv^2/r$). We can
easily measure $v$ from the optical spectra of Type 1 AGNs, since the
width of the emission lines gives a direct measurement of the circular
velocity of the clouds via their doppler broadening.

By far the more difficult measurement to make, however, is the radius,
$r$, of the BLR, since it is far too small to be resolved in even the
most nearby AGNs. For this, astronomers exploit the variability of
AGNs. When an AGN varies in brightness, this change propagates out
from the central engine at the speed of light. After a certain amount
of time (corresponding to the light travel time between the BLR and
the central engine), we see the BLR respond to the change in
luminosity of the nucleus (it is said to {\it reverberate} in
luminosity in response to the change in luminosity of the central
engine). By monitoring the luminosity of the central engine and BLR,
and measuring how long it takes for the BLR to respond to a change in
luminosity of the central engine, astronomers can determine the radius
of the BLR. Typically it takes a BLR roughly a day or two to respond
to a change in luminosity of the central engine, meaning that radius,
$r$, of the BLR is roughly a couple of light-days. With a measurement
of $r$, it is comparatively easy for astronomers to now calculate $M$.

Of course, since BLRs are only seen in Type 1 AGNs, reverberation
mapping can only be used to measure the masses of black holes in Type
1 AGN (i.e., not non-AGN galaxies).

\subsection{The Black Hole -- Bulge relationship}
Today, astronomers have measured the masses of hundreds of central
supermassive black hole by one way or another. One remarkable outcome
of this has been the realisation that the masses of the central
supermassive black holes are remarkably well-correlated with the
stellar mass of the bulge or spheroid in which they reside. This
correlation spans over four orders of magnitude in both black hole
mass and bulge mass.

The reason why this quite so remarkable is that, as we have seen, the
gravitational influence of a supermassive black hole on the stars in
its host bulge is extremely small. The black hole's gravity only
dominates over that of the stars in the bulge for about 30pc, whereas
a typical bulge will be many kiloparsecs across. Indeed, in terms of
gravity, only about 1\% of all the stars in a bulge ``know'' about
the central supermassive black hole. A such, if it were based on
purely gravity alone, we shouldn't expect there to be any relationship
between the black hole and bulge, and yet we do see a tight
correlation.

As we shall see later in the course, this tight correlation between
the black hole mass and key properties of its host bulge has led many
extragalactic astronomers to think there must be some interaction
between black holes and their galaxies beyond gravity alone. This
interaction is the ``AGN feedback'' that was referred-to in our
discussions of the theory of galaxy evolution. We will cover this
important aspect of galaxy evolution over the remaining few lectures.

\section{Lecture 12 learning objectives}
In this lecture we continued to look at the phenomena of AGNs,
focussing mainly on the black hole and accretion disk that form the
so-called ``central engine''. Here are the main learning objectives
from this lecture:
\begin{itemize}
\item Have knowledge of the general properties of AGN: high nuclear
  luminosities; compact energy generation regions; long lifetimes;
  jets, peculiar spectra containing broad emission lines
\item Understanding of why accretion of material by black holes is the
  most plausible energy generation mechanism for AGN
\item Knowledge of the evidence for supermassive black holes in the
  nuclei of nearby galaxies (i.e., material on M84, M87, and the Milky
  Way) and AGNs.
\item Awareness of the black hole -- bulge relationship. 
\end{itemize}

\end{document}