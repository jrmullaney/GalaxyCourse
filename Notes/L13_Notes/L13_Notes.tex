   
\documentclass[11pt]{article}
\renewcommand{\baselinestretch}{1.05}
\usepackage{amsmath,amsthm,verbatim,amssymb,amsfonts,amscd, graphicx}
\usepackage{graphics}
\topmargin0.0cm
\headheight0.0cm
\headsep0.0cm
\oddsidemargin0.0cm
\textheight23.0cm
\textwidth16.5cm
\footskip1.0cm

 \begin{document}
 
\title{Lecture 13: Black hole growth and formation}
\author{Dr. James Mullaney}
\maketitle

\section{Introduction}
Last week we started looking at Active Galactic Nuclei (AGN) -- sites of
rapid supermassive black hole growth at the centres of galaxies. In
those lectures, we just assumed that supermassive black holes are a
given, whereas, in fact, how these huge black holes were formed
remains a major unsolved question in extragalactic research. In this
lecture, we will consider when supermassive black holes were formed,
and consider current theories of how they were formed.

\section{The earliest known supermassive black holes}
There is no known way we can measure the ``age'' a black hole in the local
Universe, so the only way we can hope to estimate when the first
supermassive black holes were formed is by observing them at high
redshifts. Since one of the key ingredients needed to produce an AGN
is a supermassive black hole, then finding a AGNs at high redshifts is
a clear indication that supermassive black holes existed at early
times. To date, the most distant AGN known is
ULAS~J112001.48$+$064124.3 (hereafter, ULAS): a quasar at redshift
$z=7.088$. This redshift corresponds to a time only 744~Myr after the
Big Bang, meaning the supermassive black hole must have formed soon
after the beginning of the Universe.

Although simply identifying an AGN at such high redshifts provides key
insights into the formation of supermassive black holes, it would be
even better if we could determine the mass of the black hole. With
different formation scenarios likely producing different initial
masses of black holes, measuring the masses of high redshift black
holes helps us to constrain these models. Thankfully, measuring the
mass of the black hole at the centre of an AGN is {\it relatively}
straightforward, even for an AGN at $z\sim7$.
 
Firstly, we can get a crude estimation of a lower limit of the mass of
the black hole via Eddington luminosity arguments. As we saw in the
last lecture, the Eddington luminosity of a black hole is given by:
\begin{equation}
L_{\rm Edd} = 1.3\times10^{38}\left(\frac{M}{\rm M_\odot}\right)~{\rm erg~s^{-1}}
\end{equation}
which roughly corresponds to the maximum luminosity an AGN can have
(it's only ``roughly'' because the Eddington luminosity is defined for
spherical accretion, whereas an AGN accretes matter in the form of a
disk). This gives:
\begin{equation}
L_{\rm AGN} \lesssim 1.3\times10^{38}\left(\frac{M_{\rm BH}}{\rm M_\odot}\right)~{\rm erg~s^{-1}}
\end{equation}
where $L_{\rm AGN}$ is the bolometric luminosity of the AGN, and
$M_{\rm BH}$ is the mass of the black hole. After rearranging, this
gives:
\begin{equation}
M_{\rm BH} \gtrsim \frac{L_{\rm AGN}}{1.3\times10^{38}~{\rm
    ergs~s^{-1}}}~{\rm M_\odot}
\end{equation}
Since we know the redshift to the quasar, it is easy to calculate its
luminosity, which is measured as $2\times10^{47}~{\rm
  erg~s^{-1}}$. Plugging this into the above formula gives a black
hole mass of: $M_{\rm BH}\gtrsim1.5\times10^9~{\rm M_\odot}$. So, even
by this crude approximation, it is clear that we're observing a very
massive supermassive black hole just a few hundred million years after
the Big Bang.

\subsection{A more precise mass estimate}
While the Eddington luminosity method provides us with crude lower
limit for the mass of the black hole, it would be preferable to have
an actual mass measurement. Thankfully, this is possible via the
``virial technique'' which, like reverberation mapping (see Lecture
12), uses the broad line region (BLR) to provide a mass estimate.

Assuming that the motions of gas within the BLR is dominated by the
gravity of the black hole, we can use Newtonian dynamics to measure
the black hole's mass:
\begin{equation}
M_{\rm BH}=\frac{v^2r}{G}
\end{equation}
where $v$ is the velocity of the gas in the BLR, $r$ is the radius of
the BLR, and $G$ is the gravitational constant. As we saw in
reverberation mapping, measuring $v$ is straightforward, we simply
take the quasar's spectrum and measure the velocity widths of the
broad emission lines (in this case, the permitted line C~{\sc iv},
since H$\alpha$ and H$\beta$ are shifted out of the optical
range). Like in reverberation mapping, however, measuring $r$ is more
complicated. It is observationally expensive to repeatedly take the
spectrum of a high redshift quasar, so as yet we have not obtained
the multi-epoch observations required to perform reverberation
mapping for ULAS.

There is, however, another way to calculate $r$ that uses the results
of reverberation mapping of AGNs in the local Universe. As more and
more nearby AGNs had their black hole masses measured by reverberation
mapping, it was soon realised that the radius of the BLR is tightly
correlated with the luminosity of the AGN. It is thought that this is
because the increased light from high luminosity AGNs ``pushes out''
the regions in which the broad lines are produced. What this means is
that we can obtain an estimate of $r$ from the luminosity of the AGN,
i.e.,:
\begin{equation}
{\rm log}(r/{\rm light~days}) = -21.3 + 0.519{\rm log}(L_{\rm AGN}/{\rm erg~s^{-1}})
\end{equation}
From this, we get $r=3.2$~light~days, or $8.3\times10^{13}~{\rm m}$.
This gives a value for the mass of the black hole at the centre of
ULAS as $2\times10^9~{\rm M_\odot}$. This is in the ball-park of the
Eddington luminosity estimate, indicating that ULAS is accreting at
close to its Eddington limit.

\section{The implications of a $10^9~{\rm M_\odot}$ BH at $z=7$.}
Now that we have confirmed the presence of a {\it billion} solar mass
black hole just a few hundred years after the Big Bang, what does this
imply for our understanding of how supermassive black holes form? Does
this imply the presence of ``primordial'' supermassive black holes
formed by the Big Bang, or could these black holes have formed from
``stellar mass'' black holes accreting rapidly for the previous 744
million years?

To answer this question, we have to consider whether a black hole
could accrete mass quickly enough to grow by a billion solar masses in
the space of a few hundred million years. We've already seen that
there's an approximate upper limit to how quickly a black hole (or,
for than matter, any object) can accrete material: the rate
corresponding to the Eddington Luminosity (known as the Eddington
rate). So, if we assume that ULAS's black hole accreted at its
Eddington limit for the previous 744 million years, what is the
maximum mass it could have accreted in that time? To calculate that,
we need to integrate the Eddington rate with respect to time, since it
increases with time as the black hole gains mass.

We'll start with the Eddington luminosity:
\begin{equation}
\label{eddlum}
L_{\rm Edd}=\frac{4\pi c G m_p}{\sigma_T}M_{\rm BH} = KcM_{\rm BH}
\end{equation}
where I've grouped all the constants (except $c$: you'll see why
later) into $K$ (see Lecture 12 notes for the meaning of all the terms
in this equation). We also know that the luminosity of an AGN is
related to its accretion rate via:
\begin{equation}
\label{lum}
L_{\rm AGN} = \eta \dot{M}_{\rm BH}c^2 
\end{equation}
and that the maximum accretion rate occurs when $L_{\rm AGN}\approx
L_{\rm Edd}$, so subbing Eqn. \ref{lum} into Eqn. \ref{eddlum} gives:
\begin{equation}
\dot{M}_{\rm BH} = \frac{Kc}{\eta c^2}M_{\rm BH} = \frac{K}{\eta c}M_{\rm BH}
\end{equation}
Since $\dot{M}_{\rm BH}=dM_{\rm BH}/dt$ we can separate the
differential equation into:
\begin{equation}
\int_{M_{\rm S}}^{M_{\rm F}}{\frac{dM_{\rm BH}}{M_{\rm BH}}} =
{\frac{K}{\eta c}}\int_{t_{\rm S}}^{t_{\rm F}}{dt}
\end{equation}
where $t_{\rm S}$ is the start time and $t_{\rm F}$ is the finish
time, and $M_{\rm S}$ and $M_{\rm F}$ are the black hole masses at
those two times, respectively. The above equation integrates to:
\begin{equation}
{\rm ln}\left(\frac{M_{\rm F}}{M_{\rm S}}\right) = \frac{K}{\eta c}(t_{\rm
  F} - t_{\rm S})
\end{equation}
or, 
\begin{equation}
\frac{M_{\rm F}}{M_{\rm S}} = {\rm exp}\left(\Delta t\frac{K}{\eta c}\right)
\end{equation}
where $\Delta t = t_{\rm F} - t_{\rm S}$. In SI units,
$K=2.1\times10^{-8}$, so assuming $\eta=0.1$ gives:
\begin{equation}
\frac{M_{\rm F}}{M_{\rm S}} = {\rm exp}\left(2.2\times10^{-8}\Delta t\right)
\end{equation}
where here $\Delta t$ is in years.\footnote{It's a coincidence that there
  are about $0.1c$ seconds in a year, so the $\eta c$ almost cancels
  out.} For $\Delta t=744$ million years, this gives
$M_{\rm F}/M_{\rm S}\approx1.2\times10^7$. Since, in the case of ULAS,
$M_{\rm F}=2\times10^9~M_\odot$, this implies that
$M_{\rm S}=170~{\rm M_\odot}$. It is just about possible that the most
massive early stars had masses of this scale, but this still presents
a challenge for black hole formation models, as it implies that ULAS's
black hole must have constantly grown at or above its Eddington limit
throughout its entire life to that point. That's quite a remarkable
feat!

\section{Possible BH formation channels}
In the previous section, we saw how forming a $\sim10^9~{\rm M_\odot}$
black hole by $z=7.088$ challenges our assumptions on how these black
holes grow and how they are ``seeded'' (i.e., what do they grow
from?). Of course, one possibility is that massive (i.e.,
$>1000~{\rm M_\odot}$, but perhaps not supermassive) black holes were
formed out of the Big Bang. These so-called ``primordial'' black holes
would then go on to accrete matter in the early Universe to create
supermassive black holes by $z\sim7$. However, little is known about
how these ``primordial'' black holes would have formed out of the Big
Bang, so there isn't much to report on them. Instead, we'll consider
three other possibilities in which black holes form from normal matter
sometime after the Big Bang. All three scenarios start-off with a
lump of gas contained within an early dark matter halo, but it's what
happens to this gas that distinguishes between the three models.

\subsection{A single massive star}
In this scenario, the lump of gas cools to form the very first
stars. This cooling happens very slowly because the gas contains no
metals, which are an effective way of radiating energy away from the
gas cloud (via their emission lines) under normal circumstance (i.e.,
in today's Universe). Because of this slow cooling, the cloud
fragments differently from today's metal-rich gas clouds, forming far
more massive stars compared to star-forming clouds today. If the most
massive of these stars ends up around 300 times more massive than the
Sun, it will collapse to form a $200~{\rm M_\odot}$ black hole which,
as we saw earlier, is just about massive enough to form a billion
solar mass black hole by $z\sim7$. However, this would require the
black hole to accrete at or above its Eddington limit for all the
intervening time.

\subsection{A single supermassive star}
Rather than forming a single massive star, in this model a single {\it
  super}massive star forms. This comes about because rather than
fragmenting into smaller stars, the gas cloud monotonically collapses
in on itself forming a single, supermassive star with masses upwards
of 10,000~${\rm M_\odot}$. With such a high mass, the pressure in the
centre of the star is so great that it quickly collapses to form a
black hole embedded in the envelope of the rest of the star. This
central black hole then quickly consumes the envelope, rapidly growing
in size to form a million solar mass black hole. This black hole can
then accrete at (average) rates much lower than the Eddington limit to
form a billion solar mass black hole by $z\sim7$.

On writing this, I can see many drawbacks with this model. Even if it
is possible for a gas cloud to monotonically collapse to form a single
star, it doesn't explain how the resulting black-hole-embedded-in-star
doesn't blast away its outer layers due to super-Eddington accretion.

\subsection{A dense cluster of merging stars}
Rather than forming a single star, most gas clouds will collapse to
form a population of stars. This scenario exploits this feature of gas
cloud collapse to produce a massive (i.e., $\sim10^3~{\rm M_\odot}$)
solar mass star. First, the pristine gas cloud collapses to form a
population of stars, but for whatever reason (again, possibly due to
the low metallicity of the cloud) this forms a far more dense cluster
of stars than we see in the local Universe. As a result of the
extremely high density of the resulting stellar cluster, the stars
soon merge to form one or more massive stars with mass of the order a
thousand stellar masses. These massive stars then rapidly age (losing
comparatively little mass via winds due to their low metallicities)
and die, forming a massive black hole of a few hundred or thousand
solar masses. This channel benefits from being less constrained by the
Eddington limit, since that only applies to accreting gas. By
contrast, merging stars are not affected by the photon pressure that
balances gravity in gas accretion.

\section{Learning objectives for Lecture 13}
With much of this lecture dedicated to the mathematical derivation of
how quickly a supermassive black hole can grow, there isn't a huge
amount to read. It's important that you understand this derivation, as
it highlights the challenges astronomers face in understanding how
supermassive black holes were already in place by $z\sim7$. As you
will likely have noted, there remains considerable uncertainties in
all our models of how seed black holes formed. This is, in part,
because of the extreme difficulties in observing these redshifts and
the corresponding lack of empirical evidence for how early, pristine gas
clouds collapse to form stars and, ultimately, black holes. It's a
fascinating area of research. 

So, after this lecture you should:
\begin{itemize}
\item have an understanding of the Eddington luminosity/rate/limit
  (they all refer to the same physical process) and why it arises
  (i.e., photon pressure on gas balancing gravity);
\item be familiar with what the results from observations of the most
  distant quasar imply for our understanding of the formation of
  supermassive black holes;
\item understand the virial technique of measuring black hole masses;
\item be familiar with the current most popular formation mechanisms
  for seed black holes in the very early Universe.
\end{itemize}

\end{document}

