\documentclass[11pt]{article}
\renewcommand{\baselinestretch}{1.05}
\usepackage{amsmath,amsthm,verbatim,amssymb,amsfonts,amscd, graphicx}
\usepackage{graphics}

\topmargin0.0cm
\headheight0.0cm
\headsep0.0cm
\oddsidemargin0.0cm
\textheight23.0cm
\textwidth16.5cm
\footskip1.0cm
\newcommand{\inchsign}{$^{\prime\prime}$}

 \begin{document}
 
\title{Lecture 1:\\Historical Introduction}
\author{Dr. James Mullaney}
\maketitle

\section{The dawn of Extragalactic research}
While galaxies external to our own Milky Way have been observed for
many centuries -- the Magellenic Clouds and the Andromeda Galaxy are
all easily observable with the naked eye -- it wasn't until the
bulding of the 100-inch telescope (i.e., $\approx$ 2.5~m diameter
aperture) on Mt Wilson in 1917 that they were systematically studied. This was
the first fully steerable large telescope and was immediately put to
use to study galaxies external to our own. 

At the time the 100-inch was built, debate still raged over whether
galaxies -- then referred to as ``spiral nebulae'' -- were internal or
extenal to the Milky Way. Edwin Hubble used the 100-inch to make the
first distance measurements to these galaxies. He did this by using
Cepheid variables, whose peak intrinsic luminosities are known to be
tightly related to the period of their variation. By measuring the
period of Cepheid variables, Hubble was able to calculate their
intrinsic luminosity ($L$), and by measuring their peak flux ($F$) was
able to determine their distance ($r$) using:
\begin{equation}
F = \frac{L}{4\pi d^2}
\end{equation}
In 1924, Hubble published his results which demonstrated that the
galaxy NGC~6822 lies at a distance of 214~kpc ($7\times10^5$ light
years; it is now known to be even more distant: 500~kpc) -- well
beyond the most distant of stars in our Milky Way -- thus confirming
its extragalactic nature.

Hubble continued his study of these (now confirmed) external galaxies,
calculating the distances to many others. He also measured the
velocity ($v$) at which these galaxies are moving away or toward the
Milky Way by measuring the shift of emission/absorption lines in their
spectra. In his 1929 study, Hubble reported that:

\begin{enumerate}
\item the vast majority of the galaxies he studied are {\it receding} from
the Milky Way, i.e., have {\it redshifted} spectral lines; and 
\item their velocity is proportional to their distance from us.
\end{enumerate}
These two points have profound implications when combined with the
Extended Copernican Principle (i.e., we -- here meaning the Milky Way
-- hold no special place in the Universe). They imply that, no matter
where you are in the Universe, (almost) all other galaxies will be
receding from you at a velocity that is proportional to its distance
from you. The only way this can be the case is if the Universe is
expanding in all directions. Thus, this was the first clear sign that
the Universe originated in a Big Bang (although it's not conclusive
proof).

A further, perhaps more practical (but no less important) use of
Hubble's 1929 result is that we no longer have to tediously measure
the periods of Cepheids to measure distances to galaxies. Instead, we
can make the far easier measurement of the galaxy's receding velocity
($v$) from its redshift (recall redshift: $z\approx v/c$ when $v\ll c$)
and use Hubble's relation calculate the distance ($r$):
\begin{equation}
d = H_0 v
\end{equation}
where $H_0$ is Hubble's constant, which is currently measured to be
$67.6^{+0.7}_{-0.6}~{\rm km~s^{-1}~Mpc^{-1}}$.

\section{Extragalactic research today}
Hubble conducted his groundbreaking research by studying just a
handful of galaxies. Today, telescope surveys of the night sky have
identified hundreds of millions of galaxies (of the trillions that are
contained within the observable Universe). These galaxies display a
vast diversity of shapes, masses, luminosities, stellar ages,
metallicities etc. and reside in environments ranging from dense
superclusters to isolated voids. The main goal of extragalactic
research is to explain this huge diversity in galaxy parameters and
the correlations between them.

\section{What we need to explain} 
While a glance at a field of galaxies may give the impression that
their properties are randomly distributed, there is, in fact, quite a
lot of underlying order and a number of correlations between their
various physical parameters. Much of extragalactic research is
focussed on explaining what causes this order. Below, I briefly
highlight some of the key properties of galaxies that we need to
explain. We will consider each of these in more detail throughout the
course.

\begin{itemize}
\item {\bf Morphologies:} Perhaps the most obvious feature of galaxies
  is their shape, whether spiral, elliptical or irregular. But, why do
  galaxies have different shapes? Why aren't they all the same shape?
  And why those particular shapes? Why not cubes, pyramids, or
  dog-shaped?
\item {\bf Luminosity function:} Not all galaxies have the same
  luminosities. When we produce a histogram of galaxy luminosities --
  known as the galaxy {\it luminosity function} -- we find that there
  are far more low luminosity galaxies than high luminosity
  galaxies. Furthermore, the galaxy luminosity function follows a
  characteristic shape, known as a Schechter function. Why don't all
  galaxies have the same luminosity? Why are there more low luminosity
  galaxies than high luminosity galaxies? And what causes the specific
  shape of the luminosity function?
\item {\bf Stellar populations:} Different galaxies are made up of
  different mixes of stellar populations. Furthermore, the mix of
  stellar populations depends strongly on the type of galaxy, with
  elliptical galaxies tending to have old, metal-rich populations,
  while spiral galaxies typically having younger stellar
  populations. Why don't all galaxies have the same mix of old and
  yound stars? And why do different types of galaxy contain different
  proportions of old and young stars?
\item {\bf Scaling relations:} As well as the link between stellar
  populations and galaxy morphologies, a number of other correlations
  exist between various galaxy properties. For example, there exists a
  tight correlation between the rotational velocity and the luminosity
  of spiral galaxies (Tully-Fisher relation). There is a similar
  relationship between luminosity and velocity dispersion for
  Elliptical galaxies. Also, the colour of ellipticals (literally,
  whether they are bluer or redder) is also correlated with
  luminosity. There even exists a tight correlation between mass of
  the supermassive black hole that reside at the centres of galaxies,
  and the mass of their host bulges. Why aren't the various properies
  of galaxies simply randomly distributed amongst each other? What
  causes these correlations to exists?
\item {\bf Clustering:} Galaxies are not distributed evenly in
  space. Instead, they clump together in superclusters, clusters and
  groups. Furthermore, various galaxy properties are related to their
  cluster environment with, for example, large ellipticals preferring
  high density regions. Why do galaxies clump together? Why are they
  distributed in the way they are? And why do their various physical
  properties correlate with environment?
\end{itemize}
And all that is simply for today's galaxies! We can also ask how all
of these different properties and correlations were different at
different epochs of the Universe, to almost 13.6 billion years ago.

\section{Studying galaxy evolution}
If we wish to explain the physical properties of today's galaxies, it
is important that we understand how they have evolved to their present
state. There are two key means of achieving this:
\begin{itemize}
\item {\bf The fossil approach:} Examining the detailed structures,
  stellar populations, abundance patterns of galaxies in the local
  Universe. For example, piecing together the star-formation histories
  of different galaxies to determine when their stars were put in place.
\item {\bf The high redshift approach:} Examining the properties of distant galaxies as a function of redshift/lookback time.
\end{itemize}
The high redshift approach works because of the finite speed of
light. As we look to increasingly distant objects, the light from
those objects has been travelling for longer and longer times. We
therefore see those objects as they were when they emitted their
light. For example, we see the Sun as it was about 8 minutes ago; the
nearest star as it was about 4 years ago; the most distant stars in
our galaxy at they were a few 10,000 years ago; the Andromeda galaxy
as it was about 2.5 million years ago; the most distant galaxies as
they were about 13 billion years ago.

If we can find galaxies in the distant Universe that we {\it believe}
will evolve into the types of galaxies that we see today, then by
studying those distant galaxies we gain insights into what today's
galaxies likely looked like at earlier times. For example, we see that
galaxies in the distant (and therefore early) Universe tend to form
stars more rapidly than nearby (i.e., today's) galaxies. So, unless we
live in a very peculiar region of the Universe, it is highly likely
that nearby galaxies -- including the Milky Way -- went through an
episode of rapid star-formation at earlier times. As it turns out,
this is backed-up by the fossil record of nearby galaxies. 

\section{Evidence of galaxy evolution}
There are a number of key pieces of evidence indicating that galaxies
have evolved significantly over the past 13 billion years. Most of
these have been established over the past half-century or so, and I
briefly describe some of them below. First, however, we'll consider a
philosophical argument that has been offered as evidence of an
evolving Universe for almost 200 years.

\subsection{Olber's paradox}
Olber's paradox concerns the question of why the sky is dark. At
first, it may seem obvious why the sky is dark. However, if the
Universe is infinite in both size and age (and contains a roughly
constant density of stars/galaxies), then the night sky should be
bright.

This can be explained if we consider a square patch of sky say, 1
degree-by-1 degree. There will be very few ``nearby'' stars -- at, say
distance $r$ -- in this small patch, but because they are nearby, they
will appear bright. As we increase $r$, the number of stars within
that patch increases as $r^2$, but their individual brightness drops
off as $r^2$. So, the sum of the light ($F_{\rm Tot}$) from all the
$N$ ($\propto r^2$) stars in the patch at a given $r$ is constant:
\begin{equation}
  F_{\rm Tot} = \sum_{i=0}^{N} \frac{L_i}{4\pi r^2} =
  N\frac{\bar{L}}{4\pi r^2} = Cr^2\frac{\bar{L}}{4\pi r^2} = k\bar{L}
\end{equation}
where $C$ is a constant of proportionality, $k = C/4\pi$ and $\bar{L}$
is the average luminosity ($L$) of the stars. Note that the second
step comes from the definition of an average:
\begin{equation}
  \bar{L} = \frac{\sum_{i=0}^{N} L_i}{N}
\end{equation}

In theory, if we sum over all $r$ in an {\it infinite} Universe, then we get an
infinite amount of light in the patch. In practice, however, stars will start to block each other out, so in fact the whole sky will have the same surface brightness as a typical star. As such, if the Universe were
infinite in age and size, we should expect the night sky to have about as the same surface brightness as the surface of the Sun! This paradox is solved if the Universe is either
finite in size, age and/or evolving.\footnote{If you're doing PHY406, this is just an alternative way to describe the same Olber's Paradox presented in that module.}

\subsection{Quasar number counts}
Quasars are extremely bright, point-like astronomical objects. They
are the brightest non-transient objects in the observable
Universe. They are now known to be powered by matter accreting onto a
supermassive black hole at the centre of a galaxy. Because of their
extremely high luminosities, we can detect them from across the whole
of the observable Universe. Indeed, for a long time, they were the
most distant observable objects.

We will cover Quasars in a lot more detail later in the course. For
now, it is suffice to say that because of their brightness, they
provided our first insights into the very distant -- and thus early --
Universe. In doing so, they provided the first robust suggestions of
an evolving Universe via their {\it number counts}.

Number counts are the flux equivalent of a luminosity function. Rather
than a histogram of luminosities, ``number count'' is the term we use
for a histogram of fluxes. It is somewhat more fundamental than a
luminosity function, however, since it doesn't require an distance
information (recall, flux, $F$, is simply what is measured by your
telescope, whereas to calculate a luminosity we also require a
distance, $L=4\pi r^2F$).

If the density of quasars within the Universe has remained constant
with time (i.e., a non-evolving Universe), it is fairly easy to
predict the numbers of quasars there should be within a given flux
range (i.e., their number count). To do this, we'll use calculus to
sum the number of quasars with a given observed flux, $F$, throughout
the entire Universe. Consider a thin spherical shell of the Universe
with radius $r$ and thickness $dr$. The number of quasars, $dN$ with
{\it observed} flux $F$ within this shell is:
\begin{equation}
\label{dN}
dN = \rho(L)\times4\pi r^2 dr
\end{equation} 
where $\rho(L)$ is the number of quasars of luminosity $L$ per unit
volume (i.e., this is the quasar luminosity function). However, to observe a
flux $F$ from a quasar of luminosity $L$, then $r$ {\it must} be given
by:
\begin{equation}
\label{eqnr}
r = \sqrt{\frac{L}{4\pi F}}
\end{equation} 
Subbing this into Eqn. \ref{dN} gives:
\begin{equation}
\label{dn2}
dN = \rho(L) \frac{L}{F} dr
\end{equation} 
To integrate this to get $N$ -- the total number of quasars of observed
flux $F$ in {\it all} shells -- we need to re-write $dr$ in terms of
$dL$ and $F$ by differentiating Eqn. \ref{eqnr}:
\begin{equation}
\label{drdl}
\frac{dr}{dL} = \frac{1}{2}\frac{1}{\sqrt{4\pi LF}}
\end{equation} 
Subbing $dr$ from Eqn. \ref{drdl} into Eqn. \ref{dn2} gives:
\begin{equation}
dN = \frac{1}{2}\frac{1}{\sqrt{4\pi}}\rho(L)\frac{L^{\frac{1}{2}}}{F^{\frac{3}{2}}}dL
\end{equation} 
Integrating both sides from $L=0$ to $L=\infty$ (corresponding to
$r=0$ to $r=\infty$), and taking all non-$L$ terms out of the integral,
gives:
\begin{equation}
N = \frac{1}{4\sqrt{\pi}}F^{-\frac{3}{2}}\int_0^\infty\rho(L)L^{\frac{1}{2}}dL
\end{equation} 
and since it's a definite integral, it just integrates to a constant,
irrespective of the form of $\rho(L)$ (provided that $\rho(L)$ is
non-evolving, and thus independent of $r$), leaving:
\begin{equation}
N \propto F^{-\frac{3}{2}}
\end{equation} 
So, if the number density of quasars of a given $L$ (i.e., the {\it
  luminosity function}, $\rho(L)$) does not evolve, then the number
detected with a given flux, $F$, is proportional to
$F^{-\frac{3}{2}}$. Usually, however, astronomers consider {\it
  differential} number counts:
\begin{equation}
\frac{dN}{dF}\propto F^{-\frac{5}{2}}
\end{equation}

When we plot the actual numbers of {\it observed} quasars as a function
of flux, we find that these numbers deviate significantly from a $F^{-\frac{3}{2}}$ law. The actual interpretation of this is not straight forward without further information (including the redshift distribution of the quasars), but it does imply that the quasar population {\it must} have evolved over the history of the Universe.

\subsection{ The ``$V$ over $V_{\rm Max}$'' test}
Later, more formal methodologies were adopted to test for an evolving
galaxy population. One of the first of these was the $V/V_{\rm Max}$
test.

Consider a galaxy of luminosity $L$, detected in a galaxy survey with
flux $F$ and measured to have a distance from us of $r$ (corresponding
to a volume $V=\frac{4}{3}\pi r^3$). Lets say that the very lowest flux
we can detect with our survey is $F_{\rm Min}$ (so, by definition
$F>F_{\rm Min}$). So, the very furthest away {\it we could have}
detected that galaxy (of luminosity $L$) is therefore:
\begin{equation}
r_{\rm Max} = \sqrt{\frac{L}{4\pi F_{\rm Min}}}
\end{equation} 
corresponding to a limiting volume of:
\begin{equation}
V_{\rm Max} = \frac{4}{3}\pi r_{\rm Max}^3
\end{equation} 
So, for every galaxy in our survey we can calculate a $V/V_{\rm Max}$
value. In a non-evolving Universe, we would expect all galaxies to be
evenly distributed, so:
\begin{equation}
\left\langle\frac{V}{V_{\rm Max}}\right\rangle = 0.5
\end{equation}
since, on average, we'd expect half of all galaxies to be within
$\frac{V_{\max}}{2}$ and the other half to be between
$\frac{V_{\max}}{2}$ and $V_{\rm Max}$. However, what we actually find
is that $V/V_{\rm Max}>0.5$ for the most luminous galaxies. This would
only happen if there were a higher density of luminous galaxies close
to the edge of the survey volume (i.e., close to $r_{Max}$), meaning
their individual $V/V_{\rm Max}$ values are greater than 0.5. This
result therefore implies there was a higher density of luminous galaxies in
the distant, i.e., early, Universe compared to today.

\subsection{Quasar luminosity function}
As our ability to measure redshifts improved, we gathered distances
for increasing numbers of quasars. With this information, astronomers
were able to measure their luminosities and, in turn, determine the
quasar luminosity function at different epochs. In doing so, they
found that the luminosity function of quasars was, indeed, different
at earlier epochs, thereby confirming the results inferred from quasar
number counts. Again, it was found that the density of quasars was
higher in the early Universe compared to today.

\subsection{Butcher-Oemler effect}
In 1978, Butcher and Oemler published a paper in which they had
measured the colours of galaxies in two distant clusters of galaxies,
one at redshift 0.39 (corresponding to 4.2 billion years ago) and
another at redshift 0.46 (corresponding to 4.8 billion years
ago). What they found was that there were a higher fraction of blue
galaxies in the cores of the distant clusters compared to similar
clusters in the local Universe (i.e., today). Blue colours in galaxies
are normally associated with ongoing star-formation, so this result
was interpreted as evidence of increased levels of star-formation in
cluster cores in the early Universe compared to today. While at the
time, they didn't know why this was the case (we've got a better idea
now; see later lectures), it was clear evidence that galaxy clusters
were different at earlier times; i.e., they have evolved.

\subsection{Galaxy number counts}
As telescopes grew in size and detectors became more senstive,
astronomers were able to start detecting fainter and fainter {\it
  galaxies}, rather than just very bright quasars. As they reached the
faintest galaxies -- some of which will low luminosity, nearby galaxies,
but most will be very distant, luminous galaxies -- they found that
their number counts were {\it not} consistent with a non-evolving
Universe. Instead, they could only match the observed counts of faint
galaxies if the density of blue, star-forming galaxies was higher in
the early Universe compared to today.

\section{Key learning objectives for L1}
\begin{itemize}
\item What two groundbreaking discoveries did Hubble make that changed
  our understanding of galaxies and the Universe?
\item List and understand the five key properties of galaxy
  populations that we need to explain.
\item Know what the terms ``luminosity function'' and ``number
  counts'' refer to, and know the difference between them.
\item Know, in general terms, what a quasar is and understand their
  importance to early investigations of galaxy evolution.
\item Understand why studying distant galaxies and quasars can provide
  insights into the early Universe.
\item Know what is meant by the ``fossil approach'' and the ``high
  redshift approach''.
\item Be able to explain Olber's paradox.
\item Know the five key pieces of evidence of galaxy evolution prior
  to the mid-nineties.

\end{itemize}

\end{document}